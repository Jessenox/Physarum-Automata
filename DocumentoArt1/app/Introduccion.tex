\section{Introducci\'on}
\label{sec:introduccion}
    % Parrafo 1
        La biolog\'ia sin lugar a dudas ha sido una fuente inagotable de inspiraci\'on para el desarrollo de algoritmos
            y t\'ecnicas de optimizaci\'on en diversas \'areas de la ciencia y la ingenier\'ia. En este este art\'iculo,
            se presenta un algoritmo de enrutamiento basado en el moho del fango \textit{Physarum polycephalum}, un organismo
            unicelular que ha demostrado habilidades excepcionales para encontrar rutas \'optimas en entornos complejos.
            Este ser ha inspirado el desarrollo de diversos algoritmos no solo de optimizaci\'on y enrutamiento, sino tambi\'en
            de modelado y simulaci\'on de diversos tipos de sistemas biol\'ogicos y f\'isicos. En particular mencionaremos 
            el trabajo de Sun et al. \cite{sun2011}, Venkatesh et al. \cite{venkatesh2018} y Elek et al. \cite{elek2019}, para 
            mencionar algunos ejemplos.
    \vskip 0.5cm
    % Parrafo 2 mencionar ejemplos de el physarum siendo usado para enrutamiento
        En el campo de la computaci\'on y la inteligencia artificial, el moho del fango ha sido utilizado para el desarrollo
            de algoritmos de enrutamiento y optimizaci\'on de rutas en diversos tipos de redes y sistemas de transporte. 
            En particular, el trabajo de Adamatzky et al. \cite{adamatzky2010} y Tero et al. \cite{tero2010} han sido pioneros
            en la aplicaci\'on de este organismo para la resoluci\'on de problemas de enrutamiento en redes de comunicaci\'on y
            transporte. Trabajos mas recientes como el de Zhang et al. \cite{zhang2019} y Li et al. \cite{li2020} han demostrado
            la eficacia de los algoritmos basados en \textit{Physarum polycephalum} para la generaci\'on de rutas eficientes y
            robustas en entornos complejos y cambiantes. En este sentido, el algoritmo propuesto en este trabajo se enfoca en la navegaci\'on 
            en entornos subterr\'aneos complejos, como cuevas y catacumbas, donde la topolog\'ia del terreno y la presencia de obst\'aculos representan un desaf\'io
            para la generaci\'on de rutas eficientes y seguras.
    \vskip 0.5cm
    % Parrafo 3 describir el problema y la solucion propuesta
        El algoritmo propuesto en este trabajo se basa en las propiedades de adaptaci\'on y exploraci\'on del moho del fango
            para determinar rutas \'optimas en entornos subterr\'aneos. El algoritmo que desarrollamos es bastante robusto y eficiente
            en la generaci\'on de rutas en entornos complejos, y ha demostrado ser capaz de adaptarse a diferentes tipos de terreno y
            obst\'aculos, por mencionar algunas de sus caracter\'isticas. Los resultados experimentales indican que el algoritmo propuesto
            mejora la eficiencia en la generaci\'on de caminos y demuestra robustez ante obst\'aculos y variaciones topogr\'aficas, ofreciendo
            nuevas herramientas para la exploraci\'on arqueol\'ogica y geol\'ogica y avanzando hacia la automatizaci\'on de la exploraci\'on subterr\'anea.
    \vskip 0.5cm
    % Parrafo 4 describir la estructura del articulo
        El resto de este art\'iculo est\'a organizado de la siguiente manera: en la secci\'on II hablamos un poco del organismo \textit{Physarum polycephalum}
            y sus propiedades biol\'ogicas que lo hacen \'util para el desarrollo de algoritmos de enrutamiento. En la secci\'on III presentamos el algoritmo
            propuesto y describimos su funcionamiento y caracter\'isticas principales. En la secci\'on IV presentamos los resultados experimentales y discutimos
            su relevancia y aplicaciones pr\'acticas. Finalmente, en la secci\'on V lo compararemos con otro algoritmo de enrutamiento (hormigas del tipo swarm intelligence)
            y wn la secci\'on VI presentamos las conclusiones y trabajos futuros. 
            