\subsubsection{Aut\'omatas celulares de 2 Dimensiones}
\label{sec:AutomatasCel2D}

    %Parrafo 1
    Los aut\'omatas celulares de 2 dimensiones son los que se usan en este trabajo terminal, por ello es necesario
        explicarlos con m\'as detalle. Primero recordando lo ya explicado en anteriores subsecciones, un aut\'omata
        celular de 2 dimensiones es una tupla $({\mathbb{Z}^{2}},S,N,f)$\ref{sec:AutomatasCelDefFormal} y este 
        necesita de 8 elementos para poder ser definido, los cuales son los siguientes:         
    \begin{enumerate}
        \item \textbf{Celdas}: Es la unidad b\'asica del aut\'omata celular. Cada celda ocupa una posici\'on en el espacio, 
            para ser representados suelen usarse cuadr\'iculas o redes, esta tiene un estado y una vecindad.
        \item \textbf{Estados}: Cada celda puede estar en uno de varios estados posibles. En los aut\'omatas celulares
            m\'as simples, cada celda puede estar en uno de dos estados posibles (0 o 1, vivo o muerto, etc), pero en los 
            aut\'omatas celulares m\'as complejos, cada celda puede estar en uno de varios estados posibles. En el caso 
            en particular del Physarum polycephalum, cada celda puede estar en uno de nueve estados posibles 
            $\mathbb{P} = \{x \in \mathbb{Z}| 0 \leq x \leq 8\}$ entonces es $S = \mathbb{P}$.
        \item \textbf{Cuadr\'icula o Red}: Las celdas estan dispuestas a lo largo del espacio euclidiano, en 
            suelen ser dispuestas en una cuadr\'icula o red, en donde cada celda ocupa una posici\'on en el espacio. Es
            n-dimensional, pero en este caso es 2-dimensional, es decir, $n = 2$.
        \item \textbf{Vecindad}: Es el conjunto de celdas que se toman en cuenta para la actualizaci\'on de una celda. En el caso de
            la vecindad de Moore que se puede ver en la secci\'on \ref{sec:Vecindario} $N = 8$
        \item \textbf{Reglas de Transici\'on}: Son un conjunto de reglas que determinan como cambia el estado de la celula en 
            funci\'on del estado actual de ella y de sus vecinos. Estas reglas se aplinan repetidamente a lo largo del tiempo,
            generalmente de manera sincr\'ona, es decir, todas las celdas se actualizan al mismo tiempo. Estas estan definidas 
            por la funci\'on de transici\'on local $f$. Ejemplificando lo anterior tenemos que en el juego de la vida de Conway
            \cite{Conway1970} donde tenemos que $C(x,y:t)$ que es la celda central y $N(x,y:t)$ que es la vecindad de Moore de la
            celda central, adem\'as tenemos que tiene $f: \{0,1\}^9 \rightarrow \{0,1\}$, entonces podemos deducir que la funci\'on de transici\'on 
            se define como:
            \begin{equation*}
                f(C(x,y:t),N(x,y:t)) = \begin{cases}
                    1 & \text{si } C(x,y:t) = 0 \text{ y } N(x,y:t) = 3 \\
                    1 & \text{si } C(x,y:t) = 1 \text{ y } N(x,y:t) = 2 \text{ o } N(x,y:t) = 3 \\
                    0 & \text{en otro caso}
                \end{cases}
            \end{equation*}
        \item \textbf{Tiempo o Generaciones}: Es el n\'umero de veces que se aplica la funci\'on de transici\'on local $f$. En este caso
            es $t \in \mathbb{Z}^{+}$. 
        \item \textbf{Condiciones Iniciales}: Antes de que el aut\'omata celular comience a evolucionar, se debe especificar el estado de
            cada celda. En este caso es $c: {\mathbb{Z}^{2}} \rightarrow S$. Estas condiciones iniciales pueden ser aleatorias o no.
        \item \textbf{Condiciones Frontera}: Son las condiciones frontera previamente mencionadas en la secci\'on \ref{sec:AutomatasCel1D}.
    \end{enumerate}
    \vskip 0.5cm
    