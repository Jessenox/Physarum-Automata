\subsubsection{Physarym Polycephalum}
    %Parrafo 1
    El Physarum Polycephalum, tambi\'en conocido como "The Blob", 
        o "La Mancha", es un protista con formas celulares diversas. El Physarum Polycephalum
        es un mixomiceto acelular, esto proviene de la etapa plasmoidal de su ciclo de vida,
        en la cual el plasmodio es un coenocito multinucleado macroscopico de color amarillo 
        brillante, formado en una red de tubos entrelazados. Esta etapa del ciclo de vida es 
        la que se utiliza para el estudio de este organismo.\cite{Dee1960}
    \vskip 0.5cm
    %Parrafo 2
    Como vimos con anterioridad los mixomicetos se dividen en dos fases, la fase vegetativa 
        y la fase reproductiva. En el caso del Physarum Polycephalum, la fase vegetativa
        es el plasmodio, una masa de protoplasma multinucleado que se desplaza por el suelo
        en busca de alimento. Si las condiciones ambientales hacen que el plasmodio se deshidrate,
        se formara un esclerocio, que es una estructura de resistencia que le permite al
        organismo sobrevivir en condiciones adversas. Una vez que las condiciones ambientales
        son favorables, el esclerocio se hidrata y se transforma en un plasmodio, reiniciando
        el ciclo de vida del Physarum Polycephalum.\cite{Dee1960}
    \vskip 0.5cm
    %Parrafo 3
    Para moverse el Physarum Polycephalum se vale del flujo del protoplasma. El intervalo entre ida y vuelta
        del flujo de protoplasma es de aproximadamente 2 minutos.\cite{Dee1960}