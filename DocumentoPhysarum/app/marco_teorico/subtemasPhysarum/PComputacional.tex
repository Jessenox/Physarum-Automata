\subsubsection{El Physarum Polycephalum visto desde la perspectiva computacional}
    % Parrafo 1
    Como se mencion\'o en la secci\'on \ref{sec:PhysarumPolycephalum}, el Physarum Polycephalum es un organismo notable 
        que ha despertado un gran inter\'es por parte de bi\'ologos y matem\'aticos debido a su notable capacidad para exhibir comportamientos 
        emergentes y resolver problemas de optimizaci\'on de manera eficiente, demostrando una gran versatilidad. Entre sus comportamientos 
        complejos se encuentran la locomoci\'on, la formaci\'on de redes adaptativas y la toma de decisiones descentralizadas.
    \vskip 0.5cm
    % Parrafo 2
    En la computaci\'on, el Physarum Polycephalum ha sido utilizado para resolver problemas de optimizaci\'on, 
        simulaci\'on de redes de transporte, y modelado de sistemas complejos. En particular, el Physarum Polycephalum
        ha sido utilizado para resolver problemas de optimizaci\'on de rutas, como el problema del camino m\'as corto,
        el problema del flujo m\'aximo, y el problema de la cobertura de sensores. Adem\'as, el Physarum Polycephalum
        ha sido utilizado para modelar sistemas complejos, como la formaci\'on de redes de transporte, la formaci\'on
        de patrones en sistemas biol\'ogicos, y la formaci\'on de estructuras en sistemas f\'isicos.
    \vskip 0.5cm
    % Parrafo 3
    FALTAN REFERENCIAS, AUN NO SE HA INVESTIGADO ESTE TEMA. DEL ESTADO DEL ARTE.