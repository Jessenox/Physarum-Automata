\subsubsection{Mixomiceto}
    Los mixomicetos, tambi\'en conociods como myxo-mycetes o mohos mucilaginosos, son 
        un grupo de fascinantes organismos unicelulares que se encuentran en el reino Protista, m\'as concretamente
        protistas ameboides o amobozoa.
        A pesar de su apariencia poco llamativa y su tama\~no microsc\'opico,
        los mixomicetos son organismos muy interesantes, por su ciclo de vida y su comportamiento
        biol\'ogico inusual.
    \vskip 0.5cm
    A diferencia de las setas y otros hongos tradicionales, los mixomicetos no forman estructuras 
        multicelulares visibles durante la mayor parte de su ciclo de vida. En cambio, existen como 
        c\'elulas individuales, generalmente microsc\'opicas, denominadas myxamoebas, que se desplazan a
        trav\'es de ambientes h\'umedos y ricos en materia org\'anica, buscando condiciones favorables para su crecimiento.
    \vskip 0.5cm
    % Parrafo 3
    Los mixomicetos toman 3 formas distintas durante el transcurso de su vida: 
    \begin{itemize}
        \item \textbf{Amoeboides}: Son c\'elulas individuales que se mueven por medio de 
            seud\'opodos o flagelos dependiendo principalmente de la cantidad de agua en el medio.
            Estas amebas se denominan \textit{myxamoebas} y son las que se encuentran en el suelo.
        \item \textbf{Plasmodio}: Es una masa de citoplasma multinucleado sin separaci\'on de 
            membranas celulares, que se mueve por medio de la contracci\'on de sus fibras de actina.
            Este plasmodio es el que se encuentra en el interior de los troncos de los \'arboles.
        \item \textbf{Cuerpo fruct\'ifero}: Es la estructura que se forma cuando el plasmodio 
            se transforma en esporas. Estas esporas son las que se encuentran en la parte superior 
            de los troncos de los \'arboles.
    \end{itemize}
    \vskip 0.5cm
    % Parrafo 4