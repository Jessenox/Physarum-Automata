\subsubsection{?`C\'omo funcionan los Cubelets?}
\label{subsubsection:comofuncionan}
    % Parrafo 1
    Al ser tecnolog\'ia de tipo privativa y no contar con un manual de usuario, la informaci\'on
        sobre el funcionamiento interno de los Cubelets es limitada. Sin embargo, podemos
        inferir su funcionamiento a partir de la informaci\'on proporcionada por la p\'agina
        oficial de Cubelets\cite{modrobotics2023} y de nuestra propia observaci\'on.
    \vskip 0.5cm
    % Parrafo 2
    Para empezar, lo fundamental es que son piezas cuadradas hechas de pl\'astico que 
        dentro de su estructura cuentan con un microcontrolador, que a su vez tiene 
        programado su propia libreria de funciones, por ejemplo, tenemos la librer\'ia 
        'cubelet.h' y 'led.h' que nos permiten controlar ciertos aspectos dentro de los cubelets.\cite{modrobotics2012}
    \vskip 0.5cm
    % Parrafo 3
    Tambi\'es cuentan con imanes en casi cada una de sus caras, hay algunas que tienen 
        imanes en todas sus caras, otras que tinen 4 imanes en sus caras y en la otra caracter
        tienen su 'funci\'on especial' como por ejemplo el cubelet sensor de luz, como su nombre lo 
        sugiere es un sensor de luz, pero tambi\'en cuenta con un im\'an en una de sus caras.
    \vskip 0.5cm
    % Parrafo 4
    Tambi\'en cabe a\~nadir que el desarrollo en cuanto a c\'odigo es bastante limitado por lo que 
        nos dejen hacer las librerias de los cubelets y las librerias \'estandar de C, sin embargo 
        se con estos pocas acciones que podemos programar en cada uno de los cubelets, podemos
        hacer que los cubelets interactuen entre s\'i provocando comportamientos complejos.