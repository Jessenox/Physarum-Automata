\section{Trabajo a futuro y conclusiones}
    El sistema desarrollado aprovecha las capacidades adaptativas y din\'amicas de los algoritmos inspirados 
        en Physarum para navegar y monitorear entornos complejos. Una de sus principales fortalezas es su 
        adaptaci\'on continua a los cambios y la exploraci\'on eficiente con m\'inima intervenci\'on humana. 
        A trav\'es de la evaluaci\'on experimental, el sistema demostr\'o robustez frente a diversos 
        obst\'aculos y variaciones ambientales, ofreciendo una herramienta valiosa para el monitoreo 
        tiempo real y la navegaci\'on en terrenos complejos, incluyendo aplicaciones en rob\'otica aut\'onoma e industrial.
    \vskip 0.5cm
    El trabajo futuro deber\'ia enfocarse en mejorar la eficiencia y escalabilidad del algoritmo. La 
        incorporaci\'on de m\'etodos avanzados de optimizaci\'on, como algoritmos gen\'eticos o la optimizaci\'on 
        por enjambre de part\'iculas, podr\'ia refinar a\'un m\'as la sintonizaci\'on de par\'ametros y acelerar 
        la generaci\'on de rutas, especialmente en escenarios de gran escala y alta complejidad.
    \vskip 0.5cm
    La integraci\'on del sistema con tecnolog\'ias avanzadas de sensores, como LiDAR, c\'amaras de 
        infrarrojos y c\'amaras multiespectrales, es fundamental para potenciar la percepci\'on del 
        entorno. Esta mejora permitir\'ia un mapeo m\'as preciso y una detecci\'on de obst\'aculos m\'as eficiente, 
        mejorando las capacidades de navegaci\'on en tiempo real y la adaptabilidad del sistema.
    \vskip 0.5cm
    Explorar nuevas \'areas de aplicaci\'on, como bosques, entornos urbanos y desiertos, demostrar\'a la versatilidad 
        del algoritmo en diversos escenarios, aumentando su impacto pr\'actico. Adem\'as, el desarrollo de modelos 
        h\'ibridos que combinen el enfoque inspirado en Physarum con algoritmos consolidados como A*, la optimizaci\'on 
        por colonia de hormigas o Dijkstra, podr\'ia generar una estrategia de navegaci\'on m\'as integral y flexible, 
        aprovechando las fortalezas complementarias de diferentes m\'etodos.
    \vskip 0.5cm
    Los avances en la visualizaci\'on y el dise\~no de interfaces de usuario ser\'an claves para mejorar la usabilidad. 
        Herramientas de visualizaci\'on avanzadas e interfaces permitir\'an a los usuarios interactuar sin problemas 
        con los datos generados y monitorear los resultados en tiempo real de manera m\'as efectiva.
    \vskip 0.5cm
    Las pruebas de campo en condiciones reales son cr\'iticas para validar la robustez del sistema. Las pruebas 
        emp\'iricas en situaciones din\'amicas e impredecibles ayudar\'an a identificar posibles mejoras y a validar 
        el rendimiento del sistema bajo condiciones de estr\'es.
    \vskip 0.5cm
    Para asegurar la confiabilidad operativa, se deben realizar simulaciones y pruebas adicionales que exploren 
        casos extremos, como la p\'erdida de se\~nal, cambios ambientales r\'apidos y obst\'aculos complejos, refinando 
        la resistencia y estabilidad del sistema en escenarios adversos.
    \vskip 0.5cm
    Finalmente, expandir el sistema a operaciones multiagente, en las que m\'ultiples instancias trabajen de manera 
        colaborativa para mapear y navegar \'areas m\'as grandes, puede mejorar significativamente la eficiencia, ofreciendo 
        soluciones escalables para exploraciones amplias y desafiantes, como el monitoreo industrial o la navegaci\'on subterr\'anea.
    \vskip 0.5cm