\subsubsection{El Physarum Polycephalum visto desde la perspectiva computacional}
    % Parrafo 1
    Como se mencion\'o en la secci\'on \ref{sec:PhysarumPolycephalum}, el Physarum Polycephalum es un organismo notable 
        que ha despertado un gran inter\'es por parte de bi\'ologos y matem\'aticos debido a su notable capacidad para exhibir comportamientos 
        emergentes y resolver problemas de optimizaci\'on de manera eficiente, demostrando una gran versatilidad. Entre sus comportamientos 
        complejos se encuentran la locomoci\'on, la formaci\'on de redes adaptativas y la toma de decisiones descentralizadas.
    \vskip 0.5cm
    % Parrafo 2
    En la computaci\'on, el Physarum Polycephalum ha sido utilizado para resolver problemas de optimizaci\'on, 
        simulaci\'on de redes de transporte, y modelado de sistemas complejos. En particular, el Physarum Polycephalum
        ha sido utilizado para resolver problemas de optimizaci\'on de rutas, como el problema del camino m\'as corto,
        el problema del flujo m\'aximo, y el problema de la cobertura de sensores. Adem\'as, el Physarum Polycephalum
        ha sido utilizado para modelar sistemas complejos, como la formaci\'on de redes de transporte, la formaci\'on
        de patrones en sistemas biol\'ogicos, y la formaci\'on de estructuras en sistemas f\'isicos.
    \vskip 0.5cm
    % Parrafo 3
    Por mencionar algunos ejemplos de aplicaciones del Physarum Polycephalum en la computaci\'on, tenemos los siguientes:
    \begin{itemize}
        \item \textbf{A physarum-inspired prize-collecting steiner tree approach to identify subnetworks for drug repositioning}:
            En el art\'iculo se detalla c\'omo un algoritmo, inspirado en el moho Physarum polycephalum, se aplica para descubrir medicamentos 
            que podr\'ian ser \'utiles en el tratamiento de enfermedades cardiovasculares. Mediante la construcci\'on de Redes de Similitud de F\'armacos 
            (DSNs), donde los nodos representan medicamentos y las conexiones reflejan similitudes entre ellos basadas en caracter\'isticas como la 
            estructura qu\'imica y los efectos terap\'euticos, cada medicamento recibe un 'premio' seg\'un su similitud con otros ya utilizados en afecciones 
            cardiovasculares. El algoritmo busca dentro de estas redes para encontrar subredes que maximicen estos premios y minimicen los costos 
            (disimilitudes), identificando as\'i grupos de f\'armacos potencialmente reutilizables para tratar enfermedades cardiovasculares. 
            Este m\'etodo propone una forma innovadora de repensar el uso de medicamentos existentes, ofreciendo un camino acelerado hacia el 
            descubrimiento de nuevas aplicaciones terap\'euticas en el campo cardiovascular. \cite{Sun2016}
        \item \textbf{A Novel Physarum-Based Ant Colony System for Solving the Real-World Traveling Salesman Problem}:
            Este art\'iculo introduce un nuevo sistema de colonia de hormigas, inspirado en el modelo matem\'atico de Physarum, 
            para abordar el problema del viajante (Traveling Salesman Problem). Este sistema ha demostrado ser m\'as eficiente y 
            robusto en comparaci\'on con los sistemas tradicionales de colonia de hormigas, algoritmos gen\'eticos y optimizaci\'on por 
            enjambre de part\'iculas. Este estudio se encuentra en un cap\'itulo del libro 'Advances in Swarm Intelligence'. \cite{Yuxiao2014} 
        \item \textbf{Composing Popular Music with Physarum polycephalum-based Memristors}: Este art\'iculo investiga el uso de Physarum polycephalum, 
            un moho mucilaginoso, como memristor para la composici\'on de m\'usica popular, presentando una colaboraci\'on entre organismos biol\'ogicos y 
            sistemas computacionales en la creaci\'on musical. Mediante una interfaz hardware-software, el estudio transforma datos musicales en voltajes 
            y viceversa, utilizando el comportamiento no lineal del moho para influir en la composici\'on. Aunque requiere ajustes para integrar las salidas 
            del organismo en las piezas musicales, este enfoque innovador abre nuevas posibilidades en la creatividad computacional y la producci\'on musical, 
            instando a m\'usicos y no expertos a explorar el c\'omputo no convencional en sus procesos creativos. El trabajo subraya el potencial de incorporar 
            tecnolog\'ias biol\'ogicas en la composici\'on musical, marcando un paso hacia la diversificaci\'on de las herramientas creativas en la m\'usica popular. \cite{NIME20_98}
        \item \textbf{Monte Carlo Physarum Machine: Characteristics of Pattern Formation in Continuous Stochastic Transport Networks}: 
            El art\'iculo introduce la M\'aquina de Physarum Monte Carlo (MCPM), un modelo avanzado para reconstruir redes de transporte a partir 
            de datos en 2D y 3D, ampliando un modelo previo de Jones sobre el moho Physarum polycephalum. La MCPM se eval\'ua por su capacidad 
            para generar estructuras complejas denominadas poliformas y se aplica en la reconstrucci\'on de la red c\'osmica, mostrando eficacia 
            con datos cosmol\'ogicos simulados y observacionales. Los autores, afiliados a la Universidad de California, Santa Cruz y la Universidad 
            Estatal de Nuevo M\'exico, exploran tambi\'en aplicaciones futuras del MCPM en diversas disciplinas. \cite{Elek2022}
        \item \textbf{Using an Artificial Physarum polycephalum Colony for Threshold Image Segmentation}: Este art\'iculo presenta un innovador 
            algoritmo basado en la simulaci\'on de una colonia de Physarum polycephalum artificial para abordar el problema de la segmentaci\'on de 
            im\'agenes por umbral, un \'area clave en el procesamiento de im\'agenes. Tradicionalmente, los algoritmos de inteligencia artificial 
            enfrentan desaf\'ios en la selecci\'on del umbral \'optimo, tendiendo a caer en \'optimos locales. La metodolog\'ia propuesta simula la 
            expansi\'on y contracci\'on de hifas artificiales para buscar soluciones \'optimas, facilitando el aprendizaje mutuo entre diferentes 
            Physarum polycephalum y mejorando la capacidad de b\'usqueda global. Utilizando la entrop\'ia de Kapur como funci\'on de ajuste, el 
            algoritmo propuesto demuestra una mayor precisi\'on y velocidad de convergencia en comparaci\'on con m\'etodos convencionales, validado 
            a trav\'es de experimentos de referencia. Este enfoque abre nuevas perspectivas en el campo del procesamiento de im\'agenes, 
            particularmente en aplicaciones de segmentaci\'on por umbral, ofreciendo una herramienta prometedora para resolver problemas complejos en esta \'area. \cite{Cai2023}
    \end{itemize}
    \vskip 0.5cm
    % Parrafo 4
    Como se puede observar, el Physarum Polycephalum ha demostrado ser una fuente de inspiraci\'on para el desarrollo de algoritmos 
        y sistemas computacionales innovadores, que han sido aplicados en una amplia variedad de campos, desde la biolog\'ia y la medicina, 
        hasta la m\'usica y la cosmolog\'ia. Su capacidad para resolver problemas complejos de manera eficiente y su versatilidad para 
        adaptarse a diferentes entornos lo convierten en un organismo \'unico y valioso para la investigaci\'on cient\'ifica y la computaci\'on.
    