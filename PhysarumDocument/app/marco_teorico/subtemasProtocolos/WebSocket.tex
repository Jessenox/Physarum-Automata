\subsubsection{WebSocket}
\label{sub:WebSocket}
    % Parrafo 1
    El protocolo de WebSocket fue desarrollado ya que el protocolo HTTP no es adecuado para aplicaciones en tiempo real, esto por que 
        el protocolo HTTP es de petici\'on-respuesta, lo que significa que el cliente debe solicitar informaci\'on al servidor y el servidor
        debe responder a la solicitud. En cambio, el protocolo WebSocket permite una comunicaci\'on bidireccional entre el cliente y el servidor,
        en otras palabras, existe una conexi\'on persistente entre el cliente y el servidor, lo que permite que el servidor env\'ie informaci\'on
        al cliente sin que este lo solicite. \cite{Kitamura2012}
    \vskip 0.5cm
        % P\'arrafo 2
        A diferencia de los protocolos de comunicaci\'on tradicionales, WebSocket permite reducir la latencia en aplicaciones que 
            requieren una actualizaci\'on constante de datos, como juegos multijugador, chats en tiempo real, o plataformas de 
            trading financiero. Esta capacidad es posible gracias al establecimiento de una conexi\'on persistente a trav\'es de 
            un \'unico canal Protocolo de Control de Transmisi\'on (Transmission Control Protocol, TCP), que permanece abierta hasta que alguna de las partes decide cerrarla. Esto reduce la sobrecarga 
            de establecer conexiones repetidas y mejora significativamente el rendimiento de las aplicaciones que requieren 
            actualizaciones constantes. \cite{RFC6455}
    \vskip 0.5cm
        % P\'arrafo 3
        El proceso de establecimiento de una conexi\'on WebSocket comienza con un "handshake" basado en HTTP, donde el cliente 
            solicita la apertura de una conexi\'on WebSocket al servidor utilizando un encabezado espec\'ifico, y el servidor 
            responde aceptando o rechazando la conexi\'on. Una vez completado el "handshake", la conexi\'on se actualiza y ambos 
            pueden intercambiar mensajes en formato binario o texto sin necesidad de seguir el ciclo de solicitud-respuesta. 
            Esto hace que WebSocket sea altamente eficiente para aplicaciones en tiempo real que manejan grandes cantidades de 
            datos o requieren baja latencia. \cite{FetteMelnikov}
    \vskip 0.5cm
        % P\'arrafo 4
        Adem\'as, WebSocket proporciona ventajas en cuanto a la reducci\'on del uso de ancho de banda. Al evitar la necesidad de 
            crear m\'ultiples conexiones y al eliminar los encabezados HTTP innecesarios en cada intercambio de mensajes, 
            se logra una transmisi\'on de datos m\'as ligera. Esto es especialmente \'util en entornos donde los recursos de red 
            son limitados, como dispositivos m\'oviles o redes con baja velocidad. \cite{WebSocketEfficiency}
    \vskip 0.5cm
        % P\'arrafo 5
        Sin embargo, aunque WebSocket ofrece muchas ventajas en t\'erminos de rendimiento y latencia, 
            su implementaci\'on puede presentar desaf\'ios de seguridad, como la exposici\'on a ataques 
            de secuestro de WebSocket entre sitios (Cross-Site WebSocket Hijacking, CSWSH) o 
            vulnerabilidades de inyecci\'on. Por esta raz\'on, es importante integrar medidas de seguridad, 
            como el uso de WebSockets sobre el protocolo de Seguridad de la Capa de Transporte 
            (Transport Layer Security, TLS), conocido como WSS para cifrar las comunicaciones, 
            y pol\'iticas adecuadas de validaci\'on del origen de las conexiones \cite{WebSocketSecurity}.
    