\subsubsection{Protocolo de Transferencia de Hipertexto (Hypertext Transfer Protocol, HTTP)} % (fold)
\label{ssub:HTTP}

    El protocolo Protocolo de Transferencia de Hipertexto (Hypertext Transfer Protocol, HTTP) es un protocolo de comunicaci\'on utilizado en 
        la World Wide Web para la transferencia de informaci\'on entre un cliente y un servidor. 
        Fue dise\~nado para ser un protocolo simple y flexible, que permitiera la transferencia 
        de datos de manera eficiente y segura. \cite{RFC2616}
    \vskip 0.5cm
        % P\'arrafo 2
        HTTP opera bajo el modelo petici\'on-respuesta, donde el cliente env\'ia una petici\'on al servidor 
            solicitando un recurso espec\'ifico, y el servidor responde con el recurso solicitado o un c\'odigo 
            de estado que indica si la petici\'on fue exitosa o no. Las peticiones y respuestas en HTTP est\'an 
            compuestas por un conjunto de encabezados y opcionalmente un cuerpo de mensaje, que contiene la 
            informaci\'on a transferir. \cite{RFC2616}
    \vskip 0.5cm
        % P\'arrafo 3
        HTTP es un protocolo sin estado, lo que significa que cada petici\'on se procesa de manera independiente, 
            sin tener en cuenta las peticiones anteriores. Esto permite que el servidor sea m\'as escalable y 
            flexible, ya que no necesita mantener un estado de sesi\'on con cada cliente. Sin embargo, esta 
            caracter\'istica tambi\'en implica que el servidor no puede recordar informaci\'on sobre el cliente 
            entre peticiones, lo que puede limitar la interacci\'on entre el cliente y el servidor. \cite{RFC2616}
    \vskip 0.5cm
        % P\'arrafo 4
        HTTP utiliza el Protocolo de Control de Transmisi\'on (Transmission Control Protocol, TCP) como su capa de transporte, lo que garantiza 
            una comunicaci\'on fiable y ordenada entre el cliente y el servidor. Las conexiones HTTP se establecen 
            mediante un \textit{handshake} entre el cliente y el servidor, donde se negocian los par\'ametros de la 
            conexi\'on, como el tipo de contenido aceptado, la codificaci\'on de transferencia, y la longitud del 
            cuerpo del mensaje. Una vez establecida la conexi\'on, el cliente y el servidor pueden intercambiar 
            mensajes de manera eficiente y segura. \cite{RFC2616}
    \vskip 0.5cm
        % P\'arrafo 5
        A pesar de su simplicidad y flexibilidad, HTTP tiene algunas limitaciones en t\'erminos de rendimiento y 
            eficiencia. Por ejemplo, HTTP es un protocolo de texto plano, lo que significa que los mensajes enviados 
            a trav\'es de HTTP deben ser codificados en texto legible por humanos, lo que puede aumentar el tama\~no 
            de los mensajes y reducir la eficiencia de la transferencia de datos. Adem\'as, HTTP no es adecuado para 
            aplicaciones en tiempo real, ya que su modelo petici\'on-respuesta puede introducir latencia en la 
            comunicaci\'on entre el cliente y el servidor. \cite{RFC2616}
    \vskip 0.5cm
        % P\'arrafo 6
        A pesar de estas limitaciones, HTTP sigue siendo uno de los protocolos de comunicaci\'on m\'as utilizados en 
            la World Wide Web, debido a su simplicidad, flexibilidad y compatibilidad con una amplia variedad de 
            plataformas y tecnolog\'ias. Sin embargo, en aplicaciones que requieren una comunicaci\'on m\'as eficiente 
            y en tiempo real, es posible que sea necesario utilizar protocolos m\'as especializados, como WebSocket o 
            el Protocolo de Telemetr\'ia de Colas de Mensajes (Message Queuing Telemetry Transport, MQTT), que est\'an dise\~nados espec\'ificamente para este prop\'osito. \cite{RFC2616}
% subsubsection HTTP (end)