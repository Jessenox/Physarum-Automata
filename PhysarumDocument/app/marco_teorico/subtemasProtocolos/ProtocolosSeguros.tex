\subsubsection{WebRTC} % (fold)
\label{ssub:WebRTC}

    %Parrafo 1
    WebRTC (Web Real-Time Communication) es un conjunto de tecnolog\'ias que permite la comunicaci\'on en tiempo real entre navegadores web 
        y aplicaciones m\'oviles. Fue desarrollado por Google en 2011 con el objetivo de facilitar la creaci\'on de aplicaciones de 
        comunicaci\'on en tiempo real, como videollamadas, conferencias web y transmisi\'on de datos en tiempo real. \cite{WebRTC}
    \vskip 0.5cm
        % P\'arrafo 2
        WebRTC se basa en varios est\'andares abiertos, como el protocolo de transporte de datos en tiempo real (RTP), el protocolo de 
            control de transmisi\'on en tiempo real (RTCP) y el protocolo de control de sesi\'on (SDP), que permiten la transmisi\'on 
            de datos en tiempo real a trav\'es de la web. Estos est\'andares est\'an dise\~nados para ser compatibles con una amplia 
            variedad de dispositivos y plataformas, lo que facilita la creaci\'on de aplicaciones de comunicaci\'on en tiempo real 
            que funcionan en diferentes entornos. \cite{WebRTC}
    \vskip 0.5cm
        % P\'arrafo 3
        Una de las caracter\'isticas m\'as importantes de WebRTC es su capacidad para establecer conexiones punto a punto entre 
            los clientes, lo que permite una comunicaci\'on directa y segura entre los usuarios sin necesidad de pasar por un 
            servidor centralizado. Esto reduce la latencia y mejora la calidad de la comunicaci\'on, ya que los datos se transmiten 
            directamente entre los clientes sin intermediarios. Adem\'as, al utilizar cifrado de extremo a extremo, WebRTC garantiza 
            la privacidad y seguridad de las comunicaciones, protegiendo los datos de posibles ataques o interceptaciones. \cite{WebRTC}
    \vskip 0.5cm
        % P\'arrafo 4
        WebRTC es compatible con una amplia variedad de dispositivos y plataformas, incluyendo navegadores web, aplicaciones m\'oviles 
            y dispositivos IoT (Internet of Things), lo que lo convierte en una soluci\'on vers\'atil para la creaci\'on de aplicaciones 
            de comunicaci\'on en tiempo real en diferentes entornos. Adem\'as, al ser un est\'andar abierto, WebRTC est\'a respaldado por 
            una amplia comunidad de desarrolladores y empresas, lo que garantiza su compatibilidad y soporte a largo plazo. \cite{WebRTC}

% subsubsection WebRTC (end)