\subsubsection{Raspberry Pi 4 Model B} % (fold)
\label{subsubsection:rasp4}
    % Parrafo 1
    Como ya vimos con anterioridad en la secci\'on \ref{ssub:comparativa}, la Raspberry Pi 4 Model B es una computadora de placa \'unica 
    (Single Board Computer, SBC) desarrollada por la Fundaci\'on Raspberry Pi. Es la cuarta generaci\'on de la serie Raspberry Pi y fue lanzada en 
        junio de 2019. Nosotros enfatizaremos sus usos que tienen para el desarrollo de un robot aut\'onomo, como el que estamos
        desarrollando en nuestro Trabajo Terminal (TT).
    \vskip 0.5cm
    % Parrafo 2
    Ventajas de la Raspberry Pi 4 Model B:
    \begin{itemize}
        \item \textbf{Alto Rendimiento:} El procesador Broadcom BCM2711 de cuatro n\'ucleos a 1.5GHz ofrede un rendimiento 
            significativamente mayor que las versiones anteriores de la Raspberry Pi, lo que permite ejecutar algoritmos 
            de control y navegaci\'on m\'as complejos.
        \item \textbf{Conectividad Avanzada:} La Raspberry Pi 4 Model B cuenta con dos puertos USB 3.0, dos puertos USB 2.0, 
            un puerto Gigabit Ethernet, conexi\'on Wi-Fi 802.11ac y Bluetooth 5.0, lo que facilita la comunicaci\'on con otros
            dispositivos, sensores y redes.
        \item \textbf{Flexibilidad de E/S:} La placa ofrece una amplia variedad de opciones de entrada y salida, incluyendo
            GPIO, HDMI, USB, Ethernet, Wi-Fi, Bluetooth, c\'amaras y pantallas, lo que permite conectar una gran variedad de
            sensores, actuadores y perif\'ericos.
        \item \textbf{Soporte de Software:} La Raspberry Pi 4 Model B es compatible con una amplia variedad de sistemas
            operativos, incluyendo Raspbian, Ubuntu, Windows 10 IoT Core y otros, lo que facilita el desarrollo de aplicaciones
            y la integraci\'on con otros dispositivos.
        \item \textbf{Bajo Costo:} La Raspberry Pi 4 Model B es una placa de bajo costo, lo que la hace accesible para
            estudiantes, aficionados y profesionales que deseen desarrollar proyectos de rob\'otica y automatizaci\'on.
    \end{itemize}
    \vskip 0.5cm
    % Parrafo 3
    Usos de la Raspberry Pi 4 Model B en rob\'otica:
    \begin{itemize}
        \item \textbf{Control de Robots:} La Raspberry Pi 4 Model B se puede utilizar para controlar robots m\'oviles, 
            drones, brazos rob\'oticos y otros dispositivos aut\'onomos, gracias a su alto rendimiento, conectividad avanzada
            y flexibilidad de E/S.
        \item \textbf{Visi\'on por Computadora:} La Raspberry Pi 4 Model B se puede utilizar para procesar im\'agenes y 
            v\'ideos en tiempo real, lo que permite a los robots detectar objetos, seguir l\'ineas, evitar obst\'aculos y 
            realizar otras tareas de visi\'on por computadora.
        \item \textbf{Aprendizaje Autom\'atico:} La Raspberry Pi 4 Model B se puede utilizar para ejecutar algoritmos de 
            aprendizaje autom\'atico y redes neuronales, lo que permite a los robots aprender de su entorno, adaptarse a 
            nuevas situaciones y mejorar su rendimiento con el tiempo.
        \item \textbf{Interacci\'on con el Entorno:} La Raspberry Pi 4 Model B se puede utilizar para interactuar con el 
            entorno f\'isico a trav\'es de sensores y actuadores, lo que permite a los robots medir la temperatura, la 
            humedad, la luz, la distancia, la velocidad y otras variables, as\'i como controlar motores, luces, pantallas 
            y otros dispositivos.
        \item \textbf{Comunicaci\'on Inal\'ambrica:} La Raspberry Pi 4 Model B se puede utilizar para comunicarse de forma 
            inal\'ambrica con otros dispositivos, sensores y redes, lo que permite a los robots enviar y recibir datos, 
            comandos y actualizaciones de forma remota.
    \end{itemize}
    \vskip 0.5cm
    % Parrafo 4
    Por todo lo anteriormente mencionado, la Raspberry Pi 4 Model B es una excelente opci\'on para el desarrollo de un robot 
        aut\'onomo, ya que ofrece un alto rendimiento, conectividad avanzada, flexibilidad de E/S, soporte de software y bajo 
        costo, lo que la hace accesible para estudiantes, aficionados y profesionales que deseen desarrollar proyectos de rob\'otica 
        y automatizaci\'on.
