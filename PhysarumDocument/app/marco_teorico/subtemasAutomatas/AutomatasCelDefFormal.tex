\subsubsection{Definici\'on formal}
    \label{sec:AutomatasCelDefFormal}
    Primero denotemos $\mathbb{Z}$ como el conjunto de los n\'umeros enteros, es decir, $\mathbb{Z} = (-\infty,-1 ,0,1, \infty)$.
        y la longitud de cualquier tupla $x$ como $|x|$. Para todas las tuplas $x$ y $y$ de la misma longitud, denotemos $x \oplus y$
        como la tupla que resulta de la suma componente a componente de $x$ y $y$, es decir, $(x \oplus y)_i = x_i + y_i$ para todo 
        $i \in \mathbb{Z}$.
    \vskip 0.5cm
    Entonces tenemos que un aut\'omata celular es una tupla $({\mathbb{Z}^{n}},S,N,f)$ tal que la n dimensi\'on es al menos 1 donde 
        $n \in \mathbb{Z}^{+}$, $S$ es un conjunto finito no vac\'io de estados, $N$ es un conjunto finito no vac\'io de vecindades 
        perteneciente a ${\mathbb{Z}^{n}}$ y $f$ es una funci\'on de transici\'on local, es decir, $f: S^N \rightarrow S$ donde
        $S^N$ representa al conjunto de todas las posibles configuraciones de vecindad en $N$.
    \vskip 0.5cm
    La configuraci\'on inicial de un aut\'omata celular es una funci\'on $c: {\mathbb{Z}^{n}} \rightarrow S$ que asigna un estado a cada celda.
        La evoluci\'on de un aut\'omata celular es una funci\'on $F: S^{{\mathbb{Z}^{n}}} \rightarrow S^{{\mathbb{Z}^{n}}}$ que asigna una configuraci\'on a la siguiente
        configuraci\'on, es decir, $F(c) = c'$, donde $c'$ es la configuraci\'on resultante de aplicar la funci\'on de transici\'on local a cada
        celda de la configuraci\'on $c$, es decir, $c'(x) = f(c|_{x+N})$ para todo $x \in {\mathbb{Z}^{n}}$, donde $c|_{x+N}$ es la restricci\'on de $c$ a la vecindad $x+N$.
        Esto tambi\'en aplica n dimensionalmente, es decir, que se podr\'ia decir que $c'(x,y) = f(c|_{(x,y)+N})$ para todo $(x,y) \in {\mathbb{Z}^{2}}$.
        Otra notaci\'on que podemos usar, y de hecho es la que utilizaremos es C(x,y:t) donde C es el centro de la vecindad, x,y son las coordenadas de la celda y t es el tiempo,
        o generaciones.
    \vskip 0.5cm
    Cabe a\~nadir que puede haber reestricciones adicionales en el conjunto de vecindades $N$ y en la funci\'on de transici\'on local $f$.
        Por ejemplo, en el caso de los aut\'omatas celulares de una dimensi\'on, el conjunto de vecindades $N$ es un conjunto de tuplas de longitud 3,
        donde la primera componente es la celda, la segunda componente es la celda de la izquierda y la tercera componente es la celda de la derecha.
        Y la funci\'on de transici\'on local $f$ es una funci\'on de 8 variables booleanas, es decir, $f: \{0,1\}^3 \rightarrow \{0,1\}$.
    \vskip 0.5cm    
    Y en el caso de los aut\'omatas celulares de dos dimensiones, el conjunto de vecindades $N$ es un conjunto de tuplas de longitud variable, 
        dependiendo del tipo de vecindad, donde la primera componente es la celda y las dem\'as componentes son las celdas vecinas. Por ejemplo, 
        en el caso de la vecindad de Moore, el conjunto de vecindades $N$ es un conjunto de tuplas de longitud 9, donde la primera componente es la celda
        $(x,y)$ el cual ser\'ia el centro de la vecindad, la segunda componente es la celda $(x-1,y-1)$, la tercera componente es la celda $(x-1,y)$,
        la cuarta componente es la celda $(x-1,y+1)$, la quinta componente es la celda $(x,y-1)$, la sexta componente es la celda $(x,y+1)$, la s\'eptima
        componente es la celda $(x+1,y-1)$, la octava componente es la celda $(x+1,y)$ y la novena componente es la celda $(x+1,y+1)$. Aqu\'i 
        $(x,y)$ es la celda central de la vecindad. Y la funci\'on de transici\'on local $f$ es una funci\'on de 512 variables booleanas, es decir,
        $f: \{0,1\}^{9} \rightarrow \{0,1\}$.
    \vskip 0.5cm
    Esta definici\'on formal de aut\'omata celular fue tomada de \cite{Codd1968}. Una vez explicada la defiinici\'on formal de aut\'omata celular,
        podemos pasar a explicar a mas detalle los aut\'omatas celulares de 2 dimensiones, los cuales son los que se usan en este trabajo terminal.
    \vskip 0.5cm