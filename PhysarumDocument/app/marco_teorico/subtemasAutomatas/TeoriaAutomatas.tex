\subsubsection{Teor\'ia de Aut\'omatas}
    \label{sec:TeoriaAutomatas}
    % Parrafo 1
    La teor\'ia de aut\'omatas es el estudio de dispositivos de c\'alculo abtractos, es decir de las m\'aquinas.\cite{Hopcroft1979}
        Estos aut\'omatas son modelos matem\'aticos fundamentales en el \'area de estudio de las ciencias de la computaci\'on, 
        son usados para entender los procesos de c\'alculo y toma de decisiones. En la teor\'ia de aut\'omatas se estudian
        los aut\'omatas finitos, los aut\'omatas con pila, las m\'aquinas de Turing, los aut\'omatas celulares, etc.
        Los aut\'omatas regulares pueden ser jerarquizados en una jerarqu\'ia de Chomsky, que es una jerarqu\'ia de lenguajes formales.
        La cual ser\'ia la siguiente\cite{Aranda2006}:
        \begin{itemize}
            \item \textbf{Tipo 3 - Gram\'aticas regulares:} Estos generan los lenguajes regulares.
                Estas se restrinjen a producciones de la forma $A \rightarrow a\gamma$ y $A \rightarrow aB$. Son 
                asociados a los aut\'omatas finitos.
            \item \textbf{Tipo 2 - Gram\'aticas libres de contexto:} Estos generan los lenguajes independientes del contexto.
                Estas se restrinjen a producciones de la forma $A \rightarrow \gamma$. Son asociados a los aut\'omatas con pila.
            \item \textbf{Tipo 1 - Gram\'aticas sensibles al contexto:} Estos generan los lenguajes sensibles al contexto.
                Estas se restrinjen a producciones de la forma ${\alpha}A{\beta} \rightarrow {\alpha}{\gamma}{\beta}$. 
                Son asociados a las m\'aquinas de Turing linealmente acotadas (significa que la cinta de la m\'aquina de Turing
                tiene un l\'imite derterminada por un cierto n\'umero entero de veces sobre la longitud de entrada).
            \item \textbf{Tipo 0 - Gram\'aticas irrestrictas:} Estos generan los lenguajes recursivamente enumerables.
                Estas se restrinjen a producciones de la forma ${\alpha}A{\beta} \rightarrow {\delta}$. Son asociados a las m\'aquinas de Turing.
        \end{itemize}
    % Parrafo 2
    \vskip 0.5cm
    En cambio los aut\'omatas celulares, aunque diferentes en estructura y aplicaci\'on a las gram\'aticas formales, 
        tambi\'en son modelos matem\'aticos fundamentales en el \'area de estudio de las ciencias de la computaci\'on
        y forman parte de la teor\'ia de aut\'omatas. Mientras que los aut\'omatas convencionales se centran en el procesamiento 
        secuencial de cadenas de s\'imbolos y operan bas\'andose en estados y transiciones claramente definidos, los aut\'omatas 
        celulares utilizan una red de c\'elulas cuyos estados evolucionan en paralelo, siguiendo reglas locales. Esta diferencia 
        fundamental en su enfoque los hace especialmente adecuados para modelar y explorar fen\'omenos que involucran procesos din\'amicos 
        y patrones espaciales. A pesar de estas diferencias, los aut\'omatas celulares se alinean con los principios fundamentales de la 
        teor\'ia de aut\'omatas en cuanto a la representaci\'on y manipulaci\'on de informaci\'on, ofreciendo una perspectiva m\'as amplia y diversa 
        sobre lo que constituye un \textit{`aut\'omata'} en el contexto de la computaci\'on y el procesamiento de informaci\'on. Su inclusi\'on en la teor\'ia 
        de aut\'omatas subraya la amplitud y la profundidad de este campo, demostrando que la teor\'ia de aut\'omatas no solo se limita a las m\'aquinas 
        y lenguajes formales tradicionales, sino que tambi\'en abarca modelos computacionales m\'as generales y vers\'atiles.