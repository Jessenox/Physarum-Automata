\subsection{Robot Propuesto} % (fold)
\label{sub:Robot Propuesto}
    Para construir el robot, se emplear\'an diversos materiales, cada uno con una funci\'on espec\'ifica para asegurar 
        la operatividad y eficiencia del dispositivo. A continuaci\'on, se detallan los materiales y sus descripciones.
    \vskip 0.5cm
    El controlador para el motor paso a paso, Nema 23, ser\'a fundamental para manejar el movimiento del robot. 
        Este componente incluye un controlador de motor a pasos que se utilizar\'a en cuatro unidades para garantizar 
        un control preciso de los motores. Los motores a pasos Nema 23 son conocidos por su precisi\'on y confiabilidad, 
        y en este caso, se utilizar\'an cuatro unidades, cada una con una placa frontal de 2.15 x 2.15 pulgadas (57 x 57 mm).
    \vskip 0.5cm
    Para la estructura del robot, se utilizar\'a una l\'amina de aluminio de calibre 14 (1.9 mm) con dimensiones de 20 cm x 40 cm, 
        que proporcionar\'a una base s\'olida y resistente. Adem\'as, se emplear\'an perfiles de aluminio 2040, espec\'ificamente de 20 x 
        40 mm y 500 mm de longitud, en dos unidades, para construir el marco del robot. Tambi\'en se utilizar\'an barras redondas 
        s\'olidas de aluminio de 2 1/2" x 12", en cuatro unidades, para reforzar la estructura y proporcionar soporte adicional.
    \vskip 0.5cm
    La movilidad del robot ser\'a posible gracias a las ruedas omnidireccionales de 6 pulgadas (152 mm) con rodamientos de silicona 
        y cubos de aleaci\'on de aluminio, en cuatro unidades. Estas ruedas permitir\'an un movimiento fluido en m\'ultiples direcciones. 
        Adem\'as, se utilizar\'an diversos tornillos y tuercas, con un paquete de 60 unidades, para ensamblar todas las partes del 
        robot de manera segura.
    \vskip 0.5cm
    Para la energ\'ia, se utilizar\'an bater\'ias de litio de 12V y 20000mAh, recargables, que proporcionar\'an la energ\'ia necesaria 
        para la operaci\'on del robot. Se utilizar\'an dos de estas bater\'ias. Un cargador de bater\'ias de 14V y 20A, espec\'ifico 
        para bater\'ias de litio de 12V, ser\'a empleado para mantener las bater\'ias recargadas y operativas.
    \vskip 0.5cm
    La electr\'onica del robot incluir\'a una Raspberry Pi 4 B, que actuar\'a como el cerebro del dispositivo, gestionando 
        las operaciones y los datos recibidos. Un sensor de distancia LIDAR, modelo DTOF STL27L, permitir\'a al 
        robot detectar obst\'aculos y medir distancias con precisi\'on, utilizando un l\'aser LIDAR 360\degree con bus UART 
        y un rango de 895 - 915 NM (tipo 905).
    \vskip 0.5cm
    Para la visi\'on, se emplear\'a una c\'amara de visi\'on nocturna de luz infrarroja de 5MP, con un \'angulo de visi\'on de 
        130-220 grados, espec\'ifica para la Raspberry Pi 4B. Tambi\'en se incluir\'a un convertidor auto Boost Buck de CD-CD, 
        de 5A y rango de 5V-30V, que ayudar\'a a gestionar las diferentes necesidades de voltaje de los componentes 
        electr\'onicos del robot.
    \vskip 0.5cm
    Finalmente, se utilizar\'a una l\'amina de acr\'ilico transparente de 6 mm, con dimensiones de 60 x 120 cm, 
        para crear cubiertas protectoras y otras partes visibles del robot. Este material es ideal por su 
        durabilidad y resistencia a impactos.
    \vskip 0.5cm
    Estos componentes, cuidadosamente seleccionados, se ensamblar\'an para crear un robot funcional, 
        robusto y vers\'atil, capaz de realizar diversas tareas con eficiencia y precisi\'on.
% subsection Robot Propuesto (end)