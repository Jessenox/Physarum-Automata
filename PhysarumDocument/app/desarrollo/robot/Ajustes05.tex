\subsubsection{Ajustes de c\'odigo en funci\'on de las pruebas de carga 2} % (fold)
%Aqui se mejoro el algoritmo de seguir ruta, y a\~nadio un nuevo apartado gr\'afico para mostrar las dimensiones del robot en 2d 
% esto por que luego al manejarlo chocaban
\label{sub:Ajustes de c\'odigo en funci\'on de las pruebas de carga 2}
    Luego de realizar las pruebas de carga 2, se identificaron algunos problemas en el c\'odigo del robot, 
        espec\'ificamente en el algoritmo de seguimiento de ruta. Se observ\'o que el robot ten\'ia dificultades 
        para seguir la ruta de manera precisa, lo que resultaba en movimientos err\'aticos y desviaciones 
        significativas de la trayectoria deseada. Para solucionar este problema, se realizaron los siguientes ajustes 
        en el c\'odigo del robot:
    \vskip 0.5cm
    \begin{itemize}
        \item Se mejor\'o el algoritmo de seguimiento de ruta para que el robot pudiera seguir la trayectoria de manera m\'as precisa. 
            Se ajustaron los par\'ametros de control de velocidad y direcci\'on para minimizar las desviaciones y los errores en el 
            seguimiento de la ruta.
        \item Se a\~nadi\'o un nuevo apartado gr\'afico en la interfaz de usuario para mostrar las dimensiones del robot en 2D, 
            lo que permiti\'o al operador visualizar
            la posici\'on y orientaci\'on del robot en relaci\'on con la ruta planificada. Esto facilit\'o la supervisi\'on y el
            control del robot durante la operaci\'on, y ayud\'o a prevenir colisiones y desviaciones no deseadas.
    \end{itemize}
    \vskip 0.5cm
    Estos ajustes permitieron mejorar el desempe\~no del robot y garantizar un seguimiento preciso de la ruta, 
        lo que result\'o en un funcionamiento m\'as eficiente y fiable del sistema en general.
      \begin{lstlisting}[language={C++}, caption={Primer ajuste de c\'odigo}, label={Script}]
        #include "CYdLidar.h"
        #include <SFML/Graphics.hpp>
        #include <opencv2/opencv.hpp>
        #include <iostream>
        #include <cstdlib>
        #include <cstdio>
        #include <pigpio.h>
        #include <string>
        #include <map>
        #include <vector>
        #include <atomic>
        #include <thread>
        #include <random>
        #include <sys/socket.h>
        #include <sys/un.h>
        #include <unistd.h>
        #include <cstring>
        #include <pthread.h>
        #include <vector>
        #include <mutex>
        #define PI 180.f
        
        using namespace std;
        using namespace ydlidar;
        
        // Pines y configuracin para los motores
        const int PWM_PINS[] = {13, 19, 18, 12};  // Pines PWM para los motores
        const int DIR_PINS[] = {5, 6, 23, 24};    // Pines de direccin para los motores
        const int SENSOR_PROFUNIDAD_PINS[] = {2, 3};
        int frequency = 400; // Frecuencia inicial
        const sf::Vector2f robotFixedPosition(110, 110);
        std::atomic<bool> is_running(true);
        std::atomic<bool> is_manual_mode(true);
        
        void setMotorSpeed(int motor, int frequency) {
            if (motor >= 0 && motor < 4) {
                gpioSetPWMfrequency(PWM_PINS[motor], frequency);
            }
        }
        
        void setMotorDirection(int motor, int direction) {
            if (motor >= 0 && motor < 4) {
                gpioWrite(DIR_PINS[motor], direction);
            }
        }
        
        void stopMotors() {
            for (int i = 0; i < 4; ++i) {
                gpioPWM(PWM_PINS[i], 0);
            }
        }
        
        void moveForward() {
            for (int i = 0; i < 4; ++i) {
                gpioPWM(PWM_PINS[i], 128);  // Establecer ciclo de trabajo al 50%
            }
            setMotorDirection(0, 0); // Motor 1
            setMotorDirection(2, 0); // Motor 3
            setMotorDirection(1, 1); // Motor 2
            setMotorDirection(3, 1); // Motor 4
        }
        
        void moveBackward() {
            for (int i = 0; i < 4; ++i) {
                gpioPWM(PWM_PINS[i], 128);  // Establecer ciclo de trabajo al 50%
            }
            setMotorDirection(0, 1); // Motor 1
            setMotorDirection(2, 1); // Motor 3
            setMotorDirection(1, 0); // Motor 2
            setMotorDirection(3, 0); // Motor 4
        }
        
        void turnLeft() {
            std::cout << "Turning left" << std::endl;
            for (int i = 0; i < 4; ++i) {
                gpioPWM(PWM_PINS[i], 128);  // Establecer ciclo de trabajo al 50%
            }
            setMotorDirection(0, 1); // Motor 1
            setMotorDirection(2, 1); // Motor 3
            setMotorDirection(1, 1); // Motor 2
            setMotorDirection(3, 1); // Motor 4
        }
        
        void turnRight() {
            std::cout << "Turning right" << std::endl;
            for (int i = 0; i < 4; ++i) {
                gpioPWM(PWM_PINS[i], 128);  // Establecer ciclo de trabajo al 50%
            }
            setMotorDirection(0, 0); // Motor 1
            setMotorDirection(2, 0); // Motor 3
            setMotorDirection(1, 0); // Motor 2
            setMotorDirection(3, 0); // Motor 4
        }
        
        void increaseSpeed() {
            if (frequency < 2000) {
                frequency += 100;
                if (frequency > 2000) frequency = 2000;
                for (int i = 0; i < 4; ++i) {
                    setMotorSpeed(i, frequency);
                }
            }
        }
        
        void decreaseSpeed() {
            if (frequency > 400) {
                frequency -= 100;
                if (frequency < 400) frequency = 400;
                for (int i = 0; i < 4; ++i) {
                    setMotorSpeed(i, frequency);
                }
            }
        }
        
        
        sf::Color getPointColor(float distance, float maxRange) {
            float ratio = distance / maxRange;
            return sf::Color(255 * (1 - ratio), 255 * ratio, 0); // Color de rojo a verde
        }
        
        
        std::vector<sf::Vector2f> routePoints;
        bool isDrawing = false;  // Bandera para rastrear si el raton esta presionado
        
        
        
        
        void drawLiDARPoints(sf::RenderWindow &window, const LaserScan &lidarPoints, const sf::Vector2f &robotPosition, float scale, float max_range) {
            for (const auto& point : lidarPoints.points) {
                float x = point.range * cos(point.angle);
                float y = point.range * sin(point.angle);
        
                float adjustedX = robotFixedPosition.x + (x * scale) - robotPosition.x;
                float adjustedY = robotFixedPosition.y - (y * scale) - robotPosition.y;  // Invertir Y para la pantalla
        
                if (point.range > 0.05) {  // Excluir puntos cercanos al centro
                    sf::CircleShape lidarPoint(2);  // Tamao del punto
                    lidarPoint.setPosition(adjustedX, adjustedY);
                    lidarPoint.setFillColor(getPointColor(point.range, max_range));  // Color basado en la distancia
        
                    window.draw(lidarPoint);
                }
            }
        }
        
        
        
        
        
        
        void handleMouseClick(const sf::Event &event, const sf::RectangleShape &minimap) {
            // Verificar si el raton esta presionado
            if (event.type == sf::Event::MouseButtonPressed && event.mouseButton.button == sf::Mouse::Left) {
                isDrawing = true;  // Marcar que estamos dibujando
        
            // Verificar si el raton fue soltado
            } else if (event.type == sf::Event::MouseButtonReleased && event.mouseButton.button == sf::Mouse::Left) {
                isDrawing = false;  // Dejar de dibujar
            }
        }
        
        void updateRoute(const sf::RectangleShape &minimap, sf::RenderWindow &window) {
            // Si el raton esta presionado, agregar puntos a la ruta
            if (isDrawing) {
                // Obtener la posicion actual del raton
                sf::Vector2i mousePosition = sf::Mouse::getPosition(window);
                sf::Vector2f clickPosition(static_cast<float>(mousePosition.x), static_cast<float>(mousePosition.y));
        
                // Verificar si la posicion del raton esta dentro del minimapa
                if (minimap.getGlobalBounds().contains(clickPosition)) {
                    sf::Vector2f relativePosition = clickPosition - minimap.getPosition();
                    routePoints.push_back(relativePosition);
                    std::cout << "Dibujando punto: (" << relativePosition.x << ", " << relativePosition.y << ")" << std::endl;
                }
            }
        }
        
        
        
        void drawRoute(sf::RenderWindow &window, const std::vector<sf::Vector2f> &routePoints, const sf::Vector2f &robotPosition) {
            if (routePoints.size() > 1) {
                for (size_t i = 0; i < routePoints.size() - 1; ++i) {
                    // Dibujar las lneas de la ruta, ajustando las coordenadas en funcin de la posicin del robot
                    sf::Vertex line[] = {
                        sf::Vertex(robotFixedPosition + (routePoints[i] - robotPosition), sf::Color::Red),  // Ajustar posicin en funcin del robot
                        sf::Vertex(robotFixedPosition + (routePoints[i + 1] - robotPosition), sf::Color::Red)
                    };
                    window.draw(line, 2, sf::Lines);
                }
            }
        }
        
        
        
        
        
        // Ajustar la referencia para que el norte sea -1.57 rad (sur en coordenadas cartesianas es 0 rad)
        float adjustAngleForNorth(float angle) {
            float adjustedAngle = angle - (-M_PI / 2);  // Ajuste de 90 (norte en -90)
            
            // Asegurarse de que el angulo este dentro de los limites de -PI a PI
            if (adjustedAngle > M_PI) {
                adjustedAngle -= 2 * M_PI;
            } else if (adjustedAngle < -M_PI) {
                adjustedAngle += 2 * M_PI;
            }
            
            return adjustedAngle;
        }
        
        
        
        
        
        
        
        void rotateTowardsPoint(float currentAngle, float targetAngle) {
            float angleDifference = targetAngle - currentAngle;
        
            // Asegurate de que el angulo este entre -PI y PI
            if (angleDifference > M_PI) {
                angleDifference -= 2 * M_PI;
            } else if (angleDifference < -M_PI) {
                angleDifference += 2 * M_PI;
            }
        
            // Gira en la direccion correcta
            if (angleDifference > 0) {
                turnRight();  // Girar hacia la derecha
            } else {
                turnLeft();  // Girar hacia la izquierda
            }
        
            // Esperar hasta que el robot este alineado
            std::this_thread::sleep_for(std::chrono::milliseconds(static_cast<int>(std::fabs(angleDifference) * 1000)));
            
            stopMotors();
        }
        
        
        
        
        
        
        void moveTowardsPoint(float distance) {
            moveForward();  // Mover el robot hacia adelante
        
            // Esperar un tiempo proporcional a la distancia
            std::this_thread::sleep_for(std::chrono::milliseconds(static_cast<int>(distance * 1000)));
            
            stopMotors();  // Detener el robot
        }
        
        
        
        
        void moveTo(sf::Vector2f point, sf::Vector2f &robotPosition, float &currentAngle) {
            // Calcular la diferencia en X e Y
            float deltaX = point.x - robotPosition.x;
            float deltaY = robotPosition.y - point.y;  // Invertir la Y para la pantalla
        
            // Calcular la distancia al punto
            float distanceToPoint = sqrt(deltaX * deltaX + deltaY * deltaY);
        
            // Si la distancia es mayor a un umbral, mover el robot
            const float threshold = 5.0f;  // Tolerancia para considerar que el punto ha sido alcanzado
            if (distanceToPoint > threshold) {
                // Calcular el angulo hacia el punto de destino
                float targetAngle = atan2(deltaY, deltaX);
                // Ajustar el angulo para que el norte sea -90
                targetAngle = adjustAngleForNorth(targetAngle);
        
                // Girar hacia el angulo correcto
                rotateTowardsPoint(currentAngle, targetAngle);  // Funcin para girar hacia el ngulo
        
                // Mover hacia el punto
                moveTowardsPoint(distanceToPoint);
                
                // Actualizar la posicion y el ngulo del robot
                robotPosition = point;
                currentAngle = targetAngle;
            }
        }
        
        
        
        void followRoute(sf::RenderWindow &window, std::vector<sf::Vector2f> &routePoints, const LaserScan &lidarPoints, const sf::Vector2f &minimapPosition, float max_range) {
            sf::Vector2f robotPosition(110, 110);  // Posicin inicial del robot en el minimapa
            float currentAngle = -M_PI / 2;  // ngulo inicial del robot (norte en -90)
        
            while (!routePoints.empty()) {
                sf::Vector2f targetPoint = routePoints.front();
        
                moveTo(targetPoint, robotPosition, currentAngle);
        
                routePoints.erase(routePoints.begin());
        
                drawRoute(window, routePoints, minimapPosition);
                drawLiDARPoints(window, lidarPoints, robotPosition, 25.0f, max_range);
            }
        }
        
        
        
        
        
        
        
        
        
        
        int contadorObstaculoTotal = 0;
        int obstaculoHola = 0;
        int obsRelativo = 0;
        int idTipoObstaculo = 0;
        void randomMovement(CYdLidar &laser) {
            std::random_device rd;
            std::mt19937 gen(rd());
            std::discrete_distribution<> dist({0, 5}); // Distribucin para la probabilidad de movimiento
        
            const float FRONT_MIN_ANGLE = -10.0f * (M_PI / 180.0f); // -15 grados en radaianes
            const float FRONT_MAX_ANGLE = 10.0f * (M_PI / 180.0f);  // 15 grados en radianes
            const float DETECTION_RADIUS = 0.25f; // 35 cm
        
            std::vector<float> previousScanPoints;
        
            while (is_running) {
                if (!is_manual_mode) {
                    LaserScan scan;
                    if (laser.doProcessSimple(scan)) {
                        bool obstacle_detected = false;
                        std::vector<float> currentScanPoints;
                        
                        for (const auto &point : scan.points) {
                            // Convertir el ngulo del punto al ngulo relativo al "sur" del robot
                            float adjusted_angle = point.angle + PI;
                            //std::cout << "Er " << point.range << std::endl;
                            // Verificar si el punto est dentro del rango frontal de 30
                            //std::cout << point.angle << std::endl;
                            if (point.range > 0 && point.range < 0.30 && ( point.angle <= -0.5235f  && point.angle >= -2.617f)) { // Norte
                                obstacle_detected = true;
                                currentScanPoints.push_back(point.range);
                                idTipoObstaculo = 1;
                                std::cout << "Obstacle distance N: " << (float)point.range  << " Y en el angulo  "<<point.angle << std::endl;
                                break;
                            } else if(point.range > 0 && point.range < 0.25 && (point.angle <= 0.872f  && point.angle>= -0.5235f)){ // Este
                                obstacle_detected = true;
                                currentScanPoints.push_back(point.range);
                                idTipoObstaculo = 2;
                                std::cout << "Obstacle distance E: " << (float)point.range  << " Y en el angulo  "<<point.angle << std::endl;
                                break;
                            }else if(point.range > 0 && point.range < 0.45 && (point.angle <= 2.26f && point.angle >= 0.872f)){ // Sur
                                obstacle_detected = true;
                                currentScanPoints.push_back(point.range);
                                idTipoObstaculo = 3;
                                std::cout << "Obstacle distance S:" << (float)point.range  << " Y en el angulo  "<<point.angle << std::endl;
                                break;
                            }else if (point.range > 0 && point.range < 0.25 && (point.angle <= -2.61f  || point.angle >= 2.27f)){ // Oeste
                                obstacle_detected = true;
                                currentScanPoints.push_back(point.range);
                                idTipoObstaculo = 4;
                                std::cout << "Obstacle distance O:" << (float)point.range  << " Y en el angulo  "<<point.angle << std::endl;
                                break;
                            }
                            
                            
                        }
                        
                        if (obstacle_detected) {
                            // Si detecta un obstculo, retrocede por 4 segundos
                            contadorObstaculoTotal++;
                            std::cout << "Obs: " << contadorObstaculoTotal << std::endl;
                            //obsRelativo++;
                            stopMotors();
                            //Luego gira aleatoriamente a la izquierda o derecha
                            switch(idTipoObstaculo) {
                                case 1: {
                                    int turn = dist(gen) % 2;
                                    moveBackward();
                                    std::this_thread::sleep_for(std::chrono::seconds(5));
                                    if (turn == 0) {
                                        turnLeft();
                                        std::this_thread::sleep_for(std::chrono::seconds(5));
                                    } else {
                                        turnRight();
                                        std::this_thread::sleep_for(std::chrono::seconds(5));
                                    }
                                    std::this_thread::sleep_for(std::chrono::seconds(2));
                                    stopMotors();
                                }
                                break;
                                
                                case 2:
                                    turnLeft();
                                    std::this_thread::sleep_for(std::chrono::seconds(5));
                                    std::this_thread::sleep_for(std::chrono::seconds(2));
                                    stopMotors();
                                break;
                                
                                case 3:// este es el 4 y el de abajo el 3
                                    moveForward();
                                    std::this_thread::sleep_for(std::chrono::seconds(2));
                                    stopMotors();
                                    
                                break;
                                
                                case 4:
                                    turnRight();
                                    std::this_thread::sleep_for(std::chrono::seconds(7));
                                    stopMotors();
                                break;
                                
                                default: {
                                    int turn = dist(gen) % 2;
                                    if (turn == 0) {
                                        turnLeft();
                                        std::this_thread::sleep_for(std::chrono::seconds(5));
                                    } else {
                                        turnRight();
                                        std::this_thread::sleep_for(std::chrono::seconds(5));
                                    }
                                    std::this_thread::sleep_for(std::chrono::seconds(2));
                                    stopMotors();
                                }
                                break;
                            }
        
                            
                            if(!previousScanPoints.empty() && previousScanPoints.size() == currentScanPoints.size()){
                                    bool mismoObs = true;
                                    for(size_t i = 0; i < currentScanPoints.size(); ++i){
                                        if(fabs(currentScanPoints[i] - previousScanPoints[i]) > 0.05){
                                                mismoObs = false;
                                                break;
                                        }
                                    }
                                    if(mismoObs){
                                        obsRelativo++;
                                    }else{
                                        obsRelativo = 0;
                                    }
                            }
                            
                            previousScanPoints = currentScanPoints;
                            
                            if((obsRelativo > 3)){
                                //obsRelativo = 0;
                                moveBackward();
                                std::this_thread::sleep_for(std::chrono::seconds(3));
                                stopMotors();
                                
                                
                                switch(idTipoObstaculo) {
                                    case 1: {
                                        int turn = dist(gen) % 2;
                                        moveBackward();
                                        std::this_thread::sleep_for(std::chrono::seconds(5));
                                        if (turn == 0) {
                                            turnLeft();
                                            std::this_thread::sleep_for(std::chrono::seconds(5));
                                        } else {
                                            turnRight();
                                            std::this_thread::sleep_for(std::chrono::seconds(5));
                                        }
                                        std::this_thread::sleep_for(std::chrono::seconds(2));
                                        stopMotors();
                                    }
                                    break;
                                    
                                    case 2:
                                        turnLeft();
                                        std::this_thread::sleep_for(std::chrono::seconds(5));
                                        std::this_thread::sleep_for(std::chrono::seconds(2));
                                        stopMotors();
                                    break;
                                    
                                    case 3:
                                        turnRight();
                                        std::this_thread::sleep_for(std::chrono::seconds(7));
                                        stopMotors();
                                    break;
                                    
                                    case 4:
                                        moveForward();
                                        std::this_thread::sleep_for(std::chrono::seconds(2));
                                        stopMotors();
                                    break;
                                    
                                    default: {
                                        int turn = dist(gen) % 2;
                                        if (turn == 0) {
                                            turnLeft();
                                            std::this_thread::sleep_for(std::chrono::seconds(5));
                                        } else {
                                            turnRight();
                                            std::this_thread::sleep_for(std::chrono::seconds(5));
                                        }
                                        std::this_thread::sleep_for(std::chrono::seconds(2));
                                        stopMotors();
                                    }
                                    break;
                                }
        
                                
                                
                            }
                        } else {
                            moveForward();
                            // Si no hay obstculo, elige un movimiento aleatorio
                            //
                            //int move = dist(gen);
                            //switch (move) {
                              //  case 0: moveBackward(); break;
                              //  case 1: turnRight(); break;
                              //  case 2: turnLeft(); break;
                              //  case 3: moveForward(); break;
                            //}
                        }
                    }
                    
                    std::this_thread::sleep_for(std::chrono::milliseconds(500));
                    //stopMotors();
                }
                std::this_thread::sleep_for(std::chrono::milliseconds(100));
            }
        }
        
        
        // Definir las dimensiones del robot en el minimapa
            const float frontDistance = 15.0f;   // cm hacia adelante
            const float backDistance = 35.0f;    // cm hacia atrs
            const float rightDistance = 21.0f;   // cm hacia la derecha
            const float leftDistance = 21.0f;    // cm hacia la izquierda
        
            // Escalar las distancias al minimapa (considera ajustar el factor de escala segn lo necesites)
            float scale = 0.4f;  // Cambia este valor si necesitas ajustar el tamao del robot en el minimapa
        
        
        void drawRobot(sf::RenderWindow &window, const sf::Vector2f &robotPosition, const sf::RectangleShape &minimap) {
            
        
            float width = (leftDistance + rightDistance) * scale;
            float height = (frontDistance + backDistance) * scale;
        
            // Dibujar el rectngulo representando al robot
            sf::RectangleShape robotShape(sf::Vector2f(width, height));
            robotShape.setFillColor(sf::Color(100, 100, 255, 150));  // Color azul claro con algo de transparencia
        
            // Ajustar el origen del rectngulo al punto donde est el LiDAR, es decir, al centro del robot
            robotShape.setOrigin(leftDistance * scale, backDistance * scale);
        
            // Establecer la posicin del rectngulo del robot en el minimapa, centrado en la posicin del LiDAR
            robotShape.setPosition(105, 105); // Usamos la misma posicin del LiDAR para centrar el robot
        
            window.draw(robotShape);
        }
        
        
        
        
        void moveAlongRoute(const std::vector<int> &route) {
            // Las celdas est\'an numeradas de 1 a N como en la imagen proporcionada.
            std::map<int, std::pair<int, int>> cellCoordinates = {
                {1, {0, 0}}, {2, {0, 1}}, {3, {0, 2}},
                {4, {1, 0}}, {5, {1, 1}}, {6, {1, 2}},
                {7, {2, 0}}, {8, {2, 1}}, {9, {2, 2}},
                // Contin\'ua con el resto de las celdas de la cuadr\'icula...
                {10,{3, 0}}, {11,{3, 1}}, {12,{3, 2}},
                {13,{4, 0}}, {14,{4, 1}}, {15,{4, 2}},
                {16,{5, 0}}, {17,{5, 1}}, {18,{5, 2}},
                {19,{6, 0}}, {20,{6, 1}}, {21,{6, 2}},
                {22,{7, 0}}, {23,{7, 1}}, {24,{7, 2}}
            };
        
            std::cout << "Moviendo a lo largo de la ruta..." << std::endl;
            std::cout << "La ruta es: " << std::endl;
            for (int cell : route) {
                std::cout << cell << " ";
            }
            std::cout << std::endl;
            for (size_t i = 0; i < route.size() - 1; ++i) {
                int currentCell = route[i];
                int nextCell = route[i + 1];
                std::cout << "Current cell: " << currentCell << " Next cell: " << nextCell << endl;
                auto currentCoord = cellCoordinates[currentCell];
                auto nextCoord = cellCoordinates[nextCell];
        
                // Calculamos la direcci\'on en la que el robot debe moverse.
                int deltaX = nextCoord.first - currentCoord.first;
                int deltaY = nextCoord.second - currentCoord.second;
                is_manual_mode = false;
                std::cout << "Delta X: " << deltaX << " Delta Y: " << deltaY << endl;
                if (deltaX == 0 && deltaY == 1) {
                    // Movimiento hacia la derecha.
                    stopMotors();
                    std::cout << "Girando hacia la derecha y avanzando..." << std::endl;
                    turnRight();
                    std::this_thread::sleep_for(std::chrono::milliseconds(4500));  // Girar por 4.35 segundos.
                    stopMotors();
                    std::this_thread::sleep_for(std::chrono::milliseconds(500));   // Pausa antes de avanzar.
                    moveForward();
                    std::this_thread::sleep_for(std::chrono::milliseconds(5000));  // Avanzar por 5 segundos.
                    stopMotors();
                } else if (deltaX == 0 && deltaY == -1) {
                    // Movimiento hacia la izquierda.
                    stopMotors();
                    std::cout << "Girando hacia la izquierda y avanzando..." << std::endl;
                    turnLeft();
                    std::this_thread::sleep_for(std::chrono::milliseconds(4350));  // Girar por 4.35 segundos.
                    stopMotors();
                    std::this_thread::sleep_for(std::chrono::milliseconds(500));   // Pausa antes de avanzar.
                    moveForward();
                    std::this_thread::sleep_for(std::chrono::milliseconds(5000));  // Avanzar por 5 segundos.
                    stopMotors();
                } else if (deltaX == 1 && deltaY == 0) {
                    // Movimiento hacia adelante (eje X positivo).
                    stopMotors();
                    std::cout << "Avanzando hacia adelante..." << std::endl;
                    moveForward();
                    std::this_thread::sleep_for(std::chrono::milliseconds(5000));  // Avanzar por 5 segundos.
                    stopMotors();
                } else if (deltaX == -1 && deltaY == 0) {
                    // Movimiento hacia atr\'as (eje X negativo).
                    std::cout << "Retrocediendo..." << std::endl;
                    moveBackward();
                    std::this_thread::sleep_for(std::chrono::milliseconds(5000));  // Retroceder por 5 segundos.
                    stopMotors();
                } else {
                    std::cerr << "Movimiento no soportado entre las celdas: " << currentCell << " y " << nextCell << std::endl;
                }
                std::cout << "Movimiento completado." << std::endl;
                stopMotors();
                // Pausa entre movimientos para evitar sobrecargar los motores.
                std::this_thread::sleep_for(std::chrono::milliseconds(1000));
            }
        }
        
        
        
        
        
        
        
        
        
        void drawGrid(sf::RenderWindow &window, const sf::RectangleShape &minimap) {
            const float cellSizeCm = 50.0f;  // Tama\~no de cada celda en cent\'imetros
            const float scaleFactor = 10.0f; // Relaci\'on 1 cm = 10 p\'ixeles en el minimapa
            const float cellSizePx = cellSizeCm * scaleFactor;  // Tama\~no de cada celda en p\'ixeles
        
            // Obtener la posici\'on y tama\~no del minimapa
            sf::Vector2f minimapPosition = minimap.getPosition();
            sf::Vector2f minimapSize = minimap.getSize();
        
            // Calcular el origen del grid para mantenerlo fijo respecto al robot centrado en (105, 105)
            float originX = minimapPosition.x - fmod(minimapPosition.x + 105.0f, cellSizePx);
            float originY = minimapPosition.y - fmod(minimapPosition.y + 105.0f, cellSizePx);
        
            // Dibujar las l\'ineas verticales de la cuadr\'icula
            for (float x = originX; x <= minimapPosition.x + minimapSize.x; x += cellSizePx) {
                sf::Vertex verticalLine[] = {
                    sf::Vertex(sf::Vector2f(x, minimapPosition.y), sf::Color(150, 150, 150, 200)),
                    sf::Vertex(sf::Vector2f(x, minimapPosition.y + minimapSize.y), sf::Color(150, 150, 150, 200))
                };
                window.draw(verticalLine, 2, sf::Lines);
            }
        
            // Dibujar las l\'ineas horizontales de la cuadr\'icula
            for (float y = originY; y <= minimapPosition.y + minimapSize.y; y += cellSizePx) {
                sf::Vertex horizontalLine[] = {
                    sf::Vertex(sf::Vector2f(minimapPosition.x, y), sf::Color(150, 150, 150, 200)),
                    sf::Vertex(sf::Vector2f(minimapPosition.x + minimapSize.x, y), sf::Color(150, 150, 150, 200))
                };
                window.draw(horizontalLine, 2, sf::Lines);
            }
        }
        
        
        
        
        
        
        
        int main() {
            // Inicializar pigpio
            if (gpioInitialise() < 0) {
                std::cerr << "Error: No se pudo inicializar pigpio." << std::endl;
                return -1;
            }
        
            // Configurar pines de direccin como salida
            for (int i = 0; i < 4; ++i) {
                gpioSetMode(DIR_PINS[i], PI_OUTPUT);
                gpioSetMode(PWM_PINS[i], PI_OUTPUT);
                setMotorSpeed(i, frequency);  // Inicializar PWM con frecuencia inicial
            }
        
            // Asegrate de establecer XDG_RUNTIME_DIR
            if (getenv("XDG_RUNTIME_DIR") == nullptr) {
                setenv("XDG_RUNTIME_DIR", "/tmp/runtime-$(id -u)", 1);
            }
        
            // Ejecuta libcamera-vid en un proceso separado y captura la salida en YUV, sin previsualizacn
            FILE* pipe = popen("libcamera-vid -t 0 --codec yuv420 --nopreview -o -", "r");
            if (!pipe) {
                std::cerr << "Error: No se pudo ejecutar libcamera-vid." << std::endl;
                return -1;
            }
        
            // Configura la ventana SFML
            sf::RenderWindow window(sf::VideoMode(1280, 720), "Camera Visualization with LiDAR");
            sf::Texture cameraTexture;
            sf::Sprite cameraSprite;
        
            // Buffer para leer los datos de video
            const int width = 640;
            const int height = 480;
            std::vector<uint8_t> buffer(width * height * 3 / 2); // Ajusta el tamao del buffer para YUV420
        
            cv::Mat yuvImage(height + height / 2, width, CV_8UC1, buffer.data());
            cv::Mat rgbImage(height, width, CV_8UC3);
        
            std::string port;
            ydlidar::os_init();
        
            // Obtener los puertos disponibles de LiDAR
            std::map<std::string, std::string> ports = ydlidar::lidarPortList();
            if (ports.size() > 1) {
                auto it = ports.begin();
                std::advance(it, 1); // Selecciona el segundo puerto disponible
                port = it->second;
            } else if (ports.size() == 1) {
                port = ports.begin()->second;
            } else {
                std::cerr << "No se detect ningn LiDAR. Verifica la conexin." << std::endl;
                return -1;
            }
        
            // Configuracin del LiDAR
            int baudrate = 115200;
            std::cout << "Baudrate: " << baudrate << std::endl;
        
            CYdLidar laser;
            laser.setlidaropt(LidarPropSerialPort, port.c_str(), port.size());
            laser.setlidaropt(LidarPropSerialBaudrate, &baudrate, sizeof(int));
        
            bool isSingleChannel = true;
            laser.setlidaropt(LidarPropSingleChannel, &isSingleChannel, sizeof(bool));
        
            float max_range = 8.0f;
            float min_range = 0.1f;
            float max_angle = 180.0f;
            float min_angle = -180.0f;
            float frequency = 8.0f;
        
            laser.setlidaropt(LidarPropMaxRange, &max_range, sizeof(float));
            laser.setlidaropt(LidarPropMinRange, &min_range, sizeof(float));
            laser.setlidaropt(LidarPropMaxAngle, &max_angle, sizeof(float));
            laser.setlidaropt(LidarPropMinAngle, &min_angle, sizeof(float));
            laser.setlidaropt(LidarPropScanFrequency, &frequency, sizeof(float));
        
            // Inicializar LiDAR
            if (!laser.initialize()) {
                std::cerr << "Error al inicializar el LiDAR." << std::endl;
                return -1;
            }
        
            // Iniciar el escaneo
            if (!laser.turnOn()) {
                std::cerr << "Error al encender el LiDAR." << std::endl;
                return -1;
            }
        
            // Thread para movimiento aleatorio
            std::thread randomMoveThread(randomMovement, std::ref(laser));
        
            while (window.isOpen()) {
                // Crear un minimapa para el LiDAR
                sf::RectangleShape minimap(sf::Vector2f(200, 200));
                minimap.setFillColor(sf::Color(200, 200, 200, 150)); // Fondo semitransparente
                minimap.setPosition(10, 10); // Esquina superior izquierda
                
        
                // Leer los datos del video desde la tubera
                size_t bytesRead = fread(buffer.data(), 1, buffer.size(), pipe);
                if (bytesRead != buffer.size()) {
                    std::cerr << "Error: No se pudo leer suficientes datos de video." << std::endl;
                    continue;
                }
        
                // Convertir YUV420 a RGB
                cv::cvtColor(yuvImage, rgbImage, cv::COLOR_YUV2RGB_I420);
        
                // Convertir a RGBA aadiendo un canal alfa
                cv::Mat frame_rgba;
                cv::cvtColor(rgbImage, frame_rgba, cv::COLOR_RGB2RGBA);
        
                // Actualizar la textura de la cmara con los datos del frame
                if (!cameraTexture.create(frame_rgba.cols, frame_rgba.rows)) {
                    std::cerr << "Error: No se pudo crear la textura." << std::endl;
                    continue;
                }
                cameraTexture.update(frame_rgba.ptr());
        
                cameraSprite.setTexture(cameraTexture);
                cameraSprite.setScale(
                    window.getSize().x / static_cast<float>(cameraTexture.getSize().x),
                    window.getSize().y / static_cast<float>(cameraTexture.getSize().y)
                );
        
        
                // Actualizar la ruta en el minimapa
                
                window.clear();
                window.draw(cameraSprite);
        
                
                window.draw(minimap);
                drawGrid(window, minimap);
                updateRoute(minimap, window);
                // Dibujar la ruta en el minimapa
                drawRoute(window, routePoints, minimap.getPosition());
                // Dibujar el centro del LiDAR (color azul)
                drawRobot(window, robotFixedPosition, minimap);
        
        
                // Calcular el centro del rectngulo del robot para dibujar la lnea que indica el norte
                sf::Vector2f robotCenter = minimap.getPosition() + robotFixedPosition;
        
                // Dibujar la lnea hacia el norte (ajustada al centro del rectngulo)
                sf::Vertex line[] =
                {
                    sf::Vertex(sf::Vector2f(105, 105), sf::Color::Black), // Posicin inicial de la lnea
                    sf::Vertex(sf::Vector2f(105, 160), sf::Color::Black)   // Posicin final hacia arriba
                };
        
                // Dibujar la lnea que representa el norte
                window.draw(line, 2, sf::Lines);
        
        
        
                LaserScan scan;
                if (laser.doProcessSimple(scan)) {
                    drawLiDARPoints(window, scan, minimap.getPosition(), 25.0f, max_range);
                }else {
                    std::cerr << "No se pudieron obtener los datos del LiDAR." << std::endl;
                }
        
                sf::Event event;
                while (window.pollEvent(event)) {
                    if (event.type == sf::Event::Closed){
                        window.close();
                    }
        
                    if(event.type == sf::Event::MouseButtonPressed){
                        std::cout << "Mouse pressed" << std::endl;
                        handleMouseClick(event, minimap);
                    }else if(event.type == sf::Event::MouseButtonReleased){
                        std::cout << "Mouse released" << std::endl;
                        handleMouseClick(event, minimap);
                        isDrawing = false;
                    }
                    
        
        
        
                    if (event.type == sf::Event::KeyPressed) {
                        switch (event.key.code) {
                            case sf::Keyboard::W:
                                is_manual_mode = true;
                                std::cout << "W" << std::endl;
                                moveForward();
                                break;
                            case sf::Keyboard::S:
                                is_manual_mode = true;
                                std::cout << "S" << std::endl;
                                moveBackward();
                                break;
                            case sf::Keyboard::A:
                                is_manual_mode = true;
                                std::cout << "A" << std::endl;
                                turnLeft();
                                break;
                            case sf::Keyboard::D:
                                is_manual_mode = true;
                                std::cout << "D" << std::endl;
                                turnRight();
                                break;
                            case sf::Keyboard::V:
                                is_manual_mode = true;
                                std::cout << "V" << std::endl;
                                increaseSpeed();
                                break;
                            case sf::Keyboard::B:
                                is_manual_mode = true;
                                std::cout << "B" << std::endl;
                                decreaseSpeed();
                                break;
                            case sf::Keyboard::Space:
                                is_manual_mode = true;
                                std::cout << "Space" << std::endl;
                                stopMotors();
                                break;
                            case sf::Keyboard::R: {
                                is_manual_mode = false;
                                std::vector<int> route = {1, 2, 5, 8, 11, 14, 13, 16};
                                moveAlongRoute(route);
                                //std::thread routeThread(followRoute, std::ref(window), std::ref(routePoints), std::ref(scan), minimap.getPosition(), max_range);
                                //routeThread.detach();
                                break;
                            }
                            case sf::Keyboard::K:
                                is_manual_mode = false;
                                std::cout << "K" << std::endl;
                                break;
                            case sf::Keyboard::M:
                                is_manual_mode = true;
                                std::cout << "M" << std::endl;
                                break;
                            default:
                                break;
                        }
                    }
                }
        
        
                window.display();
            }
        
            // Detener el escaneo del LiDAR
            laser.turnOff();
            laser.disconnecting();
        
            // Cierra la tubera, detiene los motores y apaga el robot
            pclose(pipe);
            stopMotors();
            gpioTerminate();
        
            is_running = false;
            randomMoveThread.join();
        
            return 0;
        }
    \end{lstlisting}