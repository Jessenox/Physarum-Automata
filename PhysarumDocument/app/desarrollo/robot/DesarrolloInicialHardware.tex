\subsubsection{Desarrollo Inicial del Sistema del control del Robot}
\label{sub:Desarrollo Inicial del Sistema del control del Robot}
    En esta subsecci\'on nos centramos en desarrollar el c\'odigo inicial para poder controlar el robot, el cual es el siguiente c\'odigo:
    \vskip 0.5cm
    \begin{lstlisting}[language={C++}, caption={Primera versi\'on del c\'odigo del sistema de control del robot}, label={Script}]
        #include "CYdLidar.h"
        #include <SFML/Graphics.hpp>
        #include <opencv2/opencv.hpp>
        #include <iostream>
        #include <cstdlib>
        #include <cstdio>
        #include <pigpio.h>
        #include <string>
        #include <map>
        #include <vector>
        #include <atomic>
        #include <thread>
        #include <random>
        
        using namespace std;
        using namespace ydlidar;
        
        // Pines y configuraci\'on para los motores
        const int PWM_PINS[] = {13, 19, 18, 12};  // Pines PWM para los motores
        const int DIR_PINS[] = {5, 6, 23, 24};    // Pines de direcci\'on para los motores
        int frequency = 400; // Frecuencia inicial
        
        std::atomic<bool> is_running(true);
        std::atomic<bool> is_manual_mode(true);
        
        void setMotorSpeed(int motor, int frequency) {
            if (motor >= 0 && motor < 4) {
                gpioSetPWMfrequency(PWM_PINS[motor], frequency);
            }
        }
        
        void setMotorDirection(int motor, int direction) {
            if (motor >= 0 && motor < 4) {
                gpioWrite(DIR_PINS[motor], direction);
            }
        }
        
        void stopMotors() {
            for (int i = 0; i < 4; ++i) {
                gpioPWM(PWM_PINS[i], 0);
            }
        }
        
        void moveForward() {
            for (int i = 0; i < 4; ++i) {
                gpioPWM(PWM_PINS[i], 128);  // Establecer ciclo de trabajo al 50%
            }
            setMotorDirection(0, 0); // Motor 1
            setMotorDirection(2, 0); // Motor 3
            setMotorDirection(1, 1); // Motor 2
            setMotorDirection(3, 1); // Motor 4
        }
        
        void moveBackward() {
            for (int i = 0; i < 4; ++i) {
                gpioPWM(PWM_PINS[i], 128);  // Establecer ciclo de trabajo al 50%
            }
            setMotorDirection(0, 1); // Motor 1
            setMotorDirection(2, 1); // Motor 3
            setMotorDirection(1, 0); // Motor 2
            setMotorDirection(3, 0); // Motor 4
        }
        
        void turnLeft() {
            for (int i = 0; i < 4; ++i) {
                gpioPWM(PWM_PINS[i], 128);  // Establecer ciclo de trabajo al 50%
            }
            setMotorDirection(0, 1); // Motor 1
            setMotorDirection(2, 1); // Motor 3
            setMotorDirection(1, 1); // Motor 2
            setMotorDirection(3, 1); // Motor 4
        }
        
        void turnRight() {
            for (int i = 0; i < 4; ++i) {
                gpioPWM(PWM_PINS[i], 128);  // Establecer ciclo de trabajo al 50%
            }
            setMotorDirection(0, 0); // Motor 1
            setMotorDirection(2, 0); // Motor 3
            setMotorDirection(1, 0); // Motor 2
            setMotorDirection(3, 0); // Motor 4
        }
        
        void increaseSpeed() {
            if (frequency < 1600) {
                frequency += 100;
                if (frequency > 1600) frequency = 1600;
                for (int i = 0; i < 4; ++i) {
                    setMotorSpeed(i, frequency);
                }
            }
        }
        
        void decreaseSpeed() {
            if (frequency > 400) {
                frequency -= 100;
                if (frequency < 400) frequency = 400;
                for (int i = 0; i < 4; ++i) {
                    setMotorSpeed(i, frequency);
                }
            }
        }
        
        sf::Color getPointColor(float distance, float maxRange) {
            float ratio = distance / maxRange;
            return sf::Color(255 * (1 - ratio), 255 * ratio, 0); // Color de rojo a verde
        }
        
        void randomMovement(CYdLidar &laser) {
            std::random_device rd;
            std::mt19937 gen(rd());
            std::discrete_distribution<> dist({1, 10, 10, 70}); // Distribuci\'on para la probabilidad de movimiento
        
            while (is_running) {
                if (!is_manual_mode) {
                    LaserScan scan;
                    if (laser.doProcessSimple(scan)) {
                        bool obstacle_detected = false;
                        for (const auto &point : scan.points) {
                            if (point.range < 0.35) {  // Detecta un obst\'aculo a 35 cm
                                obstacle_detected = true;
                                break;
                            }
                        }
        
                        if (obstacle_detected) {
                            // Si detecta un obst\'aculo, retrocede por 4 segundos
                            moveBackward();
                            std::this_thread::sleep_for(std::chrono::seconds(4));
                            stopMotors();
        
                            // Luego gira aleatoriamente a la izquierda o derecha
                            int turn = dist(gen) % 2;
                            if (turn == 0) {
                                turnLeft();
                            } else {
                                turnRight();
                            }
                            std::this_thread::sleep_for(std::chrono::seconds(2));
                            stopMotors();
                        } else {
                            // Si no hay obst\'aculo, elige un movimiento aleatorio
                            int move = dist(gen);
                            switch (move) {
                                case 0: moveBackward(); break;
                                case 1: turnRight(); break;
                                case 2: turnLeft(); break;
                                case 3: moveForward(); break;
                            }
                        }
                    }
                    std::this_thread::sleep_for(std::chrono::milliseconds(500));
                    stopMotors();
                }
                std::this_thread::sleep_for(std::chrono::milliseconds(100));
            }
        }
        
        int main() {
            // Inicializar pigpio
            if (gpioInitialise() < 0) {
                std::cerr << "Error: No se pudo inicializar pigpio." << std::endl;
                return -1;
            }
        
            // Configurar pines de direcci\'on como salida
            for (int i = 0; i < 4; ++i) {
                gpioSetMode(DIR_PINS[i], PI_OUTPUT);
                gpioSetMode(PWM_PINS[i], PI_OUTPUT);
                setMotorSpeed(i, frequency);  // Inicializar PWM con frecuencia inicial
            }
        
            // Aseg\'urate de establecer XDG_RUNTIME_DIR
            if (getenv("XDG_RUNTIME_DIR") == nullptr) {
                setenv("XDG_RUNTIME_DIR", "/tmp/runtime-$(id -u)", 1);
            }
        
            // Ejecuta libcamera-vid en un proceso separado y captura la salida en YUV, sin previsualizaci\'on
            FILE* pipe = popen("libcamera-vid -t 0 --codec yuv420 --nopreview -o -", "r");
            if (!pipe) {
                std::cerr << "Error: No se pudo ejecutar libcamera-vid." << std::endl;
                return -1;
            }
        
            // Configura la ventana SFML
            sf::RenderWindow window(sf::VideoMode(1280, 720), "Camera Visualization with LiDAR");
            sf::Texture cameraTexture;
            sf::Sprite cameraSprite;
        
            // Buffer para leer los datos de video
            const int width = 640;
            const int height = 480;
            std::vector<uint8_t> buffer(width * height * 3 / 2); // Ajusta el tama\~no del buffer para YUV420
        
            cv::Mat yuvImage(height + height / 2, width, CV_8UC1, buffer.data());
            cv::Mat rgbImage(height, width, CV_8UC3);
        
            std::string port;
            ydlidar::os_init();
        
            // Obtener los puertos disponibles de LiDAR
            std::map<std::string, std::string> ports = ydlidar::lidarPortList();
            if (ports.size() > 1) {
                auto it = ports.begin();
                std::advance(it, 1); // Selecciona el segundo puerto disponible
                port = it->second;
            } else if (ports.size() == 1) {
                port = ports.begin()->second;
            } else {
                std::cerr << "No se detect\'o ning\'un LiDAR. Verifica la conexi\'on." << std::endl;
                return -1;
            }
        
            // Configuraci\'on del LiDAR
            int baudrate = 115200;
            std::cout << "Baudrate: " << baudrate << std::endl;
        
            CYdLidar laser;
            laser.setlidaropt(LidarPropSerialPort, port.c_str(), port.size());
            laser.setlidaropt(LidarPropSerialBaudrate, &baudrate, sizeof(int));
        
            bool isSingleChannel = true;
            laser.setlidaropt(LidarPropSingleChannel, &isSingleChannel, sizeof(bool));
        
            float max_range = 8.0f;
            float min_range = 0.1f;
            float max_angle = 180.0f;
            float min_angle = -180.0f;
            float frequency = 8.0f;
        
            laser.setlidaropt(LidarPropMaxRange, &max_range, sizeof(float));
            laser.setlidaropt(LidarPropMinRange, &min_range, sizeof(float));
            laser.setlidaropt(LidarPropMaxAngle, &max_angle, sizeof(float));
            laser.setlidaropt(LidarPropMinAngle, &min_angle, sizeof(float));
            laser.setlidaropt(LidarPropScanFrequency, &frequency, sizeof(float));
        
            // Inicializar LiDAR
            if (!laser.initialize()) {
                std::cerr << "Error al inicializar el LiDAR." << std::endl;
                return -1;
            }
        
            // Iniciar el escaneo
            if (!laser.turnOn()) {
                std::cerr << "Error al encender el LiDAR." << std::endl;
                return -1;
            }
        
            // Thread para movimiento aleatorio
            std::thread randomMoveThread(randomMovement, std::ref(laser));
        
            while (window.isOpen()) {
                sf::Event event;
                while (window.pollEvent(event)) {
                    if (event.type == sf::Event::Closed)
                        window.close();
        
                    if (event.type == sf::Event::KeyPressed) {
                        switch (event.key.code) {
                            case sf::Keyboard::W:
                                is_manual_mode = true;
                                moveForward();
                                break;
                            case sf::Keyboard::S:
                                is_manual_mode = true;
                                moveBackward();
                                break;
                            case sf::Keyboard::A:
                                is_manual_mode = true;
                                turnLeft();
                                break;
                            case sf::Keyboard::D:
                                is_manual_mode = true;
                                turnRight();
                                break;
                            case sf::Keyboard::Add:
                                is_manual_mode = true;
                                increaseSpeed();
                                break;
                            case sf::Keyboard::Subtract:
                                is_manual_mode = true;
                                decreaseSpeed();
                                break;
                            case sf::Keyboard::Space:
                                is_manual_mode = true;
                                stopMotors();
                                break;
                            case sf::Keyboard::K:
                                is_manual_mode = false;
                                break;
                            case sf::Keyboard::M:
                                is_manual_mode = true;
                                break;
                            default:
                                break;
                        }
                    }
                }
        
                // Leer los datos del video desde la tuber\'ia
                size_t bytesRead = fread(buffer.data(), 1, buffer.size(), pipe);
                if (bytesRead != buffer.size()) {
                    std::cerr << "Error: No se pudo leer suficientes datos de video." << std::endl;
                    continue;
                }
        
                // Convertir YUV420 a RGB
                cv::cvtColor(yuvImage, rgbImage, cv::COLOR_YUV2RGB_I420);
        
                // Convertir a RGBA a\~nadiendo un canal alfa
                cv::Mat frame_rgba;
                cv::cvtColor(rgbImage, frame_rgba, cv::COLOR_RGB2RGBA);
        
                // Actualizar la textura de la c\'amara con los datos del frame
                if (!cameraTexture.create(frame_rgba.cols, frame_rgba.rows)) {
                    std::cerr << "Error: No se pudo crear la textura." << std::endl;
                    continue;
                }
                cameraTexture.update(frame_rgba.ptr());
        
                cameraSprite.setTexture(cameraTexture);
                cameraSprite.setScale(
                    window.getSize().x / static_cast<float>(cameraTexture.getSize().x),
                    window.getSize().y / static_cast<float>(cameraTexture.getSize().y)
                );
        
                window.clear();
                window.draw(cameraSprite);
        
                // Crear un minimapa para el LiDAR
                sf::RectangleShape minimap(sf::Vector2f(200, 200));
                minimap.setFillColor(sf::Color(200, 200, 200, 150)); // Fondo semitransparente
                minimap.setPosition(10, 10); // Esquina superior izquierda
        
                window.draw(minimap);
        
                // Dibujar el centro del LiDAR (color azul)
                sf::CircleShape lidarCenter(5); // Radio del c\'irculo del LiDAR
                lidarCenter.setFillColor(sf::Color::Blue);
                lidarCenter.setPosition(105, 105); // Posici\'on del centro en el minimapa
        
                window.draw(lidarCenter);
        
                // Dibujar la l\'inea hacia el norte
                sf::Vertex line[] =
                {
                    sf::Vertex(sf::Vector2f(110, 110), sf::Color::Black),
                    sf::Vertex(sf::Vector2f(110, 60), sf::Color::Black) // L\'inea hacia arriba (norte)
                };
        
                window.draw(line, 2, sf::Lines);
        
                LaserScan scan;
                if (laser.doProcessSimple(scan)) {
                    for (const auto& point : scan.points) {
                        // Convertir coordenadas polares a cartesianas
                        float x = point.range * cos(point.angle);
                        float y = point.range * sin(point.angle);
        
                        // Ajustar los puntos al minimapa
                        float scale = 25.0f;
                        float adjustedX = 110 + x * scale;
                        float adjustedY = 110 - y * scale; // Invertir Y para coordinar con la pantalla
        
                        // Dibujar los puntos en el minimapa, excluyendo el centro (0,0)
                        if (point.range > 0.05) {
                            sf::CircleShape lidarPoint(2);
                            lidarPoint.setPosition(adjustedX, adjustedY);
                            lidarPoint.setFillColor(getPointColor(point.range, max_range));
        
                            window.draw(lidarPoint);
                        }
                    }
                } else {
                    std::cerr << "No se pudieron obtener los datos del LiDAR." << std::endl;
                }
        
                window.display();
            }
        
            // Detener el escaneo del LiDAR
            laser.turnOff();
            laser.disconnecting();
        
            // Cierra la tuber\'ia, detiene los motores y apaga el robot
            pclose(pipe);
            stopMotors();
            gpioTerminate();
        
            is_running = false;
            randomMoveThread.join();
        
            return 0;
        }
    \end{lstlisting}

    En este c\'odigo se inicializan los motores y se establecen las funciones para controlar el movimiento del robot. 
        Adem\'as, se configura el LiDAR y se inicia el escaneo. El robot se mueve aleatoriamente y evita obst\'aculos 
        detectados por el LiDAR. La c\'amara muestra la vista del robot y el minimapa muestra los puntos detectados por 
        el LiDAR. El usuario puede controlar manualmente el robot con las teclas W, A, S y D para moverse hacia adelante, 
        izquierda, atr\'as y derecha, respectivamente. Las teclas V y B aumentan y disminuyen la velocidad del robot, 
        respectivamente. La tecla Espacio detiene el robot. La tecla K activa el modo autom\'atico y la tecla M activa 
        el modo manual. El robot se mueve aleatoriamente y evita obst\'aculos detectados por el LiDAR. El c\'odigo se ejecuta 
        en un bucle hasta que se cierra la ventana de la c\'amara. Al final, se detiene el escaneo del LiDAR y se apaga el robot.
    