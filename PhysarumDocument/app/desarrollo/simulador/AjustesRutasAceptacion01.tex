\subsubsection{Ajustes de rutas basado en resultado de pruebas de aceptaci\'on 1}
    Con los resultados que fueron obtenidos a partir de la
        ejecuci\'on de las pruebas de aceptaci\'on, se toma en cuenta
        que hay algunas configuraciones o ajustes que se deben de
        hacer al algoritmo. Una de las primeras es generar una ruta
        'mas limpia', es decir, una ruta en la cual existan mucho
        menos celdas las cuales impidan o dificulten generar una
        ruta un poco m\'as optima o que causan algunos
        impedimentos a la hora de generar el ordenamiento de
        coordenadas para ser enviadas al robot para la realizaci\'on de
        su funcionalidad de llegar a su destino.    
    \vskip 0.5cm
    Una de las soluciones que se considera para poder obtener
        las rutas un poco m\'as limpias y ajustarlas es limpiar un poco
        el camino o ruta generada por el Physarum una vez esta
        \'ultima siendo obtenida es eliminar las c\'elulas que conforman
        al estado 8 y 5 respectivamente, las cuales no sean
        necesarias para el c\'alculo de la ruta o que no formen parte
        importante de la ruta que fue obtenida. Para esto se
        consideran dos casos importantes
    \vskip 0.5cm
    \begin{enumerate}
        \item Las c\'elulas alrededor de el punto inicial y el punto
            final.
        \item Las c\'elulas que no conectan una c\'elula con otra.
    \end{enumerate}
    \vskip 0.5cm
    Una de las ideas para poder eliminar estas c\'elulas las cuales
        dificultan la obtenci\'on correcta de las rutas es eliminar las
        c\'elulas que est\'an alrededor del estado inicial y el estado
        final, dejando \'unicamente las que tienen conexi\'on con otras
        c\'elulas las cuales pertenecen a la ruta general.
    \vskip 0.5cm
    Esto se realiz\'o de la siguiente forma creando un nuevo
        aut\'omata con las siguientes reglas:
    \vskip 0.5cm
    \begin{itemize}
        \item Si el estado es 1 y alrededor no hay ning\'un estado 1, entonces es estado 0, sino, es estado 1.
        \item Si el estado es 2 y alrededor no hay ning\'un estado 1, entonces es estado 0, sino, es estado 2.
        \item Si el estado es 3, entonces es estado 7.
        \item Si el estado es 4 y alrededor no hay ning\'un estado 1, entonces es estado 0, sino, es estado 4.
        \item Si el estado es 5 y alrededor hay un punto inicial, entonces es estado 2, sino, si alrededor hay un nutriente, entonces es estado 4, sino es estado 1.
        \item Si es estado 6, entonces es estado 9.
        \item Si es estado 7, se queda en estado 7.
        \item Si el estado es 8 y alrededor hay un punto inicial, entonces es estado 2, sino, si alrededor hay un nutriente, entonces es estado 4, sino es estado 1.
        \item Si es estado 9, entonces se queda en estado 9.
    \end{itemize}
    \vskip 0.5cm    
    Y queda plasmado con el siguiente c\'odigo:
    \vskip 0.5cm
    \begin{lstlisting}[language={C++}, caption={Ruta Ajuste 1}, label={Script}]
        switch (tab[i][j]) {
            case 1:
                if (!aroundState1)
                    cellsAux[i][j] = 0;
                else
                    cellsAux[i][j] = 1;
            break;
            case 2:
                if (!aroundState1)
                    cellsAux[i][j] = 0;
                else
                    cellsAux[i][j] = 2;
            break;
            case 3:
                cellsAux[i][j] = 7;
            break;
            case 4:
                if (!aroundState1)
                    cellsAux[i][j] = 0;
                else
                    cellsAux[i][j] = 4;
            break;
            case 5:
                cellsCounter++;
                if (isInitialPointAround)
                    cellsAux[i][j] = 2;
                else if (isNutrientAround)
                    cellsAux[i][j] = 4;
                else
                    cellsAux[i][j] = 1;
            break;
            case 6:
                cellsAux[i][j] = 9;
            break;
            case 7:
                cellsAux[i][j] = 7;
            break;
            case 8:
                cellsCounter++;
                if (isInitialPointAround)
                    cellsAux[i][j] = 2;
                else if(isNutrientAround)
                    cellsAux[i][j] = 4;
                else
                    cellsAux[i][j] = 1;
            break;
            case 9:
                cellsAux[i][j] = 9;
            break;
        }
    \end{lstlisting}
    % section Resultados de las pruebas (end)
\vskip 0.5cm
    Lo anterior se hace para poder obtener una ruta con una
        mejor claridad y que dificulte menos el ordenamiento y la
        manera en la cual se obtienen las coordenadas con las cuales
        se realiza el recorrido.