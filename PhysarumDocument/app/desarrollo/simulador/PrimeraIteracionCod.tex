\subsubsection{Codificaci\'on e implementaci\'on de algoritmo en el robot en la primera iteraci\'on 1} % (fold)
\label{ssub:Codi}
    Con las rutas que han sido generadas de acuerdo a cada una
        de las simulaciones que fueron ejecutadas en el programa,
        como ya se ha mencionado, se genera la informaci\'on con la
        cual el robot realizar\'a el seguimiento de esta ruta para llegar
        de un punto inicial al final. Esta informaci\'on corresponde a
        la ruta que genera el Physarum cuando termina la simulaci\'on
        de llegar desde su punto inicial de expansi\'on a un nutriente
        con el cual se alimenta.
    \vskip 0.5cm
    Para poder hacer una correcta manipulaci\'on de la
        informaci\'on, la cual pueda ser enviada al robot que este
        pueda interpretarla y avanzar de acuerdo la ruta generada, es
        necesario crear un espacio el cual represente la posici\'on del
        robot en un determinado lugar, junto con el destino al cual se
        quiere llegar, a partir de ah\'i pasar esta configuraci\'on al
        modelo del Physarum, ejecutar la simulaci\'on y obtener la
        ruta, finalizando con la recopilaci\'on de la informaci\'on
        relacionada con la ruta y su almacenamiento, lo cual es
        representado de la siguiente manera:
    \vskip 0.5cm
    \lstset{caption={Pseudo c\'odigo del simulador Physarum}}
    \begin{lstlisting}
    Iniciar programa
    Colocar la configuraci\'on inicial
    Coloca punto inicial
    Coloca punto final
    Ejecutar Simulador_Physarum
    Si Simulador_Physarum obtuvo ruta, entonces:
        Guardar las coordenadas del punto inicial
        Guardar las coordenadas del punto final
        Guardar las coordenadas de las c\'elulas por orden de aparici\'on
        Desplegar todas las coordenadas en un archivo
    Si no
        Finaliza programa
    Fin Si
    \end{lstlisting}
    \vskip 0.5cm
    A partir de lo anterior, se comprende que al robot se le ser\'a
        enviado las coordenadas correspondientes a la ruta obtenida
        por el simulador del Physarum, con un punto inicial, el cual
        marcar\'a su posici\'on actual, su punto final, representando el
        destino al cual llegar\'a el robot y las c\'elulas del Physarum
        ordenadas por orden de aparici\'on, las cuales son las
        coordenadas por las cuales el robot tendr\'a que pasar para
        llegar desde el punto inicial al final.
    \vskip 0.5cm
    Con lo anterior, es posible la creaci\'on de informaci\'on la cual
        el robot sea capaz de leer y a partir de sus propias funciones,
        llegar de un punto inicial al final. Esta informaci\'on es
        almacenada en un archivo, el cual contiene cada una de las
        coordenadas ordenadas, siendo la primera el punto inicial,
        posteriormente cada una de las dem\'as coordenadas que
        representan a cada una de las c\'elulas del Physarum,
        ordenadas por orden de aparici\'on para que el robot tenga un
        orden el cual seguir para poder llegar a su destino.
    \vskip 0.5cm
    Finalmente, la \'ultima coordenada corresponde a la
        coordenada del punto final, el cual es el destino al cual el
        robot deber\'a de llegar, finalizando as\'i su recorrido.
        
% subsubsection Codi (end)