\subsubsection{Requerimientos no funcionales}

    En esta secci\'on se describen los requerimientos no 
        funcionales, los cuales establecen los criterios de calidad y 
        desempe\~no que el sistema debe cumplir para garantizar una operaci\'on 
        robusta, eficiente y compatible en distintos entornos. Estos 
        requerimientos no funcionales aseguran que el sistema sea eficiente, 
        adaptable y compatible, proporcionando una experiencia de usuario 
        satisfactoria. Los requerimientos no funcionales se detallan en el Cuadro \ref{tab:requerimientos_no_funcionales}.
    \vskip 0.5cm
    \begin{table}[h!]
        \centering
        \begin{tabular}{|c|c|p{12cm}|}
        \hline
        \textbf{ID} & \textbf{Nombre} &\textbf{Requerimiento No Funcional} \\
        \hline
        RNF1 & Rendimiento & El sistema debe responder a las solicitudes de inicio de simulaci\'on en menos de 3 segundos para mapas de hasta 1000 nodos. \\
        \hline
        RNF2 & Escalabilidad & El sistema debe poder manejar simulaciones en mapas con hasta 5000 nodos sin una ca\'ida de rendimiento superior al 5\%. \\
        \hline
        RNF3 & Portabilidad & El sistema debe ser portable y ejecutarse correctamente en sistemas operativos Windows 10 y Linux basados en Debian. \\
        \hline
        RNF4 & Facilidad de uso & El sistema debe ser comprensible y usable para un usuario novato tras un m\'aximo de 30 minutos de uso guiado. \\
        \hline
        \end{tabular}
        \caption{Requerimientos No funcionales del sistema}
        \label{tab:requerimientos_no_funcionales}
    \end{table}