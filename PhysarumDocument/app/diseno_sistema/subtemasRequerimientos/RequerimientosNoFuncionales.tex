\subsubsection{Requerimientos no funcionales}

    En esta secci\'on se describen los requerimientos no 
        funcionales, los cuales establecen los criterios de calidad y 
        desempe\~no que el sistema debe cumplir para garantizar una operaci\'on 
        robusta, eficiente y compatible en distintos entornos. Estos 
        requerimientos no funcionales aseguran que el sistema sea eficiente, 
        adaptable y compatible, proporcionando una experiencia de usuario 
        satisfactoria. Los requerimientos no funcionales se detallan en el Cuadro \ref{tab:requerimientos_no_funcionales}.
    \vskip 0.5cm
    \begin{table}[h!]
        \centering
        \begin{tabular}{|c|p{12cm}|}
        \hline
        \textbf{ID} & \textbf{Requerimiento No Funcional} \\
        \hline
        RNF1 & El sistema debe ser eficiente en t\'erminos de tiempo de simulaci\'on para optimizar el tiempo de generaci\'on de cada iteraci\'on. \\
        \hline
        RNF2 & El sistema debe ser capaz de manejar simulaciones en mapas grandes y con obst\'aculos sin p\'erdida significativa de rendimiento. \\
        \hline
        RNF3 & El sistema debe ser compatible tanto con sistemas operativos Windows como Linux. \\
        \hline
        \end{tabular}
        \caption{Requerimientos No funcionales del sistema}
        \label{tab:requerimientos_no_funcionales}
    \end{table}