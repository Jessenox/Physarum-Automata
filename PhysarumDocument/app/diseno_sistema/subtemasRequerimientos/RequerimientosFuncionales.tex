\subsubsection{Requerimientos funcionales} % (fold)

    En esta secci\'on se presentan los requerimientos funcionales que definen las 
        caracter\'isticas y capacidades espec\'ificas que el sistema debe proporcionar 
        para cumplir su prop\'osito en el contexto de simulaci\'on y monitoreo de rutas 
        automatizadas. Estos requerimientos aseguran que el sistema cumpla con las 
        funciones clave necesarias para la interacci\'on y operaci\'on de la simulaci\'on.
        Los requerimientos funcionales se listan en el Cuadro \ref{tab:requerimientos_funcionales}.
    \vskip 0.5cm
    \begin{table}[h!]
        \centering
        \begin{tabular}{|c|p{12cm}|}
        \hline
        \textbf{ID} & \textbf{Requerimiento Funcional} \\
        \hline
        RF1 & El sistema debe permitir al usuario seleccionar estados en el simulador mediante el teclado y el rat\'on. \\
        \hline
        RF2 & El sistema debe permitir al usuario colocar los estados inicial y final en el lienzo antes de iniciar la simulaci\'on. \\
        \hline
        RF3 & El sistema debe iniciar la simulaci\'on de rutas al presionar la tecla ENTER. \\
        \hline
        RF4 & El sistema debe permitir la carga de un mapa o imagen en el lienzo para definir el entorno inicial de la simulaci\'on. \\
        \hline
        RF5 & El sistema debe permitir al usuario visualizar la ruta generada en tiempo real durante la simulaci\'on. \\
        \hline
        \end{tabular}
        \caption{Requerimientos funcionales del sistema}
        \label{tab:requerimientos_funcionales}
    \end{table}