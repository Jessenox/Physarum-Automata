\section{Propuesta a desarrollar}
\label{sec:Propuesta a desarrollar}
    % Parrafo 1
        El proyecto propuesto consiste en el desarrollo de un sistema de monitoreo poblacional basado en 
            la implementaci\'on de un aut\'omata en un modelo bidimensional no lineal. En este caso, 
            como mencionamos anteriormente, el algoritmo esta basado en el modelo de Physarum Polycephalum.
            Este modelo es un organismo unicelular que se comporta como un aut\'omata celular,
            y es capaz de resolver problemas de optimizaci\'on y ruteo.
        \vskip 0.5cm
    % Parrafo 2
        A su vez el sistema propuesto se basa en la utilizaci\'on de robots aut\'onomos, los cuales 
            se encargar\'an de recolectar informaci\'on de la poblaci\'on y de los entornos en los que 
            se encuentran. Estos robots estar\'an equipados con c\'amaras y sensores que les permitir\'an 
            detectar y clasificar entidades poblacionales. Adem\'as, los robots estar\'an conectados a 
            una red de comunicaci\'on que les permitir\'a compartir informaci\'on en tiempo real.
        \vskip 0.5cm
    % Parrafo 3
        El sistema funcionara de la siguiente manera: los robots aut\'onomos recibiran la ruta a seguir 
            por parte de nuestro simulador del Physarum Polycephalum, el cual se encargara de determinar
            la ruta optima para recolectar informaci\'on de la poblaci\'on. Una vez que los robots
            recolecten la informaci\'on, esta sera enviada a un servidor central, el cual se encargara
            de procesar la informaci\'on y de generar reportes en tiempo real.
        \vskip 0.5cm
    % Parrafo 4
        La implementaci\'on de este sistema permitira a los investigadores y a las autoridades locales
            monitorear poblaciones de manera eficiente y en tiempo real. Adem\'as, el sistema permitira
            la detecci\'on de cambios en las poblaciones y en los entornos en los que se encuentran.
        \vskip 0.5cm
    % Parrafo 5
        Por ello tendremos principalmente dos 'productos' a desarrollar, el primero sera el simulador
            del Physarum Polycephalum, el cual sera un sistema que permitira determinar rutas optimas
            para recolectar informaci\'on de la poblaci\'on. El segundo producto sera el sistema de 
            monitoreo poblacional, el cual sera un sistema que permitira a los robots aut\'onomos 
            recolectar informaci\'on de la poblaci\'on y de los entornos en los que se encuentran.
        \vskip 0.5cm
    % Parrafo 6
        Se detallara la implementaci\'on de estos sistemas en las siguientes secciones.
    