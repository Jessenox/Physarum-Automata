\subsection{Justificaci\'on}
\label{sec-1-1-2}

% Parrafo 1
    En el marco actual de la automatizaci\'on constante de procesos y el desarrollo de herramientas que facilitan la
        realizaci\'on de tareas, han surgido distintas tecnolog\'ias y procesos que han sabido atacar de la mejor manera las
        problem\'aticas que involucran a la automatizaci\'on. Sin embargo, muchas veces es complicado darles un correcto
        seguimiento a las actividades y procesos que est\'an involucrados en la realizaci\'on de tareas autom\'aticas,
        ocasionando distintos problemas que afectan en la soluci\'on del problema al que originalmente planteaban
        solucionar o complicar de manera innecesaria el proceso. Por lo que es necesario crear herramientas de
        monitorizaci\'on con m\'etodos m\'as fiables a los ya existentes, y que, adem\'as de brindar un correcto seguimiento
        a los procesos a los cuales se les hace un an\'alisis, tenga la capacidad de reaccionar ante los cambios relevantes
        e importantes que surjan durante la realizaci\'on de las distintas tareas y procesos en los que se vea involucrado.
\vskip 0.5cm
% Parrafo 2
    Es por eso por lo que se busca la implementaci\'on de un sistema rob\'otico, el cual, por medio de la aplicaci\'on de
        la teor\'ia de aut\'omatas celulares, monitorice la realizaci\'on de distintas tareas y procesos en los que se vea
        involucrado y, adem\'as, priorizando siempre que la monitorizaci\'on sea efectiva, continua y confiable. Esto para
        las distintas industrias que realicen actividades en las cuales se vean involucradas la automatizaci\'on de procesos
        y tareas.
\vskip 0.5cm
% Parrafo 3
    Los aut\'omatas celulares han sido usados para distintas disciplinas que van desde la antropolog\'ia hasta los
        gr\'aficos por computadora, sin embargo, ha sido muy poco visto en actividades que involucren sistemas
        rob\'oticos debido al comportamiento y la manera en la que los aut\'omatas celulares se comportan a partir de
        diversas entradas, siendo a veces complicado discernir el comportamiento que se tendr\'a, lo que a\~nade
        complejidad al implementar uno de estos aut\'omatas a sistemas rob\'oticos. Pero gracias al conocimiento brindado
        por algunas materias como lo son Sistemas Operativos, Arquitectura de Computadoras, Dise\~no Digital y Teor\'ia
        Computacional, se puede llegar a un procedimiento y tratamiento de la informaci\'on tal que sea posible dirigir
        el aut\'omata a la mejor soluci\'on posible.