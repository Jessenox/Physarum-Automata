\subsection{Pruebas de Aceptaci\'on en escenarios controlados con retroalimentaci\'on de los sensores} % (fold)
\label{sub:Pruebas de Aceptaci\'on en escenarios controlados con retroalimentaci\'on de los sensores}
    Para verificar el correcto funcionamiento del robot propuesto, se realizaron pruebas de aceptaci\'on en escenarios controlados, 
        con retroalimentaci\'on de los sensores. Estas pruebas se llevaron a cabo en un entorno controlado, con obst\'aculos 
        y condiciones predefinidas, para evaluar el desempe\~no del robot en situaciones espec\'ificas. 
    \vskip 0.5cm
    Las pruebas se realizaron en un \'area de 10 x 10 metros, con obst\'aculos colocados en diferentes posiciones, 
        para simular un entorno realista. El robot se program\'o para moverse en l\'ineas rectas y girar en \'angulos 
        espec\'ificos, evitando los obst\'aculos y manteniendo una distancia segura. Se utilizaron sensores de distancia 
        LIDAR para detectar los obst\'aculos y medir las distancias, y una c\'amara de visi\'on nocturna para 
        proporcionar retroalimentaci\'on visual al operador.
    \vskip 0.5cm
    Durante las pruebas, el robot demostr\'o un desempe\~no satisfactorio, movi\'endose de manera fluida y precisa, 
        evitando los obst\'aculos y manteniendo una distancia segura. Los sensores de distancia LIDAR detectaron
        los obst\'aculos con precisi\'on, permitiendo al robot ajustar su trayectoria de manera oportuna. La c\'amara
        de visi\'on nocturna proporcion\'o una retroalimentaci\'on visual clara y detallada, permitiendo al operador
        supervisar el desempe\~no del robot en tiempo real. Y se pueden ver los resultados en el siguiente video.
    \vskip 0.5cm
    \url{https://drive.google.com/file/d/1Tvr_ViPACePNEtGQX-M5AOJgwADlIEvm/view?usp=drive_link}
