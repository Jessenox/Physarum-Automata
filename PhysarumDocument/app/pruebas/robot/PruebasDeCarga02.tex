\subsection{Pruebas de carga y resistencia en escenarios extendidos para validaci\'on de fallos}
\label{sub:PruebasDeCarga02}
    Como parte de la validaci\'on del sistema en entornos complejos, se realiz\'o una prueba de carga  
        y resistencia durante un evento en el planetario del IPN. El robot estuvo en funcionamiento 
        continuo durante un per\'iodo de dos horas, con el objetivo de evaluar su capacidad para operar 
        en un entorno din\'amico y lleno de personas, incluidas condiciones espec\'ificas como la luz solar 
        directa y la presencia de m\'ultiples obst\'aculos m\'oviles, principalmente ni\~nos.
    \vskip 0.5cm
    Durante esta prueba, se observ\'o que el sistema mantuvo un rendimiento estable, a pesar de los desaf\'ios 
        que presentaba el entorno. Sin embargo, como se observ\'o en iteraciones anteriores, la luz del 
        sol interfer\'ia con la capacidad del sensor LiDAR para detectar correctamente los obst\'aculos, 
        lo cual se agravaba en \'areas donde los rayos solares incid\'ian directamente sobre los objetos. 
        Esto generaba inconsistencias en la detecci\'on de obst\'aculos, sobre todo cuando el robot interactuaba 
        con objetos en movimiento, como los ni\~nos que se encontraban en el lugar.
    \vskip 0.5cm
    Si bien no ocurrieron incidentes cr\'iticos durante las dos horas de funcionamiento, se grabaron aproximadamente 
        40 minutos de video como parte de la documentaci\'on del evento. Esta grabaci\'on incluye las interacciones 
        del robot con el entorno, permitiendo observar c\'omo el sistema respond\'ia a los desaf\'ios planteados, como 
        la presencia de ni\~nos y las variaciones en la luz natural. Aunque en gran parte del tiempo el robot oper\'o 
        de forma estable, estos eventos brindaron informaci\'on valiosa sobre las limitaciones del sistema en entornos reales.
    \vskip 0.5cm
    Puedes acceder al video de la prueba haciendo clic en el siguiente enlace: \url{https://drive.google.com/file/d/1p5S6dewukp-0Qsq_qk9K40o2LhSbtPgf/view?usp=drive_link}
    \vskip 0.5cm
    Esta prueba proporcion\'o una valiosa evaluaci\'on del rendimiento del sistema en escenarios extendidos y su 
        capacidad para mantener la operaci\'on bajo condiciones diversas, lo que contribuye a la validaci\'on general 
        del sistema y su resistencia en situaciones prolongadas.
    \vskip 0.5cm
    A su vez, se hizo otra prueba de carga en la Secreter\'ia de Hacienda y Cr\'edito P\'ublico, en la cual el robot 
        estuvo en funcionamiento durante un per\'iodo de 3 horas, con el objetivo de evaluar su capacidad para operar 
        en un entorno din\'amico y lleno de personas. Se puede observar aqu\'i: \url{https://drive.google.com/file/d/1p5S6dewukp-0Qsq_qk9K40o2LhSbtPgf/view?usp=drive_link}
% subsection Pruebas de carga y resistencia en escenarios extendidos para validaci\'on de fallos (end)