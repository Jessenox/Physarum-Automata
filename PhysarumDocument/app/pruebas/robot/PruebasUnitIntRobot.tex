\subsection{Pruebas unitarias e integraci\'on de sensores y motores en entornos controlados} % (fold)
\label{sub:Prub01}

    Para verificar el correcto funcionamiento de los motores y sensores del robot, se realizaron pruebas unitarias e integraci\'on en entornos controlados. 
    Estas pruebas permitieron identificar posibles fallas y ajustes necesarios en el hardware y software del robot, as\'i como evaluar la precisi\'on y 
    eficiencia de los motores y sensores en diferentes situaciones. A continuaci\'on, se detallan las pruebas realizadas y los resultados obtenidos.

    \subsubsection{Pruebas unitarias} % (fold)
    \label{ssub:Pruebas unitarias}
        Las pruebas unitarias se enfocaron en verificar el correcto funcionamiento de los motores paso a paso y el sensor de distancia LIDAR. 
        Para las pruebas de los motores, se utiliz\'o un controlador de motores a pasos y un programa de prueba que permiti\'o verificar el 
        movimiento y precisi\'on de los motores en diferentes direcciones. Por otro lado, para las pruebas del sensor LIDAR, se emple\'o un 
        programa de prueba que permiti\'o medir la distancia a un objeto y detectar obst\'aculos en diferentes direcciones. 
        Para las pruebas unitarias usamos lo que ser\'ia gtest, que es una biblioteca de pruebas unitarias para C++.
        \vskip 0.5cm
        El c\'odigo de las pruebas unitarias se muestra a continuaci\'on:
        \begin{lstlisting}[language={C++}, caption={Primera versi\'on del c\'odigo del sistema de control del robot}, label={Script}]
#include <gtest/gtest.h>
#include <vector>
#include <atomic>
#include <iostream>
#include <random>

// Variables globales para la simulación
std::atomic<bool> is_running(true);
std::atomic<bool> is_manual_mode(true);
int frequency = 400;  // Frecuencia inicial
std::vector<int> motor_speeds(4, 0); // Simulación de velocidad de los motores
std::vector<int> motor_directions(4, 0); // Simulación de dirección de los motores

// Funciones de simulación para GPIO
void gpioSetPWMfrequency(int pin, int frequency) {
    std::cout << "Set PWM frequency for pin " << pin << " to " << frequency << std::endl;
    motor_speeds[pin] = frequency;
}

void gpioWrite(int pin, int value) {
    std::cout << "Set direction for pin " << pin << " to " << value << std::endl;
    motor_directions[pin] = value;
}

void gpioPWM(int pin, int value) {
    std::cout << "Set PWM for pin " << pin << " to " << value << std::endl;
    motor_speeds[pin] = value;
}

void gpioTerminate() {
    std::cout << "GPIO terminated" << std::endl;
}

// Función simulada para detener los motores
void stopMotors() {
    std::cout << "Stopping all motors" << std::endl;
    for (int i = 0; i < 4; ++i) {
        motor_speeds[i] = 0;
    }
}

// Simulación de LiDAR
class SimulatedLidar {
public:
    bool doProcessSimple(LaserScan &scan) {
        scan.points = generateFakeScanData();
        return true;
    }

private:
    std::vector<LaserPoint> generateFakeScanData() {
        // Genera puntos simulados del LiDAR
        std::vector<LaserPoint> points;
        for (int i = 0; i < 100; ++i) {
            LaserPoint point;
            point.angle = static_cast<float>(rand() % 360);  // Angulo aleatorio
            point.range = static_cast<float>((rand() % 100) / 100.0);  // Rango aleatorio entre 0 y 1
            points.push_back(point);
        }
        return points;
    }
};

// Definir la estructura LaserScan y LaserPoint para simulación
struct LaserPoint {
    float angle;
    float range;
};

struct LaserScan {
    std::vector<LaserPoint> points;
};

// Prueba para simular movimiento hacia adelante
TEST(MotorControlSimulationTest, MoveForward) {
    moveForward();
    EXPECT_EQ(motor_directions[0], 0);  // Verifica que los motores se dirigen hacia adelante
    EXPECT_EQ(motor_directions[1], 1);
    EXPECT_EQ(motor_speeds[0], 128);    // Verifica que la velocidad PWM es 50%
}

// Prueba para simular detección de obstáculos con LiDAR
TEST(LidarSimulationTest, DetectObstacle) {
    SimulatedLidar lidar;
    LaserScan scan;
    bool result = lidar.doProcessSimple(scan);
    
    EXPECT_TRUE(result);  // La función debe retornar true
    EXPECT_GT(scan.points.size(), 0);  // Debe haber puntos en el escaneo
    EXPECT_LT(scan.points[0].range, 1.0);  // Los puntos deben estar dentro del rango simulado
}

// Prueba para detener motores
TEST(MotorControlSimulationTest, StopMotors) {
    stopMotors();
    for (int i = 0; i < 4; ++i) {
        EXPECT_EQ(motor_speeds[i], 0);  // Verifica que la velocidad PWM de todos los motores es 0
    }
}

// Inicialización del framework de pruebas
int main(int argc, char **argv) {
    ::testing::InitGoogleTest(&argc, argv);
    return RUN_ALL_TESTS();
}

        \end{lstlisting}
    \vskip 0.5cm
    Y los resultados de las pruebas unitarias se muestran a continuaci\'on:
    