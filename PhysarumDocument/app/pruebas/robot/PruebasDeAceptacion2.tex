\subsection{Pruebas de aceptaci\'on en entornos complejos simulados y reales}

    La prueba de aceptaci\'on del sistema no solo se limit\'o a eventos formales, 
        sino que incluy\'o su evaluaci\'on en entornos reales, espec\'ificamente en 
        la direcci\'on general, donde el robot fue puesto en funcionamiento durante 
        una hora continua. Esta prueba represent\'o un punto crucial en la validaci\'on 
        del sistema, ya que fue la primera vez que el robot oper\'o en un entorno real 
        y complejo, con m\'ultiples obst\'aculos y variaciones en el espacio.
    \vskip 0.5cm
    Durante esta prueba, el robot deb\'ia moverse de manera aut\'onoma a lo largo del espacio, 
        detectando y evitando obst\'aculos como mobiliario, paredes y elementos presentes en 
        el entorno. Adem\'as, este escenario present\'o retos particulares, como la presencia 
        de objetos porosos (mochilas, alfombras) y estructuras delgadas (patas de sillas y mesas), 
        los cuales el sistema ten\'ia dificultades para detectar correctamente debido a las 
        limitaciones del sensor LiDAR.
    \vskip 0.5cm
    A pesar de estos desaf\'ios, el robot logr\'o operar durante una hora sin interrupciones 
        significativas, lo que tambi\'en la posicion\'o como una prueba de carga. Esta fue 
        la primera vez que el sistema estuvo expuesto a un entorno tan din\'amico y real, 
        y aunque se identificaron algunas \'areas de mejora en la detecci\'on de ciertos tipos 
        de obst\'aculos, la prueba fue exitosa en t\'erminos de desempe\~no general y resistencia 
        bajo condiciones prolongadas.
    \vskip 0.5cm
    El proceso de aceptaci\'on estuvo a cargo del profesor encargado del proyecto, quien evalu\'o el 
        rendimiento del robot durante la prueba, verificando su capacidad para navegar de manera 
        aut\'onoma, su resistencia a las condiciones del entorno y su comportamiento ante los obst\'aculos 
        presentes. Tras esta evaluaci\'on, el profesor determin\'o que el sistema cumpl\'ia con los objetivos 
        t\'ecnicos y funcionales esperados, otorgando su aprobaci\'on con observaciones menores para mejorar 
        la precisi\'on en la detecci\'on de obst\'aculos espec\'ificos.
    \vskip 0.5cm
    Este escenario real fue clave para validar que el robot puede operar en condiciones complejas y que est\'a 
        listo para ser implementado en otros entornos similares, lo cual marca un hito importante en el 
        desarrollo del sistema.