\subsection{Evaluaci\'on de segunda iteraci\'on del desempe\~no del robot de pruebas integrales}
En el Cuadro \ref{tab:EvaluacionSegundaIteracion} se muestra la evaluaci\'on de la segunda iteraci\'on del desempe\~no del robot de pruebas integrales. Se evaluaron los siguientes criterios: Precisi\'on de Detecci\'on del LiDAR, Estabilidad del Hardware (Motores y Controlador), Navegaci\'on Aut\'onoma, Reacci\'on a Obst\'aculos Peque\~nos/Din\'amicos, Resistencia al Funcionamiento Prolongado, Sincronizaci\'on de Sensores, Capacidad de Esquivar Obst\'aculos y Consumo Energ\'etico. Se observ\'o que el robot mejor\'o en la mayor\'ia de los criterios, con una puntuaci\'on promedio de 3.7, lo que indica un desempe\~no satisfactorio en la segunda iteraci\'on.
% Tabla de evaluaci\'on de la primera iteraci\'on
\begin{table}[htbp]
    \centering
    \caption{Evaluaci\'on de la segunda iteraci\'on del desempe\~no del robot de pruebas integrales}
    \label{tab:EvaluacionSegundaIteracion}
    \resizebox{\textwidth}{!}{%
    \begin{tabular}{|c|c|c|c|}
    \hline
    \rowcolor[HTML]{C0C0C0} 
    \textbf{Criterio} & \textbf{Descripci\'on} & \textbf{Puntuaci\'on (1-5)} & \textbf{Observaciones} \\ \hline
    \textbf{Precisi\'on de Detecci\'on del LiDAR} & Evaluar la capacidad del LiDAR para detectar obst\'aculos a diversas distancias, especialmente en rangos cortos (<10 cm). & 3.7 & Mejoramos un poco respecto a antes, adem\'as encontramos un caso especial al obtener datos \\ \hline
    \textbf{Estabilidad del Hardware (Motores y Controlador)} & Evaluar si los motores y el controlador operan de manera estable, sin sobrecalentamientos o fallos. & 4 & Se puso un sensor de temperatura para el CPU, los motores funcionan a la perfecci\'on \\ \hline
    \textbf{Navegaci\'on Aut\'onoma} & Evaluar la capacidad del robot para moverse de manera aut\'onoma y evitar obst\'aculos en un entorno controlado. & 3 & Ya se implemento, sin embargo tiene todav\'ia algunas carencias \\ \hline
    \textbf{Reacci\'on a Obst\'aculos Peque\~nos/Din\'amicos} & Evaluar la rapidez y precisi\'on con que el robot detecta y responde a obst\'aculos peque\~nos o en movimiento. & 3.5 & Su rango de detecci\'on ahora es demasiado, gira a demasiada distancia (45cm) \\ \hline
    \textbf{Resistencia al Funcionamiento Prolongado} & Evaluar si el robot puede operar de manera continua sin fallos durante largos per\'iodos de tiempo (pruebas de carga). & 3 & No hay presupuesto para cargador \\ \hline
    \textbf{Sincronizaci\'on de Sensores} & Evaluar la capacidad del sistema para sincronizar correctamente los datos del LiDAR y las c\'amaras (formato YUV/RGB). & 4 & Se hicieron pruebas con otros formatos y seguimos en fase de pruebas \\ \hline
    \textbf{Capacidad de Esquivar Obst\'aculos} & Evaluar si el robot puede evitar colisiones con precisi\'on, bas\'andose en los datos del LiDAR. & 3 & Dura aproximadamente 2 horas encendido. \\ \hline
    \textbf{Consumo Energ\'etico} & Evaluar la eficiencia energ\'etica del sistema durante el funcionamiento prolongado. & 4 & No Aplica \\ \hline
    \end{tabular}%
    }
\end{table}