\subsection{Evaluaci\'on de tercera iteraci\'on del desempe\~no del robot de pruebas integrales}

% Tabla de evaluaci\'on de la primera iteraci\'on
\begin{table}[htbp]
    \centering
    \caption{Evaluaci\'on de la tercera iteraci\'on del desempe\~no del robot de pruebas integrales}
    \label{tab:EvaluacionPrimeraIteracion}
    \resizebox{\textwidth}{!}{%
    \begin{tabular}{|c|c|c|c|}
    \hline
    \rowcolor[HTML]{C0C0C0} 
    \textbf{Criterio} & \textbf{Descripci\'on} & \textbf{Puntuaci\'on (1-5)} & \textbf{Observaciones} \\ \hline
    \textbf{Precisi\'on de Detecci\'on del LiDAR} & Evaluar la capacidad del LiDAR para detectar obst\'aculos a diversas distancias, especialmente en rangos cortos (<10 cm). & 4 & Perfeccionado, sin embargo hay espacio de mejora \\ \hline
    \textbf{Estabilidad del Hardware (Motores y Controlador)} & Evaluar si los motores y el controlador operan de manera estable, sin sobrecalentamientos o fallos. & 4 & No Aplica \\ \hline
    \textbf{Navegaci\'on Aut\'onoma} & Evaluar la capacidad del robot para moverse de manera aut\'onoma y evitar obst\'aculos en un entorno controlado. & 4 & Es mucho mejor que antes, sobre todo ahora puede pasar por lugares estrechos \\ \hline
    \textbf{Reacci\'on a Obst\'aculos Peque\~nos/Din\'amicos} & Evaluar la rapidez y precisi\'on con que el robot detecta y responde a obst\'aculos peque\~nos o en movimiento. & 4.5 & Los rangos fueron calculados a la medida \\ \hline
    \textbf{Resistencia al Funcionamiento Prolongado} & Evaluar si el robot puede operar de manera continua sin fallos durante largos per\'iodos de tiempo (pruebas de carga). & 3 & No hay presupuesto para cargador \\ \hline
    \textbf{Sincronizaci\'on de Sensores} & Evaluar la capacidad del sistema para sincronizar correctamente los datos del LiDAR y las c\'amaras (formato YUV/RGB). & 4 & No aplica \\ \hline
    \textbf{Capacidad de Esquivar Obst\'aculos} & Evaluar si el robot puede evitar colisiones con precisi\'on, bas\'andose en los datos del LiDAR. & 3.5 & Dura aproximadamente 2 horas y media encendido. \\ \hline
    \textbf{Consumo Energ\'etico} & Evaluar la eficiencia energ\'etica del sistema durante el funcionamiento prolongado. & 4 & No Aplica \\ \hline
    \end{tabular}%
    }
\end{table}