\subsection{Pruebas de carga y resistencia en escenarios extendidos para validaci\'on de fallos} % (fold)
    Como parte del proceso de validaci\'on y an\'alisis del sistema, se realizaron pruebas de carga y resistencia en un 
        entorno controlado durante un per\'iodo prolongado de una hora, con el objetivo de identificar posibles 
        fallos o limitaciones del sensor LiDAR en escenarios complejos y extendidos. Estas pruebas se llevaron 
        a cabo en un \'area de oficinas, donde se incluyeron diferentes tipos de objetos y condiciones para evaluar 
        el desempe\~no del LiDAR.
    \vskip 0.5cm
    Durante esta prueba, se observaron algunos problemas relacionados con la capacidad del LiDAR para detectar ciertos 
        elementos porosos y objetos que no se encuentran completamente en su campo de detecci\'on. En particular, el 
        LiDAR present\'o dificultades al identificar objetos como mochilas, que debido a su estructura porosa y 
        materiales absorbentes, no reflejan correctamente las se\~nales del sensor. Asimismo, se observaron 
        problemas al detectar patas de mobiliario muy delgadas, las cuales no eran f\'acilmente captadas 
        debido a su tama\~no reducido y posici\'on fuera del rango \'optimo del sensor.
    \vskip 0.5cm
    Otro tipo de fallos observados se relacionan con objetos que se encuentran por debajo o por encima de 
        la altura del LiDAR, lo que impidi\'o su correcta detecci\'on y mapeo. Estos resultados sugieren que, 
        aunque el LiDAR es efectivo para detectar obst\'aculos dentro de su campo de visi\'on directo, presenta 
        limitaciones significativas cuando se enfrenta a objetos con caracter\'isticas particulares o posiciones 
        fuera de su alcance.
    \vskip 0.5cm
    Para ilustrar estos hallazgos, se ha incluido un video de la prueba, que muestra c\'omo el sistema interact\'ua 
        con diferentes objetos en un entorno de oficina a lo largo del tiempo. En este video, se pueden visualizar 
        claramente los problemas mencionados, as\'i como las \'areas en las que el sistema necesita mejoras para un mejor 
        rendimiento en escenarios m\'as complejos.
    \vskip 0.5cm
    Puedes acceder al video de la prueba en el siguiente enlace: \url{https://drive.google.com/file/d/1Ir2MoquPjXGfqONQ0i3mhBTB7sKAp7e6/view?usp=drive_link}
    \vskip 0.5cm
    Acerc\'andose al final del desarrollo del Trabajo Terminal, se procede con la realizaci\'on de las
    pruebas de carga y resistencia para distintos escenarios en los cuales es importante notar
    c\'omo es que el comportamiento del robot y el algoritmo interact\'uan entre si en espacios un
    poco m\'as extendidos.
    \vskip 0.5cm
    Lo que se busc\'o principalmente es encontrar errores o fallos los cuales supongan un
    detrimento en la funcionalidad del robot, as\'i como posibles fallos a la hora de generar la ruta
    en espacios mucho m\'as grandes, ya que tanto en el simulador como en el robot, el manejar
    espacios los cuales tengan un espacio considerable, hace que sea mucho m\'as f\'acil la
    generaci\'on de errores debido a que se maneja mucha m\'as informaci\'on tanto en el simulador,
    como en la ruta generada por este mismo, lo que hace que sea mucho m\'as dif\'icil controlar el
    flujo de la informaci\'on, as\'i como la interpretaci\'on de esta misma para que el robot pueda
    seguir la ruta correctamente.
    \vskip 0.5cm
    Espec\'ificamente el robot se puso a prueba en un espacio no controlado, con muchas m\'as
    variables a su alrededor. Anteriormente, las pruebas se hac\'ian en entornos donde se pod\'ia
    tener cierta injerencia el manejo de la ruta y del robot para poder evitar accidentes y acciones
    que podr\'ian llegar a causar alg\'un inconveniente a la hora del manejo del robot y de la
    generaci\'on de la ruta, pero esta vez se dej\'o a su propio paso para saber en qu\'e condiciones
    espec\'ificamente se generaban los errores o fallos, para as\'i poder encontrar las soluciones a
    estos fallos y mantenerlos controlados.