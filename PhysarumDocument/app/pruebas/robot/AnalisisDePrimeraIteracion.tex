\subsection{An\'alisis de Primera Iteraci\'on de Datos de Sensores } % (fold)
    En esta primera iteraci\'on, se recopilaron y analizaron los datos de los sensores 
        integrados en el sistema, con el objetivo de comprender el tipo de informaci\'on 
        proporcionada por el LiDAR y las c\'amaras utilizadas, as\'i como su potencial para la 
        implementaci\'on de una interfaz gr\'afica. Esta fase fue crucial, ya que marc\'o el punto 
        de partida para la integraci\'on de los datos sensoriales en un entorno visual que permitiera 
        la interpretaci\'on y el control en tiempo real.
    \vskip 0.5cm
    El sensor LiDAR proporcion\'o datos en formato de coordenadas cartesianas, calculadas a 
        partir de distancias y \'angulos. Estos datos permiten la representaci\'on de un mapa 
        del entorno en dos dimensiones, detectando la proximidad de objetos y obst\'aculos 
        en el \'area circundante. La informaci\'on recopilada incluy\'o principalmente mediciones 
        de distancia en mil\'imetros, asociadas a coordenadas angulares espec\'ificas, las cuales 
        se convirtieron en valores cartesianas para su futura visualizaci\'on en una interfaz gr\'afica. 
        Por ejemplo, las primeras mediciones capturadas del LiDAR ofrec\'ian distancias que oscilaban 
        entre los 20 cm y los 5 metros, brindando una imagen b\'asica del espacio inmediato.
    \vskip 0.5cm
    Por otro lado, se evaluaron diferentes tipos de c\'amaras, las cuales proporcionaban datos en diversos 
        formatos. Se analizaron c\'amaras que ofrec\'ian salidas en formatos como YUV (YUV420, YUV422) y 
        RGB, siendo el formato YUV uno de los m\'as eficientes para transmisi\'on y procesamiento de video, 
        especialmente en aplicaciones que requieren una compresi\'on de alta calidad. Este formato divide la 
        informaci\'on de luminancia (Y) y crominancia (UV), lo que facilita la reducci\'on del ancho de banda 
        necesario sin comprometer la calidad visual en exceso. Durante las pruebas iniciales, el formato YUV422 
        mostr\'o ser particularmente adecuado para entornos donde la transmisi\'on r\'apida de im\'agenes era prioritaria, 
        dado que proporcionaba una mayor eficiencia en t\'erminos de compresi\'on de datos en comparaci\'on con RGB, 
        aunque con menos fidelidad en los detalles de color.
    \vskip 0.5cm
    La c\'amara RGB tambi\'en fue evaluada para obtener im\'agenes con una representaci\'on precisa de los colores del 
        entorno. A diferencia del formato YUV, los datos RGB capturan la totalidad de la informaci\'on de color, 
        lo que result\'o \'util en situaciones donde se requer\'ia una mayor claridad visual para la identificaci\'on 
        de objetos. Sin embargo, el formato RGB mostr\'o ser menos eficiente en t\'erminos de tama\~no de archivo y 
        procesamiento, lo que lo hac\'ia menos adecuado para aplicaciones en tiempo real donde la optimizaci\'on 
        del rendimiento era cr\'itica.
    \vskip 0.5cm
    El an\'alisis de esta primera iteraci\'on fue fundamental para comprender las capacidades y limitaciones de los 
        sensores, lo que permiti\'o decidir sobre los formatos de datos m\'as adecuados para su futura implementaci\'on 
        en una interfaz gr\'afica. Se identific\'o que los datos del LiDAR ofrec\'ian una representaci\'on clara y precisa 
        del entorno inmediato, mientras que las diferentes c\'amaras evaluadas proporcionaban flexibilidad en la 
        captura de informaci\'on visual, dependiendo de los requisitos de resoluci\'on y velocidad de procesamiento.
% subsection An\'alisis de Primera Iteraci\'on de Datos de Sensores  (end)