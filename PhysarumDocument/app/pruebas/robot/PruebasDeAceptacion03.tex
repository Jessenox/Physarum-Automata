\subsection{Pruebas de aceptaci\'on en entornos complejos simulados y reales}
    Las pruebas de aceptaci\'on en entornos complejos se realizaron con el objetivo de verificar 
        que el sistema es capaz de operar bajo condiciones reales y simuladas, cumpliendo con 
        los requisitos funcionales establecidos para su despliegue en escenarios pr\'acticos. Una 
        de las pruebas clave fue la realizada durante el evento en el planetario del IPN, donde el 
        robot estuvo en funcionamiento durante dos horas en un entorno din\'amico y lleno de obst\'aculos, 
        lo que incluy\'o la presencia de ni\~nos en movimiento, muebles y luz solar directa.
    \vskip 0.5cm
    En esta prueba de aceptaci\'on, el robot fue sometido a un entorno complejo y real, lo que permiti\'o evaluar 
        su capacidad para reaccionar ante situaciones imprevistas y la efectividad de su sistema de navegaci\'on 
        aut\'onoma. A pesar de las interferencias causadas por la luz solar en el sensor LiDAR, el sistema logr\'o 
        operar de manera continua y sin fallos cr\'iticos. Sin embargo, se observaron limitaciones en la detecci\'on 
        de obst\'aculos en ciertas situaciones, como cuando los ni\~nos o los objetos estaban fuera del rango del 
        LiDAR o cuando la luz solar directa afectaba su precisi\'on.
    \vskip 0.5cm
    El entorno lleno de personas y la variabilidad de los objetos presentes proporcionaron una simulaci\'on adecuada 
        de escenarios reales en los que el robot podr\'ia operar. La prueba tambi\'en sirvi\'o para observar c\'omo el 
        sistema respond\'ia a la carga continua de funcionamiento prolongado. Aunque no se registraron fallos 
        significativos que impidieran el correcto funcionamiento del robot, el comportamiento observado 
        permiti\'o identificar \'areas de mejora, especialmente en cuanto a la detecci\'on de objetos porosos o 
        muy delgados, y la interferencia de se\~nales infrarrojas provenientes del sol.
    \vskip 0.5cm
    El proceso de aceptaci\'on fue validado por el profesor responsable, quien evalu\'o tanto el desempe\~no 
        del sistema en el entorno real como los datos obtenidos durante la prueba. Tras su an\'alisis, 
        el profesor concluy\'o que el robot cumpl\'ia con los criterios establecidos para su aceptaci\'on, 
        aunque con algunas recomendaciones para mejorar el rendimiento en condiciones extremas, como 
        las observadas en exteriores.
    \vskip 0.5cm
    La aprobaci\'on final del profesor, tras el an\'alisis de los resultados y la revisi\'on del desempe\~no en entornos 
        simulados y reales, confirma que el sistema est\'a listo para su implementaci\'on en otros escenarios 
        operativos m\'as complejos.
% subsection Pruebas de aceptaci\'on en entornos complejos simulados y reales (end)