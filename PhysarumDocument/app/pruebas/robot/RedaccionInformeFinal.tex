\subsection{Redacci\'on del informe final} % (fold)

    Con toda la informaci\'on recabada durante el desarrollo del proyecto, 
        se procede a la elaboraci\'on del informe final, donde se detallan 
        los aspectos fundamentales encontrados a lo largo de las pruebas 
        de funcionamiento y de rendimiento del sistema. En este documento, 
        se integran los resultados de las pruebas unitarias, de integraci\'on, 
        y de aceptaci\'on realizadas sobre el sistema, destacando los avances 
        obtenidos en cada iteraci\'on.

    \vskip 0.5cm

    En esta redacci\'on final, se incluyen los datos recopilados en las pruebas en condiciones 
        reales y simuladas, los cuales permitieron evaluar el desempe\~no del sistema en t\'erminos 
        de eficiencia, tiempo de respuesta, y capacidad de adaptaci\'on a diversos escenarios. 
        Este an\'alisis es crucial para reflejar el cumplimiento de los objetivos del proyecto y 
        para identificar \'areas en las que se pueden aplicar mejoras.

    \vskip 0.5cm

    Asimismo, se confirma la capacidad del sistema para cumplir con su funcionalidad principal 
        de generaci\'on y seguimiento de rutas a partir de puntos de inicio y fin definidos. 
        Este resultado ha demostrado la eficacia del sistema en la simulaci\'on de trayectorias 
        y en la adaptabilidad de su algoritmo de ruteo, consolidando su capacidad operativa en 
        el entorno para el que fue dise\~nado.

    \vskip 0.5cm

    Los escenarios evaluados en diferentes condiciones y configuraciones permitieron identificar 
        los factores que influyen en el rendimiento del sistema. Estas observaciones han revelado 
        ventanas de oportunidad para optimizar los algoritmos y mejorar la precisi\'on del sistema 
        en la generaci\'on de rutas, lo cual ser\'a de gran utilidad para el desarrollo de versiones futuras.

    \vskip 0.5cm

    Finalmente, este informe recopila y documenta los resultados generales del proyecto, 
        destacando el cumplimiento de los objetivos establecidos y los logros alcanzados. 
        Asimismo, ofrece un an\'alisis de las \'areas de mejora identificadas, estableciendo 
        una base para futuras investigaciones y desarrollos que puedan optimizar el sistema y 
        expandir su aplicabilidad en nuevos contextos.
    \vskip 0.5cm