\subsection{Redacci\'on del informe final} % (fold)

    Con toda la informaci\'on recabada durante el desarrollo del proyecto, 
        se procede a la elaboraci\'on del informe final, donde se detallan 
        los aspectos fundamentales encontrados a lo largo de las pruebas 
        de funcionamiento y de rendimiento del sistema. En este documento, 
        se integran los resultados de las pruebas unitarias, de integraci\'on, 
        y de aceptaci\'on realizadas sobre el sistema, destacando los avances 
        obtenidos en cada iteraci\'on.

    \vskip 0.5cm

    En esta redacci\'on final, se incluyen los datos recopilados en las pruebas en condiciones 
        reales y simuladas, los cuales permitieron evaluar el desempe\~no del sistema en t\'erminos 
        de eficiencia, tiempo de respuesta, y capacidad de adaptaci\'on a diversos escenarios. 
        Este an\'alisis es crucial para reflejar el cumplimiento de los objetivos del proyecto y 
        para identificar \'areas en las que se pueden aplicar mejoras.

    \vskip 0.5cm

    Asimismo, se confirma la capacidad del sistema para cumplir con su funcionalidad principal 
        de generaci\'on y seguimiento de rutas a partir de puntos de inicio y fin definidos. 
        Este resultado ha demostrado la eficacia del sistema en la simulaci\'on de trayectorias 
        y en la adaptabilidad de su algoritmo de ruteo, consolidando su capacidad operativa en 
        el entorno para el que fue dise\~nado.

    \vskip 0.5cm

    Los escenarios evaluados en diferentes condiciones y configuraciones permitieron identificar 
        los factores que influyen en el rendimiento del sistema. Estas observaciones han revelado 
        ventanas de oportunidad para optimizar los algoritmos y mejorar la precisi\'on del sistema 
        en la generaci\'on de rutas, lo cual ser\'a de gran utilidad para el desarrollo de versiones futuras.

    \vskip 0.5cm

    Finalmente, este informe recopila y documenta los resultados generales del proyecto, 
        destacando el cumplimiento de los objetivos establecidos y los logros alcanzados. 
        Asimismo, ofrece un an\'alisis de las \'areas de mejora identificadas, estableciendo 
        una base para futuras investigaciones y desarrollos que puedan optimizar el sistema y 
        expandir su aplicabilidad en nuevos contextos.
    \vskip 0.5cm

    En el informe se presentan tambi\'en los detalles de la implementaci\'on t\'ecnica de cada m\'odulo, 
    incluyendo diagramas de flujo, esquemas de comunicaci\'on entre los diferentes componentes 
    del sistema, y las interfaces desarrolladas para garantizar una operaci\'on eficiente. 
    Este nivel de detalle facilita la comprensi\'on del funcionamiento interno del sistema 
    y permite replicar, modificar o mejorar los procesos con base en futuras necesidades.

    \vskip 0.5cm

    Se incluyeron gr\'aficas y tablas que resumen los resultados obtenidos, mostrando 
        comparativas del desempe\~no del sistema bajo diferentes configuraciones y condiciones. 
        Estos elementos visuales ayudan a ilustrar de manera clara y concisa el impacto de cada 
        ajuste realizado y su contribuci\'on al objetivo general del proyecto, lo que facilita su 
        evaluaci\'on por parte de expertos en el campo.

    \vskip 0.5cm

    La secci\'on de conclusiones se centra en destacar los principales hallazgos y en 
        proporcionar recomendaciones espec\'ificas para el uso y mantenimiento del sistema 
        desarrollado. Estas recomendaciones est\'an orientadas a asegurar que el sistema 
        contin\'ue funcionando de manera \'optima y pueda ser escalado o adaptado seg\'un 
        las necesidades futuras de los usuarios o escenarios.

    \vskip 0.5cm

    Durante la redacci\'on se hizo especial \'enfasis en documentar no solo los aciertos, 
        sino tambi\'en los problemas y limitaciones encontradas durante el desarrollo del 
        proyecto. Este enfoque permite que otros investigadores puedan aprender de la experiencia 
        y enfocar sus esfuerzos en soluciones que mejoren el desempe\~no general del sistema.

    \vskip 0.5cm

    El informe tambi\'en contiene un apartado dedicado a las pruebas de validaci\'on, 
        donde se detalla el proceso seguido para verificar la funcionalidad y consistencia 
        del sistema en escenarios controlados y reales. Los resultados obtenidos en estas 
        pruebas proporcionan evidencia de la robustez y flexibilidad del sistema para 
        adaptarse a distintas situaciones operativas.