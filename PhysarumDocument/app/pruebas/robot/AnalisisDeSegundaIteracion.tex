\subsection{An\'alisis de Segunda Iteraci\'on de Datos de Sensores para Implementaci\'on de Interfaz G\'rafica} % (fold)

    En esta secci\'on, se lleva a cabo un an\'alisis detallado de los datos obtenidos durante la 
        segunda iteraci\'on de las pruebas con los sensores LiDAR y la c\'amara RGB. El objetivo 
        principal es evaluar la calidad y precisi\'on de los datos de sensores que ser\'an utilizados 
        posteriormente para la creaci\'on de una interfaz gr\'afica, sin centrarse a\'un en su visualizaci\'on. 
        El an\'alisis se enfoca en los datos de distancia, profundidad y sincronizaci\'on temporal para asegurar 
        una correcta representaci\'on del entorno.
    \vskip 0.5cm
    \begin{enumerate}
        \item Datos de Sensor LiDAR: El sensor LiDAR proporcion\'o datos en forma de coordenadas cartesianas 
            y distancias respecto a objetos en el entorno. Estos datos se organizaron en una matriz que incluye la 
            posici\'on relativa y el \'angulo de cada punto detectado. A continuaci\'on, se presenta un ejemplo de los 
            puntos obtenidos durante esta iteraci\'on:
            \begin{itemize}
                \item Punto 1: Coordenada (0.33839, -0.123582), Distancia: 0.36025 metros, \'angulo: -0.350157 radianes
                \item Punto 2: Coordenada (2.6953, -0.417856), Distancia: 2.7275 metros, \'angulo: -0.153807 radianes
                \item Punto 3: Coordenada (4.96504, 1.94797), Distancia: 5.3335 metros, \'angulo: 0.373882 radianes
            \end{itemize}
        Estos puntos demuestran c\'omo el LiDAR detecta objetos a distintas distancias y \'angulos. 
            En la mayor\'ia de los casos, los datos obtenidos presentan un alto grado de consistencia, 
            lo que indica que el sensor est\'a funcionando dentro de los par\'ametros esperados. Sin embargo, 
            en algunos puntos m\'as lejanos se observaron ligeras variaciones en las distancias medidas, que 
            deber\'an tenerse en cuenta para futuros refinamientos del procesamiento.
        \item Otro aspecto crucial del an\'alisis es la sincronizaci\'on entre los datos del LiDAR y 
            las im\'agenes capturadas por la c\'amara RGB. Durante esta iteraci\'on, se 
            observaron mejoras significativas en la alineaci\'on temporal de los datos de ambos sensores. 
            A continuaci\'on, se presenta un fragmento de los datos sincronizados para un cuadro de video 
            y una serie de lecturas del LiDAR:
            \begin{itemize}
                \item Cuadro RGB (Timestamp: 3860869517)
                \item LiDAR (Timestamp: 3861370540): Coordenada (0.3418, -0.11999), Distancia: 0.36225 metros
            \end{itemize}
        Esta sincronizaci\'on permite correlacionar los datos de profundidad obtenidos por el 
            LiDAR con los elementos visuales capturados por la c\'amara, lo que es esencial para el 
            proceso de construcci\'on de una representaci\'on tridimensional coherente.
        \item An\'alisis de la Precisi\'on: El an\'alisis de la precisi\'on se centr\'o en la 
            consistencia de las distancias medidas por el LiDAR. A distancias cercanas (menores a 1 metro), 
            se observ\'o una baja variabilidad en las lecturas. Por ejemplo, en varios puntos cercanos 
            a 0.36 metros, las diferencias entre lecturas sucesivas fueron de menos de 0.01 metros. 
            A mayores distancias (alrededor de 5 metros), la variabilidad aument\'o ligeramente, 
            pero se mantuvo dentro de un margen aceptable para la mayor\'ia de los usos.
            \begin{itemize}
                \item Punto a 5 metros: Coordenada (4.96504, 1.94797), Distancia: 5.3335 metros, variaci\'on de +-0.03 metros entre lecturas sucesivas.
            \end{itemize}
            Este an\'alisis sugiere que el LiDAR est\'a calibrado adecuadamente para distancias cortas y medias, 
                pero podr\'ia requerir ajustes para mejorar la precisi\'on en distancias mayores.
    \end{enumerate}
