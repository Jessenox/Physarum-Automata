\subsection{Desarrollo de m\'odulos de hardware del robot para pruebas de integraci\'on}
\label{sec:Desarrollo de m\'odulos de hardware del robot para pruebas de integraci\'on}
    % Parrafo 1
    En la siguiente secci\'on vamos a detallar el proceso de implementaci\'on de los m\'odulos
        de hardware que componen el robot. Se describir\'an las decisiones t\'ecnicas tomadas
        durante la construcci\'on, la integraci\'on f\'isica y funcional de los m\'odulos, y los
        resultados preliminares de las pruebas iniciales. El objetivo es mostrar c\'omo los
        m\'odulos de actuadores, sensores, la unidad de control y sistema de visualizaci\'on
        fueron desarrollados para llevar a cabo las pruebas de integraci\'on.
    \vskip 0.5cm
    \subsubsection{M\'odulo de actuadores}
    El m\'odulo de actuadores est\'a compuesto por cuatro motores Nema 23 conectados a
        ruedas omnidireccionales. Los motores se montaron sobre una base de perfiles de
        aluminio 2040, elegidos por su resistencia y ligereza. Los motores se conectaron a susrespectivos controladores mediante cables blindados para reducir interferencias
        electromagn\'eticas.
    \vskip 0.5cm
    Para el control de los motores, se utiliz\'o la Raspberry Pi, que env\'ia se\~nales PWM a
        trav\'es de sus pines GPIO, controlando la velocidad y direcci\'on de los motores. Se
        ajust\'o la frecuencia de las se\~nales PWM a 400 Hz, lo que permiti\'o un control suave y
        preciso del movimiento del robot.
    \vskip 0.5cm
    Para poder controlar los motores, se desarroll\'o un c\'odigo en C++ que utiliz\'o la
        biblioteca pigpio para gestionar las se\~nales PWM. El c\'odigo permite ajustar tanto la
        velocidad como la direcci\'on de los motores en tiempo real.
    \vskip 0.5cm
    \begin{lstlisting}[language={C++}, caption={C\'odigo de ejemplo de motores}, label={Script}]
        void setMotorSpeed(int motor, int frequency) {
            if (motor >= 0 && motor < 4) {
                gpioSetPWMfrequency(PWM_PINS[motor], frequency);
            }
        }
        void setMotorDirection(int motor, int direction) {
            if (motor >= 0 && motor < 4) {
                gpioWrite(DIR_PINS[motor], direction);
            }
        }
    \end{lstlisting}
    \vskip 0.5cm
    \subsubsection{M\'odulo de sensores}
    El sensor LiDAR YDLIDAR 2XL fue montado en la parte superior del robot para obtener
        un \'angulo de escaneo de 360 grados. El sensor se conect\'o a la Raspberry Pi a trav\'es de
        un puerto USB A. La configuraci\'on f\'isica permiti\'o que el LiDAR tuviera una vista
        despejada del entorno, crucial para la detecci\'on precisa de obst\'aculos.
    \vskip 0.5cm
    El sensor LiDAR se configur\'o utilizando la biblioteca proporcionada por el fabricante
        (YLiDAR SDK) y se desarroll\'o un c\'odigo para la lectura continua de los datos de escaneo.
        Los datos obtenidos del LiDAR se procesaron en tiempo real para permitir la
        navegaci\'on aut\'onoma y la detecci\'on de obst\'aculos.
    \vskip 0.5cm
    \begin{lstlisting}[language={C++}, caption={Primero se implementa la librer\'ia del LiDAR}, label={Script}]
        #include "CYdLidar.h"
        using namespace ydlidar;
    \end{lstlisting}
    \vskip 0.5cm
    Despu\'es se inicializa el sensor y se obtienen los datos de escaneo:
    \vskip 0.5cm
    \begin{lstlisting}[language={C++}, caption={Configuraci\'on de opciones del LiDAR}, label={Script}]
        CYdLidar laser;
        laser.setlidaropt(LidarPropSerialPort, port.c_str(), port.size());
        std::string ignore_array;
        ignore_array.clear();
        laser.setlidaropt(LidarPropIgnoreArray, ignore_array.c_str(), ignore_array.size());

        laser.setlidaropt(LidarPropSerialBaudrate, &baudrate, sizeof(int));
        int optval = TYPE_TRIANGLE;
        laser.setlidaropt(LidarPropLidarType, &optval, sizeof(int));
        optval = YDLIDAR_TYPE_SERIAL;
        laser.setlidaropt(LidarPropDeviceType, &optval, sizeof(int));
        optval = isSingleChannel ? 3 : 4;
        laser.setlidaropt(LidarPropSampleRate, &optval, sizeof(int));
        optval = 4;
        laser.setlidaropt(LidarPropAbnormalCheckCount, &optval, sizeof(int));

        bool b_optvalue = false;
        laser.setlidaropt(LidarPropFixedResolution, &b_optvalue, sizeof(bool));
        laser.setlidaropt(LidarPropReversion, &b_optvalue, sizeof(bool));
        laser.setlidaropt(LidarPropInverted, &b_optvalue, sizeof(bool));
        b_optvalue = true;
        laser.setlidaropt(LidarPropAutoReconnect, &b_optvalue, sizeof(bool));
        laser.setlidaropt(LidarPropSingleChannel, &isSingleChannel, sizeof(bool));
        b_optvalue = false;
        laser.setlidaropt(LidarPropIntenstiy, &b_optvalue, sizeof(bool));
        b_optvalue = true;
        laser.setlidaropt(LidarPropSupportMotorDtrCtrl, &b_optvalue, sizeof(bool));
        b_optvalue = false;
        laser.setlidaropt(LidarPropSupportHeartBeat, &b_optvalue, sizeof(bool));

        float f_optvalue = 180.0f;
        laser.setlidaropt(LidarPropMaxAngle, &f_optvalue, sizeof(float));
        f_optvalue = -180.0f;
        laser.setlidaropt(LidarPropMinAngle, &f_optvalue, sizeof(float));
        f_optvalue = 64.0f;
        laser.setlidaropt(LidarPropMaxRange, &f_optvalue, sizeof(float));
        f_optvalue = 0.05f;
        laser.setlidaropt(LidarPropMinRange, &f_optvalue, sizeof(float));
        laser.setlidaropt(LidarPropScanFrequency, &frequency, sizeof(float));

    \end{lstlisting}
    \vskip 0.5cm
    \subsubsection{Unidad de control}
    La Raspberry Pi 4 B fue montada en el centro del robot para minimizar la longitud de
        los cables de conexi\'on hacia los motores, el LiDAR y la c\'amara. Se utiliz\'o una
        estructura de soporte para asegurar la Raspberry Pi en su lugar, proporcionando un
        acceso f\'acil a sus puertos GPIO y UART.
    \vskip 0.5cm
    Se desarroll\'o un sistema de control centralizado en la Raspberry Pi, utilizando
        programaci\'on en C++ para gestionar las se\~nales enviadas a los actuadores y procesarlos datos de los sensores. Este sistema permite la toma de decisiones en tiempo real,
        como ajustar la trayectoria del robot seg\'un los datos del LiDAR.
    \vskip 0.5cm
    \begin{lstlisting}[language={C++}, caption={C\'odigo de ejemplo de Kinect y LiDAR}, label={Script}]
        #include "CYdLidar.h"
        #include <string>
        #include <map>
        #include <iostream>
        #include <SFML/Graphics.hpp>
        #include <opencv2/opencv.hpp>
        #include <libfreenect.hpp>
        #include <thread>
        #include <atomic>
        #include <vector>

        using namespace std;
        using namespace ydlidar;
        using namespace cv;

        #if defined(_MSC_VER)
        #pragma comment(lib, "ydlidar_sdk.lib")
        #endif

        // Freenect variables
        freenect_context *f_ctx;
        freenect_device *f_dev;
        Mat depthMat(Size(640, 480), CV_16UC1);
        Mat rgbMat(Size(640, 480), CV_8UC4, Scalar(0, 0, 0, 0)); // Initialize with transparent
        atomic<bool> is_running(true);

        void depth_cb(freenect_device *dev, void *depth, uint32_t timestamp) {
            Mat depth_tmp(Size(640, 480), CV_16UC1, depth);
            depth_tmp.copyTo(depthMat);
            cout << "Depth frame received at timestamp: " << timestamp << endl;
        }

        void rgb_cb(freenect_device *dev, void *rgb, uint32_t timestamp) {
            Mat rgb_tmp(Size(640, 480), CV_8UC3, rgb);
            cvtColor(rgb_tmp, rgbMat, COLOR_RGB2RGBA);
            cout << "RGB frame received at timestamp: " << timestamp << endl;
        }

        void drawGrid(sf::RenderTarget &target, float gridSpacing, float offsetX, float offsetY) {
            sf::Color gridColor(200, 200, 200); // Light gray color
            for (float x = offsetX; x < target.getSize().x; x += gridSpacing) {
                sf::Vertex line[] = {
                    sf::Vertex(sf::Vector2f(x, 0), gridColor),
                    sf::Vertex(sf::Vector2f(x, target.getSize().y), gridColor)
                };
                target.draw(line, 2, sf::Lines);
            }
            for (float x = offsetX; x > 0; x -= gridSpacing) {
                sf::Vertex line[] = {
                    sf::Vertex(sf::Vector2f(x, 0), gridColor),
                    sf::Vertex(sf::Vector2f(x, target.getSize().y), gridColor)
                };
                target.draw(line, 2, sf::Lines);
            }
            for (float y = offsetY; y < target.getSize().y; y += gridSpacing) {
                sf::Vertex line[] = {
                    sf::Vertex(sf::Vector2f(0, y), gridColor),
                    sf::Vertex(sf::Vector2f(target.getSize().x, y), gridColor)
                };
                target.draw(line, 2, sf::Lines);
            }
            for (float y = offsetY; y > 0; y -= gridSpacing) {
                sf::Vertex line[] = {
                    sf::Vertex(sf::Vector2f(0, y), gridColor),
                    sf::Vertex(sf::Vector2f(target.getSize().x, y), gridColor)
                };
                target.draw(line, 2, sf::Lines);
            }
        }

        sf::Color getPointColor(float distance, float maxRange) {
            float ratio = distance / maxRange;
            return sf::Color(255 * (1 - ratio), 255 * ratio, 0); // Color from red to green
        }

        void drawNorthIndicator(sf::RenderTexture &lidarTexture, float offsetX, float offsetY) {
            sf::Color northColor(255, 0, 0); // Red color for the north indicator
            float length = 50.0f; // Length of the north indicator line

            sf::Vertex line[] = {
                sf::Vertex(sf::Vector2f(offsetX, offsetY), northColor),
                sf::Vertex(sf::Vector2f(offsetX, offsetY - length), northColor) // Line pointing upwards
            };
            lidarTexture.draw(line, 2, sf::Lines);
        }

        void detectAndDrawDepthObstacles(Mat &depthMat, sf::RenderTexture &lidarTexture, float offsetX, float offsetY, float scale, vector<Point2f> &depthObstacles) {
            for (int y = 0; y < depthMat.rows; y += 10) {
                for (int x = 0; x < depthMat.cols; x += 10) {
                    uint16_t depth_value = depthMat.at<uint16_t>(y, x);
                    if (depth_value > 0 && depth_value < 2047) {
                        float depth_in_meters = depth_value / 1000.0f;

                        float adjustedX = offsetX + (x - depthMat.cols / 2) * scale / depth_in_meters;
                        float adjustedY = offsetY - (y - depthMat.rows / 2) * scale / depth_in_meters; // Invert Y to match screen coordinates

                        sf::CircleShape circle(3); // Small circle to represent the obstacle
                        circle.setPosition(adjustedX, adjustedY);
                        circle.setFillColor(sf::Color::Red);

                        lidarTexture.draw(circle);
                        
                        // Add the obstacle to the vector
                        depthObstacles.push_back(Point2f(adjustedX, adjustedY));
                    }
                }
            }
        }

        void compareObstacles(const vector<Point2f> &depthObstacles, const vector<Point2f> &lidarObstacles) {
            cout << "Comparing Obstacles:" << endl;
            for (const auto &d : depthObstacles) {
                bool matchFound = false;
                for (const auto &l : lidarObstacles) {
                    float distance = sqrt(pow(d.x - l.x, 2) + pow(d.y - l.y, 2));
                    if (distance < 10.0f) { // If the distance is less than a threshold, consider it a match
                        matchFound = true;
                        break;
                    }
                }
                if (matchFound) {
                    cout << "Match found for depth obstacle at (" << d.x << ", " << d.y << ")" << endl;
                } else {
                    cout << "No match for depth obstacle at (" << d.x << ", " << d.y << ")" << endl;
                }
            }
        }

        void freenect_thread_func() {
            while (is_running) {
                freenect_process_events(f_ctx);
            }
        }

        void print_frame_modes() {
            cout << "Available video modes:" << endl;
            for (int res = FREENECT_RESOLUTION_LOW; res <= FREENECT_RESOLUTION_HIGH; ++res) {
                for (int fmt = FREENECT_VIDEO_RGB; fmt <= FREENECT_VIDEO_IR_8BIT; ++fmt) {
                    freenect_frame_mode mode = freenect_find_video_mode((freenect_resolution)res, (freenect_video_format)fmt);
                    if (mode.is_valid) {
                        cout << "Resolution: " << res << ", Format: " << fmt << ", Width: " << mode.width << ", Height: " << mode.height << ", Bytes per pixel: " << mode.data_bits_per_pixel << endl;
                    }
                }
            }

            cout << "Available depth modes:" << endl;
            for (int res = FREENECT_RESOLUTION_LOW; res <= FREENECT_RESOLUTION_HIGH; ++res) {
                for (int fmt = FREENECT_DEPTH_11BIT; fmt <= FREENECT_DEPTH_REGISTERED; ++fmt) {
                    freenect_frame_mode mode = freenect_find_depth_mode((freenect_resolution)res, (freenect_depth_format)fmt);
                    if (mode.is_valid) {
                        cout << "Resolution: " << res << ", Format: " << fmt << ", Width: " << mode.width << ", Height: " << mode.height << ", Bytes per pixel: " << mode.data_bits_per_pixel << endl;
                    }
                }
            }
        }

        int main(int argc, char *argv[]) {
            std::string port;
            ydlidar::os_init();

            // Initialize Freenect
            if (freenect_init(&f_ctx, NULL) < 0) {
                cerr << "Freenect init failed" << endl;
                return -1;
            }
            freenect_set_log_level(f_ctx, FREENECT_LOG_DEBUG);
            int nr_devices = freenect_num_devices(f_ctx);
            if (nr_devices < 1) {
                cerr << "No Kinect devices found" << endl;
                return -1;
            }
            if (freenect_open_device(f_ctx, &f_dev, 0) < 0) {
                cerr << "Could not open Kinect device" << endl;
                return -1;
            }

            // Print available frame modes
            print_frame_modes();

            // Set the video and depth modes
            freenect_frame_mode video_mode = freenect_find_video_mode(FREENECT_RESOLUTION_MEDIUM, FREENECT_VIDEO_RGB);
            if (!video_mode.is_valid) {
                cerr << "Invalid video mode" << endl;
                return -1;
            }
            freenect_frame_mode depth_mode = freenect_find_depth_mode(FREENECT_RESOLUTION_MEDIUM, FREENECT_DEPTH_11BIT);
            if (!depth_mode.is_valid) {
                cerr << "Invalid depth mode" << endl;
                return -1;
            }

            if (freenect_set_video_mode(f_dev, video_mode) < 0) {
                cerr << "Failed to set video mode" << endl;
                return -1;
            }

            if (freenect_set_depth_mode(f_dev, depth_mode) < 0) {
                cerr << "Failed to set depth mode" << endl;
                return -1;
            }

            freenect_set_depth_callback(f_dev, depth_cb);
            freenect_set_video_callback(f_dev, rgb_cb);
            freenect_start_depth(f_dev);
            freenect_start_video(f_dev);

            // Start Kinect processing thread
            std::thread freenect_thread(freenect_thread_func);

            // Get available LiDAR ports
            std::map<std::string, std::string> ports = ydlidar::lidarPortList();
            if (ports.size() == 1) {
                port = ports.begin()->second;
            } else {
                int id = 0;
                for (auto it = ports.begin(); it != ports.end(); it++) {
                    printf("%d. %s\n", id, it->first.c_str());
                    id++;
                }
                if (ports.empty()) {
                    printf("No LiDAR was detected. Please enter the LiDAR serial port:");
                    std::cin >> port;
                } else {
                    while (ydlidar::os_isOk()) {
                        printf("Please select the LiDAR port:");
                        std::string number;
                        std::cin >> number;
                        if ((size_t)atoi(number.c_str()) >= ports.size()) continue;
                        auto it = ports.begin();
                        id = atoi(number.c_str());
                        while (id) {
                            id--;
                            it++;
                        }
                        port = it->second;
                        break;
                    }
                }
            }

            // Baud rate selection
            int baudrate = 115200;
            printf("Baudrate: %d\n", baudrate); 
            if (!ydlidar::os_isOk()) {
                return 0;
            }

            // Check for single channel communication
            bool isSingleChannel = false;
            std::string input_channel;
            isSingleChannel = true;

            if (!ydlidar::os_isOk()) {
                return 0;
            }

            // Scan frequency
            float frequency = 8.0;
            while (ydlidar::os_isOk() && !isSingleChannel) {
                printf("Please enter the LiDAR scan frequency[5-12]:");
                std::string input_frequency;
                std::cin >> input_frequency;
                frequency = atof(input_frequency.c_str());
                if (frequency <= 12 && frequency >= 5.0) {
                    break;
                }
                fprintf(stderr, "Invalid scan frequency. The scanning frequency range is 5 to 12 Hz. Please re-enter.\n");
            }

            if (!ydlidar::os_isOk()) {
                return 0;
            }

            CYdLidar laser;
            //////////////////////string property/////////////////
            laser.setlidaropt(LidarPropSerialPort, port.c_str(), port.size());
            std::string ignore_array;
            ignore_array.clear();
            laser.setlidaropt(LidarPropIgnoreArray, ignore_array.c_str(), ignore_array.size());

            //////////////////////int property/////////////////
            laser.setlidaropt(LidarPropSerialBaudrate, &baudrate, sizeof(int));
            int optval = TYPE_TRIANGLE;
            laser.setlidaropt(LidarPropLidarType, &optval, sizeof(int));
            optval = YDLIDAR_TYPE_SERIAL;
            laser.setlidaropt(LidarPropDeviceType, &optval, sizeof(int));
            optval = isSingleChannel ? 3 : 4;
            laser.setlidaropt(LidarPropSampleRate, &optval, sizeof(int));
            optval = 4;
            laser.setlidaropt(LidarPropAbnormalCheckCount, &optval, sizeof(int));

            //////////////////////bool property/////////////////
            bool b_optvalue = false;
            laser.setlidaropt(LidarPropFixedResolution, &b_optvalue, sizeof(bool));
            laser.setlidaropt(LidarPropReversion, &b_optvalue, sizeof(bool));
            laser.setlidaropt(LidarPropInverted, &b_optvalue, sizeof(bool));
            b_optvalue = true;
            laser.setlidaropt(LidarPropAutoReconnect, &b_optvalue, sizeof(bool));
            laser.setlidaropt(LidarPropSingleChannel, &isSingleChannel, sizeof(bool));
            b_optvalue = false;
            laser.setlidaropt(LidarPropIntenstiy, &b_optvalue, sizeof(bool));
            b_optvalue = true;
            laser.setlidaropt(LidarPropSupportMotorDtrCtrl, &b_optvalue, sizeof(bool));
            b_optvalue = false;
            laser.setlidaropt(LidarPropSupportHeartBeat, &b_optvalue, sizeof(bool));

            //////////////////////float property/////////////////
            float f_optvalue = 180.0f;
            laser.setlidaropt(LidarPropMaxAngle, &f_optvalue, sizeof(float));
            f_optvalue = -180.0f;
            laser.setlidaropt(LidarPropMinAngle, &f_optvalue, sizeof(float));
            f_optvalue = 64.0f;
            laser.setlidaropt(LidarPropMaxRange, &f_optvalue, sizeof(float));
            f_optvalue = 0.05f;
            laser.setlidaropt(LidarPropMinRange, &f_optvalue, sizeof(float));
            laser.setlidaropt(LidarPropScanFrequency, &frequency, sizeof(float));

            // Initialize LiDAR
            bool ret = laser.initialize();
            if (ret) {
                // Start scanning
                ret = laser.turnOn();
            } else {
                cerr << "Error initializing YDLIDAR: " << laser.DescribeError() << endl;
                return -1;
            }

            sf::RenderWindow window(sf::VideoMode(1280, 720), "Camera and LiDAR Visualization");

            // Create SFML texture and sprite for the camera feed
            sf::Texture cameraTexture;
            sf::Sprite cameraSprite;

            // Main loop
            while (ret && window.isOpen() && ydlidar::os_isOk()) {
                sf::Event event;
                while (window.pollEvent(event)) {
                    if (event.type == sf::Event::Closed)
                        window.close();
                }

                LaserScan scan;

                if (laser.doProcessSimple(scan)) {
                    window.clear();

                    // Update the camera texture with the frame data
                    if (!cameraTexture.create(rgbMat.cols, rgbMat.rows)) {
                        cerr << "Failed to create texture" << endl;
                        break;
                    }
                    cameraTexture.update(rgbMat.ptr());

                    cameraSprite.setTexture(cameraTexture);
                    cameraSprite.setScale(
                        window.getSize().x / static_cast<float>(cameraTexture.getSize().x),
                        window.getSize().y / static_cast<float>(cameraTexture.getSize().y)
                    );

                    // Draw the camera feed
                    window.draw(cameraSprite);

                    // Draw the LiDAR minimap
                    sf::RenderTexture lidarTexture;
                    lidarTexture.create(300, 300);
                    lidarTexture.clear(sf::Color::White);

                    float gridSpacing = 20.0f; // Grid spacing in pixels
                    float offsetX = 150.0f; // Center of the minimap
                    float offsetY = 150.0f; // Center of the minimap
                    float scale = 50.0f; // Scale for LiDAR points

                    drawGrid(lidarTexture, gridSpacing, offsetX, offsetY);
                    drawNorthIndicator(lidarTexture, offsetX, offsetY);

                    // Detect and draw obstacles from the depth sensor
                    vector<Point2f> depthObstacles;
                    detectAndDrawDepthObstacles(depthMat, lidarTexture, offsetX, offsetY, scale, depthObstacles);

                    // Draw the LiDAR itself (center point)
                    sf::CircleShape lidarShape(5); // Larger circle for the center
                    lidarShape.setFillColor(sf::Color::Blue); // Blue color for the center
                    lidarShape.setPosition(offsetX - lidarShape.getRadius(), offsetY - lidarShape.getRadius());
                    lidarTexture.draw(lidarShape);

                    // Draw the points from LiDAR
                    vector<Point2f> lidarObstacles;
                    for (const auto& point : scan.points) {
                        // Convert polar coordinates to Cartesian coordinates
                        float x = point.range * cos(point.angle);
                        float y = point.range * sin(point.angle);

                        // Scale and translate points to fit minimap
                        float adjustedX = offsetX + x * scale;
                        float adjustedY = offsetY - y * scale; // Invert Y to match screen coordinates

                        // Set point color based on distance
                        sf::Color pointColor = getPointColor(point.range, 64.0f); // Assuming max range is 64 meters

                        sf::CircleShape circle(2); // Small circle to represent the point
                        circle.setPosition(adjustedX, adjustedY);
                        circle.setFillColor(pointColor);

                        lidarTexture.draw(circle);

                        // Add the obstacle to the vector
                        lidarObstacles.push_back(Point2f(adjustedX, adjustedY));
                    }

                    // Compare the obstacles detected by the depth sensor and LiDAR
                    compareObstacles(depthObstacles, lidarObstacles);

                    lidarTexture.display();
                    sf::Sprite lidarSprite(lidarTexture.getTexture());
                    lidarSprite.setPosition(10, 10); // Position the minimap at the top-left corner
                    window.draw(lidarSprite);

                    window.display();
                } else {
                    cerr << "Failed to get LiDAR data" << endl;
                }
            }

            // Stop scanning
            laser.turnOff();
            laser.disconnecting();
            is_running = false;
            freenect_thread.join();
            freenect_stop_depth(f_dev);
            freenect_stop_video(f_dev);
            freenect_close_device(f_dev);
            freenect_shutdown(f_ctx);

            return 0;
        }

    \end{lstlisting}

    \subsubsection{Sistema de visualizaci\'on}
    El sistema de visualizaci\'on est\'a compuesto por una c\'amara infrarroja de 5 MP
        conectada a la Raspberry Pi mediante el puerto CSI. La c\'amara fue montada en la parte
        frontal del robot, permitiendo la captura de im\'agenes en tiempo real.
    \vskip 0.5cm
    Se desarroll\'o un c\'odigo en C++ utilizando la biblioteca OpenCV para capturar y
        procesar las im\'agenes de la c\'amara. El sistema de visualizaci\'on muestra la imagen
        capturada en una ventana de la interfaz gr\'afica, permitiendo la observaci\'on del entorno
        del robot. El c\'odigo de ejemplo se muestra en el script anterior.
    \vskip 0.5cm
    \subsubsection{Integraci\'on de m\'odulos}
    Todos los m\'odulos fueron integrados f\'isicamente en la estructura del robot. Se utiliz\'o
        una disposici\'on centralizada para la Raspberry Pi, con los cables de conexi\'on
        organizados de manera que se minimicen las interferencias y se optimice el espacio
        dentro del chasis del robot.
    \vskip 0.5cm
    Se integraron todos los sistemas de control, detecci\'on de obst\'aculos y visualizaci\'on,
        permitiendo que el robot funcione de manera aut\'onoma. La coordinaci\'on entre los
        motores, el sensor LiDAR y la c\'amara se logr\'o mediante la programaci\'on en la
        Raspberry Pi.
    \vskip 0.5cm
    %Listings 
    \begin{lstlisting}[language={C++}, caption={C\'odigo de pruebas de integraci\'organismo}, label={Script}]
        #include <iostream>
        #include <cassert>
        #include <thread>
        #include <chrono>
        #include <atomic>
        #include "CYdLidar.h"
        #include <pigpio.h>

        using namespace std;
        using namespace ydlidar;

        // Prototipos de funciones (se extraen del c\'odigo original para modularizaci\'on)
        void setMotorSpeed(int motor, int frequency);
        void setMotorDirection(int motor, int direction);
        void stopMotors();
        void moveForward();
        void moveBackward();
        void turnLeft();
        void turnRight();
        void testMotors();
        void testLidar();
        void testLidarObstacleDetection(CYdLidar &laser);

        // Variables globales de prueba
        std::atomic<bool> is_running(true);
        std::atomic<bool> is_manual_mode(true);

        // Funci\'on de prueba para el control de motores
        void testMotors() {
            // Inicializar pigpio
            assert(gpioInitialise() >= 0 && "Error al inicializar pigpio.");

            // Configurar pines de direcci\'on como salida
            for (int i = 0; i < 4; ++i) {
                gpioSetMode(DIR_PINS[i], PI_OUTPUT);
                gpioSetMode(PWM_PINS[i], PI_OUTPUT);
                setMotorSpeed(i, frequency);  // Inicializar PWM con frecuencia inicial
            }

            std::cout << "Prueba: Movimientos b\'asicos de los motores" << std::endl;

            // Prueba de movimiento hacia adelante
            moveForward();
            std::this_thread::sleep_for(std::chrono::seconds(2));
            assert(is_manual_mode == true && "Error: el modo manual no est\'a activo durante la prueba de movimiento hacia adelante.");

            // Prueba de movimiento hacia atr\'as
            moveBackward();
            std::this_thread::sleep_for(std::chrono::seconds(2));

            // Prueba de giro a la izquierda
            turnLeft();
            std::this_thread::sleep_for(std::chrono::seconds(2));

            // Prueba de giro a la derecha
            turnRight();
            std::this_thread::sleep_for(std::chrono::seconds(2));

            // Detener los motores
            stopMotors();
            std::cout << "Prueba de motores completada correctamente." << std::endl;

            gpioTerminate();
        }

        // Funci\'on de prueba para el sensor LiDAR
        void testLidar() {
            std::string port;
            ydlidar::os_init();

            // Obtener los puertos disponibles de LiDAR
            std::map<std::string, std::string> ports = ydlidar::lidarPortList();
            if (ports.size() > 1) {
                auto it = ports.begin();
                std::advance(it, 1); // Selecciona el segundo puerto disponible
                port = it->second;
            } else if (ports.size() == 1) {
                port = ports.begin()->second;
            } else {
                std::cerr << "No se detect\'o ning\'un LiDAR. Verifica la conexi\'on." << std::endl;
                assert(false && "Error: No se detect\'o LiDAR.");
                return;
            }

            CYdLidar laser;
            int baudrate = 115200;
            laser.setlidaropt(LidarPropSerialPort, port.c_str(), port.size());
            laser.setlidaropt(LidarPropSerialBaudrate, &baudrate, sizeof(int));

            bool isSingleChannel = true;
            laser.setlidaropt(LidarPropSingleChannel, &isSingleChannel, sizeof(bool));

            float max_range = 8.0f;
            float min_range = 0.1f;
            float max_angle = 180.0f;
            float min_angle = -180.0f;
            float frequency = 8.0f;

            laser.setlidaropt(LidarPropMaxRange, &max_range, sizeof(float));
            laser.setlidaropt(LidarPropMinRange, &min_range, sizeof(float));
            laser.setlidaropt(LidarPropMaxAngle, &max_angle, sizeof(float));
            laser.setlidaropt(LidarPropMinAngle, &min_angle, sizeof(float));
            laser.setlidaropt(LidarPropScanFrequency, &frequency, sizeof(float));

            // Inicializar LiDAR
            assert(laser.initialize() && "Error al inicializar el LiDAR.");
            assert(laser.turnOn() && "Error al encender el LiDAR.");

            std::cout << "Prueba: LiDAR encendido y funcionando." << std::endl;

            // Simular un escaneo
            LaserScan scan;
            assert(laser.doProcessSimple(scan) && "Error al procesar el escaneo LiDAR.");
            std::cout << "Prueba de escaneo LiDAR completada correctamente." << std::endl;

            // Apagar LiDAR
            laser.turnOff();
            laser.disconnecting();
        }

        // Funci\'on de prueba para la detecci\'on de obst\'aculos con LiDAR
        void testLidarObstacleDetection(CYdLidar &laser) {
            // Prueba de detecci\'on de obst\'aculos con el LiDAR
            LaserScan scan;
            if (laser.doProcessSimple(scan)) {
                bool obstacle_detected = false;
                for (const auto &point : scan.points) {
                    if (point.range > 0 && point.range < 0.30) { // Condici\'on de obst\'aculo
                        obstacle_detected = true;
                        break;
                    }
                }
                assert(obstacle_detected && "Error: No se detect\'o ning\'un obst\'aculo cuando se esperaba.");
                std::cout << "Prueba: Detecci\'on de obst\'aculos correcta." << std::endl;
            }
        }

        int main() {
            std::cout << "Iniciando pruebas de integraci\'on..." << std::endl;

            // Ejecutar prueba de motores
            testMotors();

            // Ejecutar prueba de LiDAR
            testLidar();

            // Ejecutar prueba de detecci\'on de obst\'aculos (simulada)
            CYdLidar laser;
            testLidarObstacleDetection(laser);

            std::cout << "Todas las pruebas de integraci\'on se completaron correctamente." << std::endl;

            return 0;
        }


    \end{lstlisting}
    \vskip 0.5cm
