\subsection{Recopilaci\'on y an\'alisis de datos del rendimiento del robot de IEEE 1872.1 e IEEE 2914}
    En el mundo de las pruebas y otro tipo de documentaci\'on,
        son importantes las normas, debido a que permiten un
        est\'andar para el manejo de distintos elementos dentro de un
        sistema, adem\'as de que la estandarizaci\'on fomenta el
        intercambio sencillo o interacci\'on entre componentes los cuales no tienen mucho que ver con otros, pero que al existir
        una norma, un modo de estandarizaci\'on, se pueden crear
        formas en las que ambas partes puedan coexistir entre s\'i y
        entre muchas otras m\'as.
    \vskip 0.5cm
    La importancia de la IEEE 1872.1 radica en que esta norma
        nos permite definir ciertos elementos y contextos en los
        cuales se puede realizar un intercambio de informaci\'on entre
        pares, los cuales usualmente est\'an relacionados a la
        electr\'onica y rob\'otica, as\'i como al hardware y software.
        En el caso del software y del robot al cual el software le
        genera una ruta para su posterior seguimiento, usa criterios
        de interoperabilidad que son los m\'as comunes a la hora de
        poder conectar los resultados obtenidos por el programa, as\'i
        como la forma de interpretar esos datos para cumplir el
        objetivo principal del algoritmo de Physarum: generar una
        ruta la cual el robot pueda seguir sin problemas para llegar a
        un destino final.
    \vskip 0.5cm
    Para el caso de el env\'io de informaci\'on entre
        el robot y el software, se hace uso de herramientas FTP, as\'i
        como de un servidor el cual recibe peticiones HTTP con las
        cuales se pueden recibir y enviar los datos que sean
        necesarios.
    \vskip 0.5cm
    A pesar de que el software pertenezca a una de las jerarqu\'ias
        m\'as bajas dentro de la estructura de la IEEE, es de vital
        importancia que \'este ultimo funcione correctamente y
        cumpla su funci\'on para que el robot tenga un prop\'osito y
        pueda tener la utilidad necesaria.
    \vskip 0.5cm
    Y dentro de las dem\'as normas, tambi\'en se toma en cuenta la
        norma IEEE 2914, que dentro de la parte generadora de la
        ruta y dem\'as trabajo del software de ruteo a partir del
        Physarum, se vuelve importante tomarla en cuenta. Esto
        porque a diferencia de otros sistemas donde la toma de
        decisiones trata de ser siempre la \'optima y en muchos casos
        est\'an se pueden llegar a replicar para tener un registro y una
        idea de que decisiones te llevan a un determinado resultado,
        en el caso del simulador del Physarum las decisiones son
        aleatorias en todo momento, adem\'as, de que la aleatoriedad
        es aplicada a todos los elementos que componen al arreglo
        de dos dimensiones, por lo que se vuelve muy dif\'icil de
        rastrear y obtener siempre alg\'un mismo resultado, puesto
        que en cada ejecuci\'on del algoritmo, la configuraci\'on y
        resultados obtenidos ser\'an distintos, por lo que lo \'unico que 
        se debe de revisar es que el resultado sea en su mayor\'ia
        satisfactorio, que a pesar de que no siempre la ruta sea la
        m\'as \'optima, se acerque a hacerlo, para as\'i generar una mejor
        movilidad y toma de decisiones y tratamiento de la
        informaci\'on.
    \vskip 0.5cm
    La interoperabilidad entre sistemas rob\'oticos se beneficia 
        del uso de normas como la IEEE 1872.1, que no s\'olo define 
        t\'erminos y conceptos, sino tambi\'en los contextos necesarios para 
        asegurar un intercambio efectivo de datos. Esto implica establecer 
        conexiones claras entre los sensores, el software de control, y el
        hardware rob\'otico, lo que facilita una sincronizaci\'on continua y un flujo de datos eficiente.
    \vskip 0.5cm
    En el marco de la evaluaci\'on de datos del rendimiento, es crucial contar con mecanismos que validen la 
        integridad y calidad de la informaci\'on obtenida por el robot. Se hace especial \'enfasis en la 
        consistencia de los datos recogidos, el manejo de errores y la representaci\'on fidedigna del 
        comportamiento del robot en escenarios controlados y reales, donde las normas IEEE ayudan a uniformar estos criterios.
    \vskip 0.5cm
    Un aspecto relevante de la aplicaci\'on de la norma IEEE 2914 radica en la gesti\'on del 
        comportamiento estoc\'astico del algoritmo Physarum. Debido a la variabilidad en las rutas generadas, 
        es necesario implementar mecanismos de observaci\'on y registro que permitan un an\'alisis comparativo 
        entre ejecuciones, estableciendo patrones y tendencias que puedan mejorarse en iteraciones futuras.
    \vskip 0.5cm
    La generaci\'on de rutas mediante el algoritmo de Physarum implica una interacci\'on constante entre elementos 
        aleatorios y deterministas, los cuales deben evaluarse bajo criterios de eficacia y seguridad. Las normas IEEE 
        proporcionan las pautas para medir y asegurar que, aunque los resultados puedan variar, las soluciones propuestas 
        mantengan un grado aceptable de estabilidad y funcionalidad.
    \vskip 0.5cm
    La integraci\'on de herramientas de comunicaci\'on, como servidores FTP y protocolos HTTP, dentro del entorno 
        rob\'otico asegura un manejo eficiente de los datos. Este proceso debe regirse por est\'andares que garanticen 
        la seguridad, la velocidad y la precis\'on del intercambio de informaci\'on, tanto en ambientes controlados como 
        en situaciones en las que el robot interact\'ue con diferentes dispositivos.
    \vskip 0.5cm
    Dado que el software que rige el comportamiento del robot act\'ua como un intermediario cr\'itico entre los sensores 
        y el hardware, su correcto funcionamiento es fundamental. La norma IEEE 1872.1 se encarga de establecer 
        un conjunto de reglas para garantizar que esta comunicaci\'on sea efectiva, asegurando que todos los componentes 
        involucrados operen de manera conjunta y coherente.
    \vskip 0.5cm
    El \'exito en la generaci\'on de rutas \'optimas o satisfactorias mediante el modelo de Physarum depende, en gran medida, 
        de la capacidad del sistema para adaptarse a condiciones cambiantes. Este enfoque flexible se encuentra respaldado 
        por la norma IEEE 2914, que permite explorar diversas configuraciones y evaluar su eficacia en tiempo real, 
        promoviendo mejoras continuas y soluciones creativas dentro del contexto rob\'otico.