\subsection{Recopilaci\'on y an\'alisis de datos del rendimiento del robot de IEEE 1872.1 e IEEE 2914}
    En el mundo de las pruebas y otro tipo de documentaci\'on,
        son importantes las normas, debido a que permiten un
        est\'andar para el manejo de distintos elementos dentro de un
        sistema, adem\'as de que la estandarizaci\'on fomenta el
        intercambio sencillo o interacci\'on entre componentes los cuales no tienen mucho que ver con otros, pero que al existir
        una norma, un modo de estandarizaci\'on, se pueden crear
        formas en las que ambas partes puedan coexistir entre s\'i y
        entre muchas otras m\'as.
    \vskip 0.5cm
    La importancia de la IEEE 1872.1 radica en que esta norma
        nos permite definir ciertos elementos y contextos en los
        cuales se puede realizar un intercambio de informaci\'on entre
        pares, los cuales usualmente est\'an relacionados a la
        electr\'onica y rob\'otica, as\'i como al hardware y software.
        En el caso del software y del robot al cual el software le
        genera una ruta para su posterior seguimiento, usa criterios
        de interoperabilidad que son los m\'as comunes a la hora de
        poder conectar los resultados obtenidos por el programa, as\'i
        como la forma de interpretar esos datos para cumplir el
        objetivo principal del algoritmo de Physarum: generar una
        ruta la cual el robot pueda seguir sin problemas para llegar a
        un destino final.
    \vskip 0.5cm
    Para el caso de el envio de informaci\'on entre
        el robot y el software, se hace uso de herramientas FTP, as\'i
        como de un servidor el cual recibe peticiones HTTP con las
        cuales se pueden recibir y enviar los datos que sean
        necesarios.
    \vskip 0.5cm
    A pesar de que el software pertenezca a una de las jerarqu\'ias
        m\'as bajas dentro de la estructura de la IEEE, es de vital
        importancia que \'este ultimo funcione correctamente y
        cumpla su funci\'on para que el robot tenga un prop\'osito y
        pueda tener la utilidad necesaria.
    \vskip 0.5cm
    Y dentro de las dem\'as normas, tambi\'en se toma en cuenta la
        norma IEEE 2914, que dentro de la parte generadora de la
        ruta y dem\'as trabajo del software de ruteo a partir del
        Physarum, se vuelve importante tomarla en cuenta. Esto
        porque a diferencia de otros sistemas donde la toma de
        decisiones trata de ser siempre la \'optima y en muchos casos
        est\'an se pueden llegar a replicar para tener un registro y una
        idea de que decisiones te llevan a un determinado resultado,
        en el caso del simulador del Physarum las decisiones son
        aleatorias en todo momento, adem\'as, de que la aleatoriedad
        es aplicada a todos los elementos que componen al arreglo
        de dos dimensiones, por lo que se vuelve muy dif\'icil de
        rastrear y obtener siempre alg\'un mismo resultado, puesto
        que en cada ejecuci\'on del algoritmo, la configuraci\'on y
        resultados obtenidos ser\'an distintos, por lo que lo \'unico que 
        se debe de revisar es que el resultado sea en su mayor\'ia
        satisfactorio, que a pesar de que no siempre la ruta sea la
        m\'as \'optima, se acerque a hacerlo, para as\'i generar una mejor
        movilidad y toma de decisiones y tratamiento de la
        informaci\'on.
    \vskip 0.5cm