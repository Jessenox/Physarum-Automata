\subsection{Informe final con resultados de la Iteraci\'on 3}
    El informe final de la Iteraci\'on 3 presenta una evaluaci\'on
        detallada del rendimiento del software, con base en los datos
        obtenidos de las pruebas y simulaciones realizadas. Se
        documentaron las rutas generadas en escenarios complejos,
        los tiempos de simulaci\'on y las mejoras implementadas para
        optimizar el desempe\~no en entornos m\'as exigentes.
    \vskip 0.5cm
    Entre los principales resultados, se destaca la capacidad del
        algoritmo para adaptarse a mapas de mayor tama\~no y con
        m\'ultiples obst\'aculos, aunque con un incremento en el tiempo
        de procesamiento. Tambi\'en se destaca la mejora en la
        limpieza de las rutas generadas, lo que permiti\'o que el robot
        siguiera las coordenadas con mayor precisi\'on.
    \vskip 0.5cm
    Este informe concluye con recomendaciones para la
        siguiente iteraci\'on, enfocadas en optimizar el tiempo de
        ejecuci\'on del software en escenarios m\'as grandes y
        complejos, as\'i como en continuar mejorando la integraci\'on
        con los sistemas de control del robot para asegurar un
        seguimiento m\'as preciso de las rutas generadas.
    \vskip 0.5cm
    Algunos de los datos recopilados tienen relaci\'on con las pruebas realizadas
        directamente sobre casos de la realidad, donde algunos de los datos que recopilamos,
        como el comportamiento del robot en un entorno con mucha gente, as\'i como dejando
        que llegue a una distancia mucho m\'as larga de lo habitual, nos representaron datos de
        suma importancia para su an\'alisis y reproducci\'on de posibles nuevas pruebas futuras
        para el software y sus subsecuentes aplicaciones.
    \vskip 0.5cm
    Los resultados obtenidos tambi\'en comprenden a los casos aplicados en la vida real,
        los cuales representan un gran paso para el uso y posible aplicaci\'on del robot en
        entornos los cuales representen un reto para otro tipo de veh\'iculos o incluso, otro tipo
        de robots los cuales encuentren cierta dificultad a la hora de realizar las mismas tareas
        que nuestro robot.
    \vskip 0.5cm
    Surgen distintas y varias aplicaciones en las cuales el robot es capaz de actuar en
        distintos tipos de escenarios, cuyos retos a la hora de navegaci\'on sean variados y
        donde la posibilidad de obtener rutas que sean complicadas, y que con el uso de el
        simulador del Physarum, podemos solventar varios de estos retos, como lo son la
        generaci\'on de rutas de forma aleatoria, pero que siempre se acercan a una ruta \'optima,
        la constante aparici\'on de obst\'aculos y adem\'as la generaci\'on de una nueva ruta a partir
        de que se sea encontrado un obst\'aculo el cual originalmente no se encontraba la
        primera vez que fue generada la ruta.
    \vskip 0.5cm
    Por \'ultimo, los resultados nos funcionan como una bit\'acora para poder analizar el
        comportamiento y las caracter\'isticas que fueron cambiando durante el desarrollo e
        implementaci\'on del robot, as\'i como una forma de ver qu\'e cambios tuvieron una gran
        relevancia a la hora de implementarlos y sobre qu\'e otro tipo de cambios son posibles
        y qu\'e se puede hacer para poder seguir mejorando tanto el algoritmo, as\'i como la
        implementaci\'on en el robot