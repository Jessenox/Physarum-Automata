\subsection{Documentaci\'on del avance segunda iteraci\'on}
    Durante la segunda iteraci\'on, se realizaron mejoras
        significativas en la funcionalidad del simulador y sucapacidad para generar rutas m\'as \'optimas y adaptarse a
        entornos m\'as complejos. Se implementaron ajustes en el
        algoritmo basado en el Physarum Polycephalum para reducir
        los tiempos de simulaci\'on y mejorar la precisi\'on en la
        selecci\'on de rutas. Adem\'as, se realizaron pruebas
        exhaustivas en distintos sistemas operativos (Windows y
        Linux), documentando las diferencias de rendimiento y
        comportamiento del software en ambos entornos.
    \vskip 0.5cm
    En esta iteraci\'on, tambi\'en se introdujeron mejoras en la
        interfaz de control del robot, facilitando el seguimiento de
        rutas generadas y la interacci\'on con los sensores del robot,
        como el sistema LiDAR. La documentaci\'on incluye los
        cambios implementados, los resultados obtenidos y las
        estrategias propuestas para las pruebas de aceptaci\'on en
        entornos complejos, que servir\'an de base para la pr\'oxima
        iteraci\'on.
    \vskip 0.5cm
    Durante la segunda iteraci\'on, se realizaron mejoras
        significativas en la funcionalidad del simulador y su
        capacidad para generar rutas m\'as \'optimas y adaptarse a
        entornos m\'as complejos. Se implementaron ajustes en el
        algoritmo basado en el Physarum Polycephalum para reducir
        los tiempos de simulaci\'on y mejorar la precisi\'on en la
        selecci\'on de rutas. Adem\'as, se realizaron pruebas
        exhaustivas en distintos sistemas operativos (Windows y
        Linux), documentando las diferencias de rendimiento y
        comportamiento del software en ambos entornos.
    \vskip 0.5cm
    En esta iteraci\'on, tambi\'en se introdujeron mejoras en la
        interfaz de control del robot, facilitando el seguimiento de
        rutas generadas y la interacci\'on con los sensores del robot,
        como el sistema LiDAR. La documentaci\'on incluye los
        cambios implementados, los resultados obtenidos y las
        estrategias propuestas para las pruebas de aceptaci\'on en
        entornos complejos, que servir\'an de base para la pr\'oxima
        iteraci\'on.