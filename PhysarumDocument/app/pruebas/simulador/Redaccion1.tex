\subsection{Redacci\'on del informe t\'ecnico inicial basado en la Iteraci\'on 1}
    Con la informaci\'on que hasta el momento se ha recabado, se
        comienza con la redacci\'on del reporte t\'ecnico, en la cual
        quedan asentados varios aspectos que fueron siendo
        encontrados durante las pruebas de funcionamiento, as\'i
        como los resultados arrojados por las pruebas unitarias
        aplicadas al programa.
        \vskip 0.5cm
    Adem\'as, se tiene en consideraci\'on la recopilaci\'on de
        informaci\'on obtenida a partir de la evaluaci\'on del
        desempe\~no con algunas de las pruebas iniciales de
        funcionamiento aplicadas en el programa.
        \vskip 0.5cm
    A dem\'as en esta iteraci\'on se pudo comprobar el funcionamiento en
        general del Physarum, obteniendo as\'i el funcionamiento
        principal de este, el cual es obtener una ruta a partir de un
        punto inicial y de un punto final. Es importante ver que
        debido a que es la primera iteraci\'on, han surgido muchas
        observaciones las cuales son de vital importancia para el
        an\'alisis y mejora del programa, por lo que eso significa que
        mientras m\'as se avance, el programa estar\'a en un proceso de
        mejora continua.
    \vskip 0.5cm
    Principalmente lo que se pudo notar es el funcionamiento en
        el caso base, que es el de generar la ruta a trav\'es de distintas
        configuraciones del estado inicial, siendo obtenidas rutas las
        cuales, son las que el robot interpretar\'ia para poder realizar
        su funcionalidad de moverse de un lugar a otro.
    \vskip 0.5cm
    En los distintos escenarios y pruebas que fueron realizadas
        en este proceso, se obtuvieron distintos resultados los cuales
        nos pueden dar una idea sobre el rendimiento y ventanas de
        oportunidad para la mejora del algoritmo, esto es debido aque se plantearon escenarios y pruebas cuyos resultados son
        reveladores y que ayudar\'ian a obtener mejores resultados
        para el objetivo principal del programa, el cual es generar
        rutas.
    \vskip 0.5cm