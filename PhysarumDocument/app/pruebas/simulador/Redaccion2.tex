\subsection{Redacci\'on del informe con mejoras de la Iteraci\'on 2}
    En esta etapa se realiz\'o un an\'alisis exhaustivo de los
        resultados obtenidos en la Iteraci\'on 2, enfoc\'andonos en la
        identificaci\'on de \'areas clave de mejora en el software. A
        partir de las pruebas realizadas y los datos recopilados, se
        implementaron ajustes en el algoritmo para optimizar su
        rendimiento en entornos m\'as complejos. Estos ajustes se
        centraron en mejorar la eficiencia en la generaci\'on de rutas,
        reduciendo los tiempos de procesamiento en mapas de
        mayor tama\~no y con m\'as obst\'aculos.
        \vskip 0.5cm
    Adem\'as, se revisaron y actualizaron las pruebas unitarias
        para asegurar la correcta implementaci\'on de las mejoras,
        enfoc\'andonos en la estabilidad del sistema al enfrentarse a
        diferentes configuraciones iniciales. Los cambios
        implementados lograron una reducci\'on significativa en los
        tiempos de simulaci\'on, especialmente en los casos donde se
        inclu\'ian m\'ultiples barreras o rutas m\'as largas.
        \vskip 0.5cm
    Asimismo, se refinaron las estrategias para eliminar c\'elulas
        innecesarias en las rutas generadas, lo que permiti\'o un
        recorrido m\'as eficiente por parte del robot, optimizando su
        capacidad de seguir las coordenadas calculadas por el
        simulador. Estos avances fueron documentados y evaluados,
        dejando asentadas las mejoras que ser\'an fundamentales para
        las siguientes iteraciones.
        \vskip 0.5cm
    El informe incluye un an\'alisis detallado de los tiempos de
        ejecuci\'on, las rutas generadas en diversas configuraciones y
        las m\'etricas de desempe\~no obtenidas en sistemas operativos
        tanto Windows como Linux. Estos resultados muestran un
        progreso claro en el desarrollo del software, destacando las
        mejoras implementadas para su robustez y precisi\'on en
        escenarios m\'as exigentes.
    \vskip 0.5cm