\subsection{Recopilaci\'on final de datos del software para el informe 1}
    Los datos hasta ahora recopilados nos muestran un panorama
        mucho m\'as amplio del que se ten\'ia al inicio de la realizaci\'on
        y desarrollo del software, esto es, se han recopilado
        bastantes datos los cuales ayudan a que se pueda encontrar
        una mucho mejor forma de realizar algunas de las
        funcionalidades o se hayan encontrado algunas carencias de
        las cuales muchas veces pasan desapercibidas en una
        primera instancia.
        \vskip 0.5cm
    El programa, como vimos al inicio, funciona correctamente
        y cumple su funci\'on de generar rutas, no obstante, las rutas
        muchas veces tienden a ser un poco imperfectas debido a la
        propia naturaleza del algoritmo en s\'i, adem\'as de que muchas
        veces se generan bifurcaciones o hay demasiadas c\'elulas en
        determinadas partes de la ruta, lo que fomenta que se
        encuentre una mayor dificultad a la hora de generar una ruta
        \'optima para ser enviada al robot, adem\'as de que el proceso
        de generaci\'on de coordenadas se complica debido a que cada
        una de estas debe de tener alg\'un tipo de orden el cual ayude
        a que el robot tenga una forma mucho m\'as f\'acil de moverse
        en su entorno real y que lo haga sin tener que dar muchas
        vueltas.
        \vskip 0.5cm
    Por lo que se empezaron a generar soluciones, como lo es
        limpiar y eliminar algunas de las c\'elulas dentro del arreglo
        con el objetivo de optimizar la ruta, adem\'as de generar
        algoritmos los cuales puedan tener una mejor aproximaci\'onal correcto ordenamiento de las c\'elulas presentes y que al
        final, la ruta sea una la cual el robot pueda seguir sin
        problemas y que, respecto a la generada inicialmente por el
        simulador del Physarum, sea mucho m\'as conveniente.
        Por \'ultimo, se pudo notar que mientras m\'as incrementaba el
        tama\~no del lienzo, el programa tiende a alentarse un poco
        m\'as, esto es debido a la gran cantidad de informaci\'on que
        tiene que desplegar al mismo tiempo y dificulta el correcto
        manejo del software, por lo que en las futuras
        implementaciones e iteraciones se tendr\'a que mejorar este
        aspecto para as\'i poder optimizar los recursos que son usados
        y tener una mejor experiencia a la hora de usar el simulador
        y as\'i, tener un mejor confort a la hora de generar una ruta.
        \vskip 0.5cm