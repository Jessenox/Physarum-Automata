\subsection{Pruebas de aceptaci\'on en entornos complejos simulados 1}
    Estas pruebas comprenden a las anteriores pruebas, ya que al
        igual que las pruebas de aceptaci\'on de la iteraci\'on anterior,
        para el entorno m\'as complejo, el tama\~no de lienzo es lo
        \'unico que cambia, pero lo \'unico que se agrega ahora es la
        carga de un mapa de un entorno real.
    \vskip 0.5cm
    \subsubsection{1ra Prueba de aceptaci\'on: Modificaci\'on del tama\~no del lienzo}
        Descripci\'on: En el c\'odigo se colocan las dimensiones del
            lienzo y el arreglo de dimensi\'on n x n correspondiente al
            aut\'omata celular, por lo que se quiere comprobar si la
            modificaci\'on de estos valores cambia debidamente el
            tama\~no del lienzo mostrado en pantalla.
        \vskip 0.5cm
        Flujo:
        \begin{itemize}
            \item Se cambian los valores del lienzo en el c\'odigo.
            \item Se inicia el programa.
            \item El programa muestra los cambios correspondientes a los
            valores que fueron colocados en el c\'odigo al
            desplegarse el lienzo.
        \end{itemize}
        \vskip 0.5cm
        Criterios de aceptaci\'on:
        \begin{itemize}
            \item El programa inicia correctamente.
            \item El tama\~no del lienzo es acorde a los valores colocados
            en el c\'odigo.
            \item Se pueden colocar estados en cualquier \'area
            correspondiente al lienzo.
        \end{itemize}
        \vskip 0.5cm
        El software cumple con cada uno de los criterios de
            aceptaci\'on, adem\'as de seguir el flujo con el que se hab\'ia
            planeado desde el inicio, as\'i que se considera a esta prueba
            como finalizada y con buenos resultados obtenidos.
    \vskip 0.5cm
    \subsubsection{2da Prueba de aceptaci\'on: Carga de mapas}
        Descripci\'on: En el c\'odigo se cargan im\'agenes, las cuales
            corresponden a mapas, donde \'estas se cargan para colocarlos
            en el lienzo al iniciar el programa
        \vskip 0.5cm
        Flujo:
        \begin{itemize}
            \item Se coloca la direcci\'on de la imagen
            \item Se coloca un tama\~no para tanto la imagen como para el
            lienzo
            \item Se inicia el programa.
            \item El programa al iniciar, carga la configuraci\'on,
            mostrando la imagen del mapa, convertido al lienzo.
        \end{itemize}
        \vskip 0.5cm
        Criterios de aceptaci\'on:
        \begin{itemize}
            \item El programa inicia correctamente.
            \item El tama\~no del lienzo es acorde a los valores colocados
            en el c\'odigo.
            \item Se pueden colocar estados en cualquier \'area
            correspondiente al lienzo.
            \item La imagen del mapa es colocada correctamente en el
            lienzo.
        \end{itemize}
        \vskip 0.5cm
        El software cumple con cada uno de los criterios de
            aceptaci\'on, adem\'as de seguir el flujo con el que se hab\'ia
            planeado desde el inicio, as\'i que se considera a esta prueba
            como finalizada y con buenos resultados obtenidos.
    \vskip 0.5cm
    \subsubsection{3ra Prueba de aceptaci\'on: Elecci\'on de estados a trav\'es del teclado.}
        Descripci\'on: El usuario, al iniciar el programa, puede elegir por medio de las teclas el estado actual que quiera colocar en
            el lienzo desplegado.
        Flujo:
        \begin{itemize}
            \item El programa es cargado.
            \item Se le presenta el lienzo en pantalla.
            \item Al elegir los estados con las teclas num\'ericas, estas son
            reflejadas en la pantalla.
        \end{itemize}
        \vskip 0.5cm
        Criterios de aceptaci\'on:
        \begin{itemize}
            \item Al presionar una tecla num\'erica, cambia al respectivo
            estado que representa.
            \item El estado elegido es plasmado en el lienzo al dar clic.
            \item Puede cambiar en cualquier momento el estado de su
            elecci\'on
        \end{itemize}
        \vskip 0.5cm
        Para la prueba anterior, los resultados fueron satisfactorios
            puesto que cumpli\'o con todos los criterios de aceptaci\'on,
            siguiendo el flujo que se describ\'ia en la prueba.
    \vskip 0.5cm
    \subsubsection{4ta Prueba de aceptaci\'on: Colocaci\'on de los estados inicial y final.}
    Descripci\'on: Al presionar con el bot\'on izquierdo del mouse
        en el lienzo que se es desplegado, se colocan en pantalla el
        estado correspondiente al seleccionado previamente con el
        teclado. Particularmente se eval\'ua que se coloque el estado
        inicial y el estado que corresponde al nutriente no
        encontrado, siendo los principales componentes en el estado
        inicial que es necesario para iniciar la simulaci\'on.
    \vskip 0.5cm
    Flujo:
    \begin{itemize}
        \item Se despliega el lienzo en pantalla
        \item Se elige por medio del teclado el estado deseado.
        \item Al dar clic en pantalla, este estado es puesto en pantalla
    \end{itemize}
    Criterios de aceptaci\'on:
    \begin{itemize}
        \item El estado es colocado en el lienzo correctamente.
        \item La pantalla muestra el estado actual y el color
        correspondiente al teclado.
        \item La colocaci\'on de los estados solo puede ser colocada
        dentro del \'area del lienzo y no por fuera.
    \end{itemize}
    \vskip 0.5cm
    En la segunda prueba, se lograron cumplir los criterios de
        aceptaci\'on de manera satisfactoria, por lo que la prueba se
        considera como realizada y cumplida correctamente.
    \vskip 0.5cm
    \subsubsection{5ta Prueba de aceptaci\'on: Inicio de la simulaci\'on}
    Al presionar el bot\'on de ENTER en el teclado, la simulaci\'on del organismo Physarum es iniciada, y si fueron colocados
        correctamente los estados correspondientes al estado inicial
        y al nutriente no encontrado, entonces se genera una ruta
        entre cada estado.
    \vskip 0.5cm
    Flujo:
    \begin{itemize}
        \item El usuario coloca los estados deseados en pantalla.
        \item El usuario presiona la tecla ENTER.
        \item El programa inicia con la simulaci\'on y si son colocados
        los estados necesarios, se crea una ruta entre estos.
    \end{itemize}
    \vskip 0.5cm
    Criterios de aceptaci\'on:
    \begin{itemize}
        \item El n\'umero de generaciones aumenta.
        \item Si es colocado el estado 3, entonces el Physarum inicia
        su expansi\'on.
        \item Si son colocados el estado 3 y 1, se genera una ruta entre
        estos una vez finalizada la simulaci\'on.
    \end{itemize}
    \vskip 0.5cm
    La prueba anterior fue realizada y cumpli\'o con los criterios
        de aceptaci\'on que fueron solicitados, por lo que esta prueba
        se da por finalizada y con un resultado positivo.