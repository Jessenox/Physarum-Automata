\subsection{Pruebas de aceptaci\'on en entornos complejos simulados 2}
    Para validar el rendimiento del software en escenarios m\'as
        desafiantes, se llevaron a cabo pruebas de aceptaci\'on en
        entornos complejos simulados. En estas pruebas, se
        introdujeron obst\'aculos adicionales y laberintos m\'as
        elaborados en el lienzo de simulaci\'on, lo que permiti\'o
        observar el comportamiento del algoritmo ante situaciones
        de alta densidad de barreras.
    \vskip 0.5cm
    Los resultados mostraron que el algoritmo es capaz de
        adaptarse a estos entornos, generando rutas eficientes incluso
        con una mayor presencia de obst\'aculos. Sin embargo, se
        identificaron \'areas donde el tiempo de c\'alculo aumenta
        significativamente en mapas de gran tama\~no, lo que sugiere
        la necesidad de optimizar ciertos componentes del algoritmo
        en futuras iteraciones.
    \vskip 0.5cm
    Estas pruebas fueron realizadas en entornos simulados de
        tama\~no m\'as grande, y los resultados revelaron que las rutas
        generadas contin\'uan siendo precisas, aunque con un mayor
        tiempo de procesamiento, especialmente en sistemas con
        recursos limitados.
    \vskip 0.5cm