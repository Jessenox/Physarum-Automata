\section{Introducci\'on}
\label{sec:introduccion}
    % P\'arrafo 1
    En el complejo mundo actual, en el cual los procesos industriales se valen de la automatizaci\'on y digitalizaci\'on, 
        los aut\'omatas celulares emergen como sistemas din\'amicos discretos de gran potencial. Dichos sistemas est\'an 
        constituidos por matrices de celdas; estas celdas son la unidad b\'asica de los aut\'omatas celulares y cada celda 
        puede estar en un estado determinado, como 1 o 0, vivo o muerto. En realidad, pueden tener m\'ultiples 
        interpretaciones, pero b\'asicamente es un sistema binario. Adem\'as, estas matrices de celdas se rigen 
        por las reglas del aut\'omata, que evolucionan conforme lo hacen las generaciones. En este proyecto nos 
        enfocaremos en el desarrollo y aplicaci\'on de un aut\'omata programable en una Raspberry Pi, con el objetivo 
        de proporcionar a las organizaciones una soluci\'on integral y adaptable. Esta soluci\'on les permitir\'a el 
        monitoreo y control eficiente de sistemas y procesos en tiempo real.
    % P\'arrafo 2
    \vskip 0.5cm
    La necesidad de un monitoreo en tiempo real es muy alta en el entorno social, empresarial y en algunas 
        aplicaciones para uso gubernamental, donde la agilidad y la eficacia son esenciales. En esta situaci\'on, 
        los aut\'omatas celulares se presentan como herramientas bastante vers\'atiles a la hora de interactuar 
        din\'amicamente con su entorno. En t\'erminos simples, cada uno de los elementos en un aut\'omata celular se 
        relaciona con sus vecinos (celdas que se encuentran alrededor de nuestra celda actual), y su estado en 
        la pr\'oxima generaci\'on se determina seg\'un el estado de sus vecinos en la generaci\'on actual. Esta capacidad 
        \'unica de interacci\'on y cambio din\'amico de estados los convierte en instrumentos poderosos para abordar una
        variedad de desaf\'ios.
    % P\'arrafo 3
    \vskip 0.5cm
    Este proyecto no solo se enfoca en la implementaci\'on t\'ecnica de un aut\'omata celular, sino tambi\'en 
        en su aplicaci\'on pr\'actica en entornos espec\'ificos. Los aut\'omatas celulares han demostrado su 
        val\'ia en diversas \'areas, desde el monitoreo y predicci\'on del cambio de uso de la tierra hasta 
        la planificaci\'on de rutas sin colisi\'on para robots. Dos referencias particulares, \cite{Tzionas1997} 
        y \cite{Lopes2023}, destacan por su relevancia directa a nuestra propuesta de Trabajo Terminal (TT), 
        ya que se centran en tareas de monitoreo y la implementaci\'on de robots sin colisiones, respectivamente. 
        Justamente en nuestro caso, es el monitoreo y la prevenci\'on de colisiones de robots.
    % P\'arrafo 4
    \vskip 0.5cm
    En este enfoque buscamos resaltar la versatilidad de los aut\'omatas celulares como herramientas de soluci\'on aplicables 
        en situaciones del mundo real. A su vez, se espera que el aut\'omata sea adaptable para que pueda ser implementado 
        en sectores emergentes, como la Inteligencia Artificial (IA), ampliando a\'un m\'as su alcance y utilidad.
    % Objetivo
    \subsection{Objetivo}
\label{sec-1-1-1}
    \subsubsection{Objetivo general}
    \label{sec-1-1-1-1}
        Implementar un aut\'omata que sea capaz de determinar sus trayectos en espacios bidimensionales para
            monitorear trazando rutas en tiempo real.
    \subsubsection{Objetivos espec\'ificos}
    \label{sec-1-1-1-2}
        \begin{itemize}
            \setlength\itemsep{-0.5em}
            \item Dise\~nar un aut\'omata basado en el mixomiceto \textit{Physarum polycephalum} que sea capaz de determinar
                trayectos en espacios bidimensionales.
            \item Implementar una simulaci\'on del aut\'omata en un programa desarrollado en el lenguaje de programaci\'on
                C++.
            \item Implementar el aut\'omata en robot cuyo controlador sea una Raspberry Pi 4.
            \item Dise\~nar un sistema de monitoreo que permita visualizar el estado del aut\'omata y del robot.
            \item Realizar las pruebas en un entorno controlado.
            \item Realizar las pruebas en un entorno real.
        \end{itemize}
    % Justificaci\'on
    \subsection{Justificaci\'on}
\label{sec-1-1-2}

% Parrafo 1
    En el marco actual de la automatizaci\'on constante de procesos y el desarrollo de herramientas que facilitan la
        realizaci\'on de tareas, han surgido distintas tecnolog\'ias y procesos que han sabido atacar de la mejor manera las
        problem\'aticas que involucran a la automatizaci\'on. Sin embargo, muchas veces es complicado darles un correcto
        seguimiento a las actividades y procesos que est\'an involucrados en la realizaci\'on de tareas autom\'aticas,
        ocasionando distintos problemas que afectan en la soluci\'on del problema al que originalmente planteaban
        solucionar o complicar de manera innecesaria el proceso. Por lo que es necesario crear herramientas de
        monitorizaci\'on con m\'etodos m\'as fiables a los ya existentes, y que, adem\'as de brindar un correcto seguimiento
        a los procesos a los cuales se les hace un an\'alisis, tenga la capacidad de reaccionar ante los cambios relevantes
        e importantes que surjan durante la realizaci\'on de las distintas tareas y procesos en los que se vea involucrado.
\vskip 0.5cm
% Parrafo 2
    Es por eso por lo que se busca la implementaci\'on de un sistema rob\'otico, el cual, por medio de la aplicaci\'on de
        la teor\'ia de aut\'omatas celulares, monitorice la realizaci\'on de distintas tareas y procesos en los que se vea
        involucrado y, adem\'as, priorizando siempre que la monitorizaci\'on sea efectiva, continua y confiable. Esto para
        las distintas industrias que realicen actividades en las cuales se vean involucradas la automatizaci\'on de procesos
        y tareas.
\vskip 0.5cm
% Parrafo 3
    Los aut\'omatas celulares han sido usados para distintas disciplinas que van desde la antropolog\'ia hasta los
        gr\'aficos por computadora, sin embargo, ha sido muy poco visto en actividades que involucren sistemas
        rob\'oticos debido al comportamiento y la manera en la que los aut\'omatas celulares se comportan a partir de
        diversas entradas, siendo a veces complicado discernir el comportamiento que se tendr\'a, lo que a\~nade
        complejidad al implementar uno de estos aut\'omatas a sistemas rob\'oticos. Pero gracias al conocimiento brindado
        por algunas materias como lo son Sistemas Operativos, Arquitectura de Computadoras, Dise\~no Digital y Teor\'ia
        Computacional, se puede llegar a un procedimiento y tratamiento de la informaci\'on tal que sea posible dirigir
        el aut\'omata a la mejor soluci\'on posible.
    % Alcances
    %\input{./app/intro/Alcances}
    % Limitaciones
    %\input{./app/intro/Limitaciones}
    % Estructura
    %\input{./app/intro/Estructura}
\clearpage