\subsubsection{Uso de veh\'iculos a\'ereos  no tripulados (VANT's) para el monitoreo y manejo de los recursos naturales: una s\'intesis} % (fold)
\label{ssub:subsubsection name}
    El art\'iculo \textit{'Uso de veh\'iculos a\'ereos no tripulados (VANT's) para el monitoreo y manejo de los recursos naturales: una s\'intesis'} 
        proporciona una base robusta para contextualizar la tesis titulada \textit{"Dise\~no de un aut\'omata para monitoreo"} en 
        el estado del arte. La revisi\'on exhaustiva que presenta este art\'iculo sobre la utilizaci\'on de drones en diversas aplicaciones 
        de monitoreo y manejo de recursos naturales en Am\'erica Latina es directamente relevante para esta investigaci\'on, 
        ya que ambos trabajos se enfocan en el desarrollo y aplicaci\'on de tecnolog\'ias aut\'onomas para la recolecci\'on 
        de datos ambientales \cite{Guevara2020}.
    \vskip 0.5cm
    En primer lugar, el art\'iculo destaca c\'omo los drones, equipados con una variedad de sensores como RGB, infrarrojos, 
        multiespectrales, hiperespectrales y LIDAR, han revolucionado la capacidad de los investigadores para monitorear 
        con precisi\'on diversos aspectos del medio ambiente. Estos avances permiten obtener datos de alta resoluci\'on espacial 
        de manera r\'apida y a un costo reducido, lo cual es crucial para la efectividad y eficiencia del monitoreo ambiental. 
        En el contexto de la tesis, el dise\~no de un aut\'omata para monitoreo puede beneficiarse enormemente de estos conocimientos 
        y tecnolog\'ias, aplicando principios similares de autonom\'ia y precisi\'on en la recolecci\'on de datos.
    \vskip 0.5cm
    Adem\'as, el art\'iculo proporciona una visi\'on detallada de las ventajas y limitaciones de diferentes tipos de drones y sensores, 
        ofreciendo informaci\'on valiosa que puede influir en las decisiones de dise\~no y selecci\'on de componentes para el aut\'omata. 
        Por ejemplo, la comprensi\'on de las capacidades y restricciones de los sensores hiperespectrales y LIDAR puede guiar 
        la integraci\'on de tecnolog\'ias adecuadas en el sistema de monitoreo, asegurando una recopilaci\'on de datos eficiente y 
        precisa. Esta informaci\'on es crucial para el dise\~no de sistemas que requieran alta resoluci\'on y precisi\'on en la medici\'on 
        de variables ambientales.
% subsubsection subsubsection name (end)