\subsubsection{Mobile robot's sampling algorithms for monitoring of insects' populations in agricultural fields} % (fold)
\label{ssub:subsubsection name2}
    El art\'iculo \textit{'Mobile robot's sampling algorithms for monitoring of insects' populations in agricultural fields'} 
        proporciona una base s\'olida para contextualizar la tesis titulada \textit{'Dise\~no de un aut\'omata para monitoreo'} 
        en el estado del arte. La investigaci\'on presentada en este documento aborda el desarrollo y evaluaci\'on de varios 
        algoritmos de muestreo para robots m\'oviles, destinados a la detecci\'on de insectos en campos agr\'icolas, 
        una problem\'atica directamente relevante para la tesis, que se centra en el dise\~no de un aut\'omata para el monitoreo 
        ambiental. \cite{Yehoshua2023}
    \vskip 0.5cm
    En primer lugar, el art\'iculo destaca la importancia de los algoritmos de muestreo para maximizar la eficiencia en la 
        detecci\'on de plagas, considerando las limitaciones de recursos como el tiempo y la energ\'ia. Los algoritmos 
        desarrollados, tanto aquellos que operan sin informaci\'on previa como los que utilizan datos en tiempo real, 
        ofrecen estrategias para optimizar la recolecci\'on de datos en entornos agr\'icolas. En el contexto de la tesis, 
        estos conocimientos pueden ser aplicados al dise\~no del aut\'omata, integrando algoritmos de muestreo din\'amico 
        que prioricen puntos de muestreo estrat\'egicos basados en patrones de distribuci\'on de plagas, mejorando as\'i 
        la eficiencia y precisi\'on del monitoreo. \cite{Yehoshua2023}
    \vskip 0.5cm
    Adem\'as, el art\'iculo proporciona una evaluaci\'on detallada de la efectividad de estos algoritmos en diferentes 
        escenarios de simulaci\'on, considerando variables como el tama\~no del campo y la tasa de propagaci\'on de insectos. 
        Esta informaci\'on es crucial para la tesis, ya que ofrece una comprensi\'on profunda de c\'omo los diferentes 
        algoritmos pueden ser implementados y ajustados seg\'un las condiciones espec\'ificas del entorno de monitoreo. 
        La integraci\'on de estas estrategias en el dise\~no del aut\'omata permitir\'a una adaptaci\'on m\'as r\'apida y precisa 
        a las condiciones cambiantes del campo, asegurando una detecci\'on temprana y gesti\'on eficaz de plagas.
    \vskip 0.5cm
    El art\'iculo tambi\'en incluye estudios de caso basados en datos reales de infestaciones de insectos, 
        proporcionando ejemplos pr\'acticos y lecciones aprendidas que pueden ser directamente aplicables 
        a la tesis. Estos estudios demuestran c\'omo la implementaci\'on de algoritmos de muestreo din\'amico
        ha llevado a mejoras significativas en la eficiencia y precisi\'on del monitoreo, validando la 
        relevancia y aplicabilidad del dise\~no del aut\'omata en contextos agr\'icolas reales. \cite{Yehoshua2023}

% subsubsection subsubsection name (end)